\documentclass[a4paper,12pt]{article}

\input{packages}
\input{styles}

\title{Harrix Data 1.0}
\author{А.\,Б. Сергиенко}
\date{\today}


\begin{document}

\input{names}

\maketitle

\begin{abstract}
\textbf{Harrix Data 1.0} --- формат файлов вида \textbf{*.hdata} для представления данных для обработки и отображения в графиках в программах Harrix.
\end{abstract}

\tableofcontents

\newpage

\section{Вводная информация}

Библиотека распространяется по лицензии Apache License, Version 2.0.

Описание данного формата файлов располагается по адресу \href {https://github.com/Harrix/HarrixFileFormats} {https://github.com/Harrix/HarrixFileFormats}.

С автором можно связаться по адресу \href {mailto:sergienkoanton@mail.ru} {sergienkoanton@mail.ru} или  \href {http://vk.com/harrix} {http://vk.com/harrix}. Сайт автора, где публикуются последние новости: \href {http://blog.harrix.org} {http://blog.harrix.org}, а проекты располагаются по адресу \href {http://harrix.org} {http://harrix.org}.


\section{Краткое описание формата данных}

Файл формата \textbf{Harrix Data 1.0} имеет расширение вида \textbf{*.hdata}.

Файл представляет собой обычный текстовой файл, где информация располагается по строкам. Вначале файла идет служебная информация, а потом идут непосредственно данные.

В качестве разделителя для числовых данных используется точка, а не запятая.

\section{Пример файла *.hdata}

Обратите внимание, что если Вы будете копировать примеры данных между строчками \textbf{BeginData} и \textbf{EndData} из данного документа, то между числами вместо знака табуляции будет знак пробела. Это неправильно! Так что копируйте из файлов, которые находятся на сайте \href {https://github.com/Harrix/HarrixFileFormats} {https://github.com/Harrix/HarrixFileFormats} или меняйте пробел на знак табуляции.

\begin{lstlisting}[label=Example01,caption=Пример файла *.hdata]
HarrixFileFormat = Harrix Data 1.0
Site = https://github.com/Harrix/HarrixFileFormats
Type = TwoIndependentChartsOfPointsAndLine
Title = Тестовый график
AxisX = x
AxisX = y
Parameters = ShowLine, ShowPoints, ShowArea, ShowSpecPoints
BeginNamesOfCharts
Количество деревьев
Количество домов
EndNamesOfCharts
BeginData
-0.807560891	-0.94366779	-3.14	-0.001592653
0.00127521	-0.128120962	-3.04	-0.101417986
-2.489437639	-0.611951554	-2.94	-0.200229985
0.445603028	0.599178044	-2.84	-0.297041351
1.586889318	1.085454093	-2.74	-0.390884779
0.746497053	0.566555897	-2.64	-0.480822615
2.568177686	0.617068194	-2.54	-0.56595623
EndData
\end{lstlisting}

\section{Структура файла}

Вначале в файле в первых двух строчках идет служебная информация, которая описывает формат файла.

\begin{itemize}
\item \textbf{Harrix File Format}.
В первой строчке содержится название формата файла:
\begin{lstlisting}[label=Line01,caption=Первая строчка файла *.hdata]
HarrixFileFormat = Harrix Data 1.0
\end{lstlisting}
Она должна содержаться в каждом файле и не подлежит изменению.

\item \textbf{Site}.
Во второй строчке содержится информация о сайте, на котором содержится данное описание формата файла:
\begin{lstlisting}[label=Line02,caption=Вторая строчка файла *.hdata]
Site = https://github.com/Harrix/HarrixFileFormats
\end{lstlisting}
Она должна содержаться в каждом файле и не подлежит изменению.
\end{itemize}

Далее идут строчки описывающие данные, которые содержатся в файле. Данные строчки могут отсутствовать.

\begin{itemize}
\item \textbf{Type}.
В следующей строчке написан формат данных, содержащихся в данном файле.
\begin{lstlisting}[label=Line03,caption=Строчка с описанием типа данных в файле]
Type = TwoIndependentChartsOfPointsAndLine
\end{lstlisting}
Данный параметр может принимать следующие значения:

\begin{itemize}
\item \textbf{Line};
\item \textbf{TwoLines};
\end{itemize}

Рассмотрим их более подробно.

\begin{itemize}
\item \textbf{Line} --- файл содержат информацию о массиве точек с координатами x и y. То есть на графике выведется одна линия по точкам. Массив точек не должен быть отсортирован по какой-нибудь координате.

В блоке между строчками \textbf{BeginNamesOfCharts} и \textbf{EndNamesOfCharts} должна содержаться только одна строчка, которая содержит название линии:
\begin{lstlisting}[label=Line03_2,caption=Для Line ]
BeginNamesOfCharts
Название линии
EndNamesOfCharts
\end{lstlisting}

Пример данных, которые содержатся в между строчками \textbf{BeginData} и \textbf{EndData} (с этими строчками включительно):
\begin{lstlisting}[label=Line03_3,caption=Для Line ]
BeginData
3	0.111111
4	0.0666667
5	0.04
6	0.0285714
7	0.0204082
8	0.015873
9	0.0123457
10	0.010101
EndData
\end{lstlisting}

\item \textbf{TwoLines} --- файл содержат информацию о двух линиях с одинаковыми значениями по оси Ox. То есть на графике выведется две линии по точкам. Массив точек не должен быть отсортирован по какой-нибудь координате.

В блоке между строчками \textbf{BeginNamesOfCharts} и \textbf{EndNamesOfCharts} должна содержаться две строчки, которая содержит название линии:
\begin{lstlisting}[label=Line03_4,caption=Для Line ]
BeginNamesOfCharts
Количество домов
Количество участков
EndNamesOfCharts
\end{lstlisting}

Пример данных, которые содержатся в между строчками \textbf{BeginData} и \textbf{EndData} (с этими строчками включительно):
\begin{lstlisting}[label=Line03_5,caption=Для Line ]
BeginData
3	5.111111	9
7	7.0666667	-1
5	4.04	12
6	2.0285714	5
7.1	8.0204082	8
8	0.015873	4
9	6.0123457	9
5.5	4.010101	1
11	15.00826446	7
EndData
\end{lstlisting}

\item \textbf{TwoLines} --- файл содержат информацию о двух линиях с одинаковыми значениями по оси Ox. То есть на графике выведется две линии по точкам. Массив точек не должен быть отсортирован по какой-нибудь координате.

В блоке между строчками \textbf{BeginNamesOfCharts} и \textbf{EndNamesOfCharts} должна содержаться две строчки, которая содержит название линии:
\begin{lstlisting}[label=Line03_4,caption=Для Line ]
BeginNamesOfCharts
Количество домов
Количество участков
EndNamesOfCharts
\end{lstlisting}

Пример данных, которые содержатся в между строчками \textbf{BeginData} и \textbf{EndData} (с этими строчками включительно):
\begin{lstlisting}[label=Line03_5,caption=Для Line ]
BeginData
3	5.111111	9
7	7.0666667	-1
5	4.04	12
6	2.0285714	5
7.1	8.0204082	8
8	0.015873	4
9	6.0123457	9
5.5	4.010101	1
11	15.00826446	7
EndData
\end{lstlisting}

\end{itemize}

\end{itemize}

Потом идет блок, который обрамляется строчками \textbf{BeginNamesOfCharts} и \textbf{EndNamesOfCharts}.

Наконец, идет блок, который обрамляется строчками \textbf{BeginData} и \textbf{EndData}. В данном блоке идут непосредственно данные. Данные располагаются в виде стандартного представления строк и столбцов. То есть каждая строка описывает одну точку или несколько точек (для некоторых типов данных). Каждый столбец содержит данные одного типа. Все столбцы разделяются знаком табуляции. Если в какой-то ячейке нет данных (например, данные содержат данные о двух линиях с разным количеством точек), то на месте пропуска ставится знак минуса <<->>.

\section{Функции, которые обрабатывают данный формат файлов}

В библиотеке \href {https://github.com/Harrix/QtHarrixLibrary} {https://github.com/Harrix/QtHarrixLibrary} на языке С++ имеются функции, которые обрабатывают данный формат файлов с среде Qt. К таким функциям относятся:

\end{document}
