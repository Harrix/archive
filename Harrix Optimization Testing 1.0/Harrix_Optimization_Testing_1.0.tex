\documentclass[a4paper,12pt]{article}

\input{packages}
\input{styles}

\title{Harrix Optimization Testing 1.0}
\author{А.\,Б. Сергиенко}
\date{\today}


\begin{document}

\input{names}

\maketitle

\begin{abstract}
\textbf{Harrix Optimization Testing 1.0} --- формат файлов вида \textbf{*.xml} для представления данных об исследовании эффективности алгоритмов оптимизации на тестовых функциях.
\end{abstract}

\tableofcontents

\newpage

\section{Вводная информация}

Описание данного формата файлов располагается по адресу \href {https://github.com/Harrix/HarrixFileFormats} {https://github.com/Harrix/HarrixFileFormats}.

С автором можно связаться по адресу \href {mailto:sergienkoanton@mail.ru} {sergienkoanton@mail.ru} или  \href {http://vk.com/harrix} {http://vk.com/harrix}. Сайт автора, где публикуются последние новости: \href {http://blog.harrix.org} {http://blog.harrix.org}, а проекты располагаются по адресу \href {http://harrix.org} {http://harrix.org}.


\section{Краткое описание формата данных}

Файл формата \textbf{Harrix Optimization Testing 1.0} имеет расширение вида \textbf{*.xml}.

Файл представляет собой обычный файл формата XML. Вначале файла идет служебная информация, а потом идут непосредственно данные об эффективности алгоритма.

\section{Пример файла Harrix Optimization Testing}

Предложенный ниже файл не является полным исследованием алгоритма, а является лишь тестовым примером.

\begin{lstlisting}[label=Example01,caption=Пример части файла Harrix Optimization Testing]
<?xml version="1.0" encoding="UTF-8"?>
<document>
<harrix_file_format>
	<format>Harrix Optimization Testing</format>
	<version>1.0</version>
	<link>https://github.com/Harrix/HarrixFileFormats</link>
</harrix_file_format>
<about>
	<author>Сергиенко Антон Борисович</author>
	<date>12.08.2013 23:17:24</date>
</about>
<about_data>
	<!-- Обозначение алгоритма (по названию функции, которая его реализует) -->
	<name_algorithm>MHL_StandartRealGeneticAlgorithm</name_algorithm>
	<!-- Полное название алгоритма -->
	<full_name_algorithm>Стандартный генетический алгоритм на вещественных строках</full_name_algorithm>
	<!-- Ссылка на описание алгоритма оптимизации (если нет, то NULL) -->
	<link_algorithm>https://github.com/Harrix/HarrixOptimizationAlgorithms</link_algorithm>
	<!-- Название тестовой функции (по названию функции, которая его реализует) -->
	<name_test_function>MHL_TestFunction_Ackley</name_test_function>
	<!-- Полное название тестовой функции -->
	<full_name_test_function>Функция Ackley</full_name_test_function>
	<!-- Ссылка на описание тестовой функции (если нет, то NULL) -->
	<link_test_function>https://github.com/Harrix/HarrixTestFunctions</link_test_function>
	<!-- Размерность задачи оптимизации -->
	<chromosome_length>5</chromosome_length>
	<!-- Количество измерений для каждого варианта настроек алгоритма (сколько точек получим) -->
	<number_of_measuring>10</number_of_measuring>
	<!-- Количество запусков алгоритма в каждом из экспериментов -->
	<number_of_runs>10</number_of_runs>
	<!-- Максимальное допустимое число вычислений целевой функции -->
	<max_count_of_fitness>2500</max_count_of_fitness>
	<!-- Количество проверяемых параметров алгоритма оптимизации -->
	<number_of_parameters>5</number_of_parameters>
	<!-- Количество комбинаций вариантов настроек -->
	<number_of_experiments>1</number_of_experiments>
</about_data>
<data>
	<experiment parameters_of_algorithm_1="Тип селекции = Пропорциональная селекция" parameters_of_algorithm_2="Тип скрещивания = Одноточечное скрещивание" parameters_of_algorithm_3="Тип мутации = Слабая мутация" parameters_of_algorithm_4="Тип формирования нового поколения = Только потомки" parameters_of_algorithm_5="Тип преобразования задачи вещественной оптимизации в задачу бинарной оптимизации = Стандартное представление целого числа - номер узла в сетке дискретизации">
		<measuring number="1">
			<Ex>0.102733</Ex>
			<Ey>1.40394</Ey>
			<R>0</R>
		</measuring>
		<measuring number="2">
			<Ex>0.0840828</Ex>
			<Ey>1.4134</Ey>
			<R>0</R>
		</measuring>
		<measuring number="3">
			<Ex>0.0674963</Ex>
			<Ey>1.20694</Ey>
			<R>0</R>
		</measuring>
		<measuring number="4">
			<Ex>0.103118</Ex>
			<Ey>1.57915</Ey>
			<R>0</R>
		</measuring>
		<measuring number="5">
			<Ex>0.0795264</Ex>
			<Ey>1.4047</Ey>
			<R>0</R>
		</measuring>
		<measuring number="6">
			<Ex>0.0626839</Ex>
			<Ey>1.17213</Ey>
			<R>0</R>
		</measuring>
		<measuring number="7">
			<Ex>0.0974347</Ex>
			<Ey>1.46336</Ey>
			<R>0</R>
		</measuring>
		<measuring number="8">
			<Ex>0.10858</Ex>
			<Ey>1.26652</Ey>
			<R>0</R>
		</measuring>
		<measuring number="9">
			<Ex>0.0990866</Ex>
			<Ey>1.41937</Ey>
			<R>0</R>
		</measuring>
		<measuring number="10">
			<Ex>0.0901381</Ex>
			<Ey>1.17268</Ey>
			<R>0.1</R>
		</measuring>
	</experiment>
</data>
</document>
\end{lstlisting}

\section{Подробное описание формата данных}

Файл имеет строгую структуру данных, которую не следует нарушать. Все тэги являются обязательными, на те или иные параметры накладываются ограничения, которые будут ниже описаны.



\section{Функции, которые обрабатывают данный формат файлов}

В библиотеке \href{https://github.com/Harrix/DataOfHarrixOptimizationTesting} {https://github.com/Harrix/DataOfHarrixOptimizationTesting} на языке С++ имеется класс \textbf{DataOfHarrixOptimizationTesting}, который парсит и анализирует данный формат файлов с среде Qt.

\end{document}
