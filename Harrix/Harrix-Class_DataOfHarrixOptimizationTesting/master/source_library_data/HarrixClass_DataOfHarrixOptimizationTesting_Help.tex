\documentclass[a4paper,12pt]{article}

%%% HarrixLaTeXDocumentTemplate
%%% Версия 1.22
%%% Шаблон документов в LaTeX на русском языке. Данный шаблон применяется в проектах HarrixTestFunctions, MathHarrixLibrary, Standard-Genetic-Algorithm  и др.
%%% https://github.com/Harrix/HarrixLaTeXDocumentTemplate
%%% Шаблон распространяется по лицензии Apache License, Version 2.0.

%%% Проверка используемого TeX-движка %%%
\usepackage{ifxetex}

%%% Поля и разметка страницы %%%
\usepackage{lscape} % Для включения альбомных страниц
\usepackage{geometry} % Для последующего задания полей

%%% Кодировки и шрифты %%%
\ifxetex
\usepackage{polyglossia} % Поддержка многоязычности
\usepackage{fontspec} % TrueType-шрифты
\else
\usepackage{cmap}  % Улучшенный поиск русских слов в полученном pdf-файле
\usepackage[T2A]{fontenc} % Поддержка русских букв
\usepackage[utf8]{inputenc} % Кодировка utf8
\usepackage[english, russian]{babel} % Языки: русский, английский
\IfFileExists{pscyr.sty}{\usepackage{pscyr}}{} % Красивые русские шрифты
\fi

%%% Математические пакеты %%%
\usepackage{amsthm,amsfonts,amsmath,amssymb,amscd} % Математические дополнения от AMS
% Для жиного курсива в формулах %
\usepackage{bm}
% Для рисования некоторых математических символов (например, закрашенных треугольников)
\usepackage{mathabx}

%%% Оформление абзацев %%%
\usepackage{indentfirst} % Красная строка
\usepackage{setspace} % Расстояние между строками
\usepackage{enumitem} % Для список обнуление расстояния до абзаца

%%% Цвета %%%
\usepackage[usenames]{color}
\usepackage{color}
\usepackage{colortbl}

%%% Таблицы %%%
\usepackage{longtable} % Длинные таблицы
\usepackage{multirow,makecell,array} % Улучшенное форматирование таблиц

%%% Общее форматирование
\usepackage[singlelinecheck=off,center]{caption} % Многострочные подписи
\usepackage{soul} % Поддержка переносоустойчивых подчёркиваний и зачёркиваний
\usepackage{icomma} % Запятая в десятичных дробях

%%% Библиография %%%
\usepackage{cite}

%%% Гиперссылки %%%
\usepackage{hyperref}

%%% Изображения %%%
\usepackage{graphicx} % Подключаем пакет работы с графикой
\usepackage{epstopdf}
\usepackage{subcaption}

%%% Оглавление %%%
\usepackage{tocloft}

%%% Колонтитулы %%%
\usepackage{fancyhdr}

%%% Отображение кода %%%
\usepackage{xcolor}
\usepackage{listings}
\usepackage{caption}

%%% Псевдокоды %%%
\usepackage{algorithm} 
\usepackage{algpseudocode}

%%% Рисование графиков %%%
\usepackage{pgfplots}

%%% HarrixLaTeXDocumentTemplate
%%% Версия 1.22
%%% Шаблон документов в LaTeX на русском языке. Данный шаблон применяется в проектах HarrixTestFunctions, MathHarrixLibrary, Standard-Genetic-Algorithm  и др.
%%% https://github.com/Harrix/HarrixLaTeXDocumentTemplate
%%% Шаблон распространяется по лицензии Apache License, Version 2.0.

%%% Макет страницы %%%
% Выставляем значения полей (ГОСТ 7.0.11-2011, 5.3.7)
\geometry{a4paper,top=2cm,bottom=2cm,left=2.5cm,right=1cm}

%%% Выравнивание и переносы %%%
\sloppy % Избавляемся от переполнений
\clubpenalty=10000 % Запрещаем разрыв страницы после первой строки абзаца
\widowpenalty=10000 % Запрещаем разрыв страницы после последней строки абзаца


%%% Библиография %%%
\makeatletter
\bibliographystyle{utf8gost71u}  % Оформляем библиографию по ГОСТ 7.1 (ГОСТ Р 7.0.11-2011, 5.6.7)
\renewcommand{\@biblabel}[1]{#1.} % Заменяем библиографию с квадратных скобок на точку
\makeatother

%%% Изображения %%%
\graphicspath{{images/}} % Пути к изображениям
% Поменять двоеточние на точку в подписях к рисунку
\RequirePackage{caption}
\DeclareCaptionLabelSeparator{defffis}{. }
\captionsetup{justification=centering,labelsep=defffis}

%%% Абзацы %%%
% Отсупы между строками
\singlespacing
\setlength{\parskip}{0.3cm} % отступы между абзацами
\linespread{1.3} % Полуторный интвервал (ГОСТ Р 7.0.11-2011, 5.3.6)

% Оформление списков
\setlist{leftmargin=1.5cm,topsep=0pt}
% Используем дефис для ненумерованных списков (ГОСТ 2.105-95, 4.1.7)
\renewcommand{\labelitemi}{\normalfont\bfseries{--}}

%%% Цвета %%%
% Цвета для кода
\definecolor{string}{HTML}{B40000} % цвет строк в коде
\definecolor{comment}{HTML}{008000} % цвет комментариев в коде
\definecolor{keyword}{HTML}{1A00FF} % цвет ключевых слов в коде
\definecolor{morecomment}{HTML}{8000FF} % цвет include и других элементов в коде
\definecolor{сaptiontext}{HTML}{FFFFFF} % цвет текста заголовка в коде
\definecolor{сaptionbk}{HTML}{999999} % цвет фона заголовка в коде
\definecolor{bk}{HTML}{FFFFFF} % цвет фона в коде
\definecolor{frame}{HTML}{999999} % цвет рамки в коде
\definecolor{brackets}{HTML}{B40000} % цвет скобок в коде
% Цвета для гиперссылок
\definecolor{linkcolor}{HTML}{799B03} % цвет ссылок
\definecolor{urlcolor}{HTML}{799B03} % цвет гиперссылок
\definecolor{citecolor}{HTML}{799B03} % цвет гиперссылок
\definecolor{gray}{rgb}{0.4,0.4,0.4}
\definecolor{tableheadcolor}{HTML}{E5E5E5} % цвет шапки в таблицах
\definecolor{darkblue}{rgb}{0.0,0.0,0.6}
% Цвета для графиков
\definecolor{plotcoordinate}{HTML}{88969C}% цвет точек на координатых осях (минимум и максимум)
\definecolor{plotgrid}{HTML}{ECECEC} % цвет сетки
\definecolor{plotmain}{HTML}{97BBCD} % цвет основного графика
\definecolor{plotsecond}{HTML}{FF0000} % цвет второго графика, если графика только два
\definecolor{plotsecondgray}{HTML}{CCCCCC} % цвет второго графика, если графика только два. В сером виде.
\definecolor{darkgreen}{HTML}{799B03} % цвет темно-зеленого

%%% Отображение кода %%%
% Настройки отображения кода
\lstset{
language=C++, % Язык кода по умолчанию
morekeywords={*,...}, % если хотите добавить ключевые слова, то добавляйте
% Цвета
keywordstyle=\color{keyword}\ttfamily\bfseries,
%stringstyle=\color{string}\ttfamily,
stringstyle=\ttfamily\color{red!50!brown},
commentstyle=\color{comment}\ttfamily\itshape,
morecomment=[l][\color{morecomment}]{\#}, 
% Настройки отображения     
breaklines=true, % Перенос длинных строк
basicstyle=\ttfamily\footnotesize, % Шрифт для отображения кода
backgroundcolor=\color{bk}, % Цвет фона кода
frame=lrb,xleftmargin=\fboxsep,xrightmargin=-\fboxsep, % Рамка, подогнанная к заголовку
rulecolor=\color{frame}, % Цвет рамки
tabsize=3, % Размер табуляции в пробелах
% Настройка отображения номеров строк. Если не нужно, то удалите весь блок
%numbers=left, % Слева отображаются номера строк
%stepnumber=1, % Каждую строку нумеровать
%numbersep=5pt, % Отступ от кода 
%numberstyle=\small\color{black}, % Стиль написания номеров строк
% Для отображения русского языка
extendedchars=true,
literate={Ö}{{\"O}}1
  {Ä}{{\"A}}1
  {Ü}{{\"U}}1
  {ß}{{\ss}}1
  {ü}{{\"u}}1
  {ä}{{\"a}}1
  {ö}{{\"o}}1
  {~}{{\textasciitilde}}1
  {а}{{\selectfont\char224}}1
  {б}{{\selectfont\char225}}1
  {в}{{\selectfont\char226}}1
  {г}{{\selectfont\char227}}1
  {д}{{\selectfont\char228}}1
  {е}{{\selectfont\char229}}1
  {ё}{{\"e}}1
  {ж}{{\selectfont\char230}}1
  {з}{{\selectfont\char231}}1
  {и}{{\selectfont\char232}}1
  {й}{{\selectfont\char233}}1
  {к}{{\selectfont\char234}}1
  {л}{{\selectfont\char235}}1
  {м}{{\selectfont\char236}}1
  {н}{{\selectfont\char237}}1
  {о}{{\selectfont\char238}}1
  {п}{{\selectfont\char239}}1
  {р}{{\selectfont\char240}}1
  {с}{{\selectfont\char241}}1
  {т}{{\selectfont\char242}}1
  {у}{{\selectfont\char243}}1
  {ф}{{\selectfont\char244}}1
  {х}{{\selectfont\char245}}1
  {ц}{{\selectfont\char246}}1
  {ч}{{\selectfont\char247}}1
  {ш}{{\selectfont\char248}}1
  {щ}{{\selectfont\char249}}1
  {ъ}{{\selectfont\char250}}1
  {ы}{{\selectfont\char251}}1
  {ь}{{\selectfont\char252}}1
  {э}{{\selectfont\char253}}1
  {ю}{{\selectfont\char254}}1
  {я}{{\selectfont\char255}}1
  {А}{{\selectfont\char192}}1
  {Б}{{\selectfont\char193}}1
  {В}{{\selectfont\char194}}1
  {Г}{{\selectfont\char195}}1
  {Д}{{\selectfont\char196}}1
  {Е}{{\selectfont\char197}}1
  {Ё}{{\"E}}1
  {Ж}{{\selectfont\char198}}1
  {З}{{\selectfont\char199}}1
  {И}{{\selectfont\char200}}1
  {Й}{{\selectfont\char201}}1
  {К}{{\selectfont\char202}}1
  {Л}{{\selectfont\char203}}1
  {М}{{\selectfont\char204}}1
  {Н}{{\selectfont\char205}}1
  {О}{{\selectfont\char206}}1
  {П}{{\selectfont\char207}}1
  {Р}{{\selectfont\char208}}1
  {С}{{\selectfont\char209}}1
  {Т}{{\selectfont\char210}}1
  {У}{{\selectfont\char211}}1
  {Ф}{{\selectfont\char212}}1
  {Х}{{\selectfont\char213}}1
  {Ц}{{\selectfont\char214}}1
  {Ч}{{\selectfont\char215}}1
  {Ш}{{\selectfont\char216}}1
  {Щ}{{\selectfont\char217}}1
  {Ъ}{{\selectfont\char218}}1
  {Ы}{{\selectfont\char219}}1
  {Ь}{{\selectfont\char220}}1
  {Э}{{\selectfont\char221}}1
  {Ю}{{\selectfont\char222}}1
  {Я}{{\selectfont\char223}}1
  {і}{{\selectfont\char105}}1
  {ї}{{\selectfont\char168}}1
  {є}{{\selectfont\char185}}1
  {ґ}{{\selectfont\char160}}1
  {І}{{\selectfont\char73}}1
  {Ї}{{\selectfont\char136}}1
  {Є}{{\selectfont\char153}}1
  {Ґ}{{\selectfont\char128}}1
  {\{}{{{\color{brackets}\{}}}1 % Цвет скобок {
  {\}}{{{\color{brackets}\}}}}1 % Цвет скобок }
}
% Для настройки заголовка кода
\DeclareCaptionFont{white}{\color{сaptiontext}}
\DeclareCaptionFormat{listing}{\parbox{\linewidth}{\colorbox{сaptionbk}{\parbox{\linewidth}{#1#2#3}}\vskip-4pt}}
\captionsetup[lstlisting]{format=listing,labelfont=white,textfont=white}
\renewcommand{\lstlistingname}{Код} % Переименование Listings в нужное именование структуры
% Для отображения кода формата xml
\lstdefinelanguage{XML}
{
  morestring=[s]{"}{"},
  morecomment=[s]{?}{?},
  morecomment=[s]{!--}{--},
  commentstyle=\color{comment},
  moredelim=[s][\color{black}]{>}{<},
  moredelim=[s][\color{red}]{\ }{=},
  stringstyle=\color{string},
  identifierstyle=\color{keyword}
}

%%% Гиперссылки %%%
\hypersetup{pdfstartview=FitH,  linkcolor=linkcolor,urlcolor=urlcolor,citecolor=citecolor, colorlinks=true}

%%% Псевдокоды %%%
% Добавляем свои блоки
\makeatletter
\algblock[ALGORITHMBLOCK]{BeginAlgorithm}{EndAlgorithm}
\algblock[BLOCK]{BeginBlock}{EndBlock}
\makeatother

% Нумерация блоков
\usepackage{caption}% http://ctan.org/pkg/caption
\captionsetup[ruled]{labelsep=period}
\makeatletter
\@addtoreset{algorithm}{chapter}% algorithm counter resets every chapter
\makeatother
\renewcommand{\thealgorithm}{\thechapter.\arabic{algorithm}}% Algorithm # is <chapter>.<algorithm>

%%% Формулы %%%
%Дублирование символа при переносе
\newcommand{\hmm}[1]{#1\nobreak\discretionary{}{\hbox{\ensuremath{#1}}}{}}

%%% Таблицы %%%
% Раздвигаем в таблице без границ отступы между строками в новой команде
\newenvironment{tabularwide}%
{\setlength{\extrarowheight}{0.3cm}\begin{tabular}{  p{\dimexpr 0.45\linewidth-2\tabcolsep} p{\dimexpr 0.55\linewidth-2\tabcolsep}  }}  {\end{tabular}}
\newenvironment{tabularwide08}%
{\setlength{\extrarowheight}{0.3cm}\begin{tabular}{  p{\dimexpr 0.8\linewidth-2\tabcolsep} p{\dimexpr 0.2\linewidth-2\tabcolsep}  }}  {\end{tabular}}

% Многострочная ячейка в таблице
\newcommand{\specialcell}[2][c]{%
  {\renewcommand{\arraystretch}{1}\begin{tabular}[t]{@{}l@{}}#2\end{tabular}}}

% Многострочная ячейка, где текст не может выйти за границы
\newcolumntype{P}[1]{>{\raggedright\arraybackslash}p{#1}}
\newcommand{\specialcelltwoin}[2][c]{%
  {\renewcommand{\arraystretch}{1}\begin{tabular}[t]{@{}P{1.98in}@{}}#2\end{tabular}}}
  
% Команда для переворачивания текста в ячейке таблицы на 90 градусов
\newcommand*\rot{\rotatebox{90}}

%%% Рисование графиков %%%
\pgfplotsset{
every axis legend/.append style={at={(0.5,-0.13)},anchor=north,legend cell align=left},
tick label style={font=\tiny\scriptsize},
label style={font=\scriptsize},
legend style={font=\scriptsize},
grid=both,
minor tick num=2,
major grid style={plotgrid},
minor grid style={plotgrid},
axis lines=left,
legend style={draw=none},
/pgf/number format/.cd,
1000 sep={}
}
% Карта цвета для трехмерных графиков в стиле графиков Mathcad
\pgfplotsset{
/pgfplots/colormap={mathcad}{rgb255(0cm)=(76,0,128) rgb255(2cm)=(0,14,147) rgb255(4cm)=(0,173,171) rgb255(6cm)=(32,205,0) rgb255(8cm)=(229,222,0) rgb255(10cm)=(255,13,0)}
}
% Карта цвета для трехмерных графиков в стиле графиков Matlab
\pgfplotsset{
/pgfplots/colormap={matlab}{rgb255(0cm)=(0,0,128) rgb255(1cm)=(0,0,255) rgb255(3cm)=(0,255,255) rgb255(5cm)=(255,255,0) rgb255(7cm)=(255,0,0) rgb255(8cm)=(128,0,0)}
}

%%% Разное %%%
% Галочки для отмечания в тескте вариантов как OK
\def\checkmark{\tikz\fill[black,scale=0.3](0,.35) -- (.25,0) -- (1,.7) -- (.25,.15) -- cycle;}
\def\checkmarkgreen{\tikz\fill[darkgreen,scale=0.3](0,.35) -- (.25,0) -- (1,.7) -- (.25,.15) -- cycle;} 
\def\checkmarkred{\tikz\fill[red,scale=0.3](0,.35) -- (.25,0) -- (1,.7) -- (.25,.15) -- cycle;}
\def\checkmarkbig{\tikz\fill[black,scale=0.5](0,.35) -- (.25,0) -- (1,.7) -- (.25,.15) -- cycle;}
\def\checkmarkbiggreen{\tikz\fill[darkgreen,scale=0.5](0,.35) -- (.25,0) -- (1,.7) -- (.25,.15) -- cycle;} 
\def\checkmarkbigred{\tikz\fill[red,scale=0.5](0,.35) -- (.25,0) -- (1,.7) -- (.25,.15) -- cycle;}

%% Следующие блоки расскоментировать при необходимости

%%% Кодировки и шрифты %%%
%\ifxetex
%\setmainlanguage{russian}
%\setotherlanguage{english}
%\defaultfontfeatures{Ligatures=TeX,Mapping=tex-text}
%\setmainfont{Times New Roman}
%\newfontfamily\cyrillicfont{Times New Roman}
%\setsansfont{Arial}
%\newfontfamily\cyrillicfontsf{Arial}
%\setmonofont{Courier New}
%\newfontfamily\cyrillicfonttt{Courier New}
%\else
%\IfFileExists{pscyr.sty}{\renewcommand{\rmdefault}{ftm}}{}
%\fi

%%% Колонтитулы %%%
% Порядковый номер страницы печатают на середине верхнего поля страницы (ГОСТ Р 7.0.11-2011, 5.3.8)
%\makeatletter
%\let\ps@plain\ps@fancy              % Подчиняем первые страницы каждой главы общим правилам
%\makeatother
%\pagestyle{fancy}                   % Меняем стиль оформления страниц
%\fancyhf{}                          % Очищаем текущие значения
%\fancyhead[C]{\thepage}             % Печатаем номер страницы на середине верхнего поля
%\renewcommand{\headrulewidth}{0pt}  % Убираем разделительную линию

%%% Оглавление %%%
%\renewcommand{\cftchapdotsep}{\cftdotsep}
%\renewcommand{\cftchapleader}{\cftdotfill{\cftdotsep}}
%\renewcommand{\cftsecleader}{\cftdotfill{\cftdotsep}}
%\renewcommand{\cftfigleader}{\cftdotfill{\cftdotsep}}
%\renewcommand{\cfttableader}{\cftdotfill{\cftdotsep}}

\title{Har\-rix\-Class\_Da\-ta\-Of\-Har\-rix\-Op\-ti\-mi\-za\-ti\-on\-Test\-ing - Har\-rix\-Class\_Da\-ta\-Of\-Har\-rix\-Op\-ti\-mi\-za\-ti\-on\-Test\-ing v.1.27}
\author{А.\,Б. Сергиенко}
\date{\today}


\begin{document}

%%% HarrixLaTeXDocumentTemplate
%%% Версия 1.22
%%% Шаблон документов в LaTeX на русском языке. Данный шаблон применяется в проектах HarrixTestFunctions, MathHarrixLibrary, Standard-Genetic-Algorithm  и др.
%%% https://github.com/Harrix/HarrixLaTeXDocumentTemplate
%%% Шаблон распространяется по лицензии Apache License, Version 2.0.

%%% Именования %%%
\renewcommand{\abstractname}{Аннотация}
\renewcommand{\alsoname}{см. также}
\renewcommand{\appendixname}{Приложение} % (ГОСТ Р 7.0.11-2011, 5.7)
\renewcommand{\bibname}{Список литературы} % (ГОСТ Р 7.0.11-2011, 4)
\renewcommand{\ccname}{исх.}
\renewcommand{\chaptername}{Глава}
\renewcommand{\contentsname}{Оглавление} % (ГОСТ Р 7.0.11-2011, 4)
\renewcommand{\enclname}{вкл.}
\renewcommand{\figurename}{Рисунок} % (ГОСТ Р 7.0.11-2011, 5.3.9)
\renewcommand{\headtoname}{вх.}
\renewcommand{\indexname}{Предметный указатель}
\renewcommand{\listfigurename}{Список рисунков}
\renewcommand{\listtablename}{Список таблиц}
\renewcommand{\pagename}{Стр.}
\renewcommand{\partname}{Часть}
\renewcommand{\refname}{Список литературы} % (ГОСТ Р 7.0.11-2011, 4)
\renewcommand{\seename}{см.}
\renewcommand{\tablename}{Таблица} % (ГОСТ Р 7.0.11-2011, 5.3.10)

%%% Псевдокоды %%%
% Перевод данных об алгоритмах
\renewcommand{\listalgorithmname}{Список алгоритмов}
\floatname{algorithm}{Алгоритм}

% Перевод команд псевдокода
\algrenewcommand\algorithmicwhile{\textbf{До тех пока}}
\algrenewcommand\algorithmicdo{\textbf{выполнять}}
\algrenewcommand\algorithmicrepeat{\textbf{Повторять}}
\algrenewcommand\algorithmicuntil{\textbf{Пока выполняется}}
\algrenewcommand\algorithmicend{\textbf{Конец}}
\algrenewcommand\algorithmicif{\textbf{Если}}
\algrenewcommand\algorithmicelse{\textbf{иначе}}
\algrenewcommand\algorithmicthen{\textbf{тогда}}
\algrenewcommand\algorithmicfor{\textbf{Цикл. }}
\algrenewcommand\algorithmicforall{\textbf{Выполнить для всех}}
\algrenewcommand\algorithmicfunction{\textbf{Функция}}
\algrenewcommand\algorithmicprocedure{\textbf{Процедура}}
\algrenewcommand\algorithmicloop{\textbf{Зациклить}}
\algrenewcommand\algorithmicrequire{\textbf{Условия:}}
\algrenewcommand\algorithmicensure{\textbf{Обеспечивающие условия:}}
\algrenewcommand\algorithmicreturn{\textbf{Возвратить}}
\algrenewtext{EndWhile}{\textbf{Конец цикла}}
\algrenewtext{EndLoop}{\textbf{Конец зацикливания}}
\algrenewtext{EndFor}{\textbf{Конец цикла}}
\algrenewtext{EndFunction}{\textbf{Конец функции}}
\algrenewtext{EndProcedure}{\textbf{Конец процедуры}}
\algrenewtext{EndIf}{\textbf{Конец условия}}
\algrenewtext{EndFor}{\textbf{Конец цикла}}
\algrenewtext{BeginAlgorithm}{\textbf{Начало алгоритма}}
\algrenewtext{EndAlgorithm}{\textbf{Конец алгоритма}}
\algrenewtext{BeginBlock}{\textbf{Начало блока. }}
\algrenewtext{EndBlock}{\textbf{Конец блока}}
\algrenewtext{ElsIf}{\textbf{иначе если }}

\maketitle

\begin{abstract}
Класс HarrixClass\_DataOfHarrixOptimizationTesting для считывания информации формата данных Harrix Optimization Testing на C++ для Qt. Рассматривается HarrixClass\_DataOfHarrixOptimizationTesting.cpp.
\end{abstract}

\tableofcontents

\newpage

\section{Введение}

Класс HarrixClass\_DataOfHarrixOptimizationTesting для считывания информации формата данных Harrix Optimization Testing на C++ для Qt.

Последнюю версию документа можно найти по адресу:

\href{https://github.com/Harrix/HarrixClass\_DataOfHarrixOptimizationTesting}{https://github.com/Harrix/HarrixClass\_DataOfHarrixOptimizationTesting}

Об установке библиотеки можно прочитать тут:

\href{http://blog.harrix.org/?p=992}{http://blog.harrix.org/?p=992}

С автором можно связаться по адресу \href{mailto:sergienkoanton@mail.ru}{sergienkoanton@mail.ru} или  \href{http://vk.com/harrix}{http://vk.com/harrix}.

Сайт автора, где публикуются последние новости: \href{http://blog.harrix.org/}{http://blog.harrix.org/}, а проекты располагаются по адресу \href{http://harrix.org/}{http://harrix.org/}.

%%%%%%%%%%%%%%%%%%%%%%%%%%%%%%%%%%%%%%%%%%%%%%%%%%%%%%%%%% ВСТАВЛЯТЬ НИЖЕ
\newpage
\section{Список функций}\label{section_listfunctions}
\textbf{\_Конструкторы и деструкторы}
\begin{enumerate}

\item \textbf{\hyperref[HarrixClass_DataOfHarrixOptimizationTesting]{HarrixClass\_DataOfHarrixOptimizationTesting}} --- Конструктор. Функция считывает данные о тестировании алгоритма оптимизации из файла формата HarrixOptimizationTesting. Во второй реализации это конструктор., который создает пустой экземпляр.

\item \textbf{\hyperref[tildaHarrixClass_DataOfHarrixOptimizationTesting]{$\sim$HarrixClass\_DataOfHarrixOptimizationTesting}} --- Деструктор класса.

\end{enumerate}

\textbf{Возвращение HTML данных из класса}
\begin{enumerate}

\item \textbf{\hyperref[getHtml]{getHtml}} --- Получение текста переменной Html. Это итоговый Html документ. Помните, что это не полноценный Html код. Его нужно применять в виде temp.html для макета:      https://github.com/Harrix/QtHarrixLibraryForQWebView.

\item \textbf{\hyperref[getHtmlMessageOfError]{getHtmlMessageOfError}} --- Получение текста переменной HtmlMessageOfError. Это часть html документа в виде кода о сообщениях ошибок. Помните, что это не полноценный Html код. Его нужно применять в виде temp.html для макета: https://github.com/Harrix/QtHarrixLibraryForQWebView.

\item \textbf{\hyperref[getHtmlReport]{getHtmlReport}} --- Получение текста переменной HtmlReport. Это часть html документа в виде отчета о проделанной работе. Помните, что это не полноценный Html код. Его нужно применять в виде temp.html для макета: https://github.com/Harrix/QtHarrixLibraryForQWebView.

\end{enumerate}

\textbf{Возвращение LaTeX данных из класса}
\begin{enumerate}

\item \textbf{\hyperref[getFullLatex]{getFullLatex}} --- Получение текста переменной Latex в полном составе с началом и концовкой в Latex файле. Здесь собран полный файл анализа данных из считываемого xml файла. Это полноценный Latex код. Его нужно применять с файлами из макета: https://github.com/Harrix/Harrix-Document-Template-LaTeX.

\item \textbf{\hyperref[getFullLatexAboutParameters]{getFullLatexAboutParameters}} --- Получение текста переменной LatexAboutParameters --- отображение данных о обнаруженных параметрах алгоритма и какие они бывают с началом и концовкой в Latex файле.

\item \textbf{\hyperref[getFullLatexAnalysis]{getFullLatexAnalysis}} --- Получение текста переменной LatexAnalysis --- отображение данных первоначального анализа данных.

\item \textbf{\hyperref[getFullLatexInfo]{getFullLatexInfo}} --- Получение текста переменной LatexInfo --- отображение информации о исследовании с началом и концовкой в Latex файле.

\item \textbf{\hyperref[getFullLatexTable]{getFullLatexTable}} --- Получение текста переменной LatexTable в полном составе с началом и концовкой в Latex файле.

\item \textbf{\hyperref[getFullLatexTableEx]{getFullLatexTableEx}} --- Получение текста переменной LatexTableEx --- отображение сырых данных таблицы данных о ошибке Ex с началом и концовкой в Latex файле.

\item \textbf{\hyperref[getFullLatexTableEy]{getFullLatexTableEy}} --- Получение текста переменной LatexTableEy --- отображение сырых данных таблицы данных о ошибке Ey с началом и концовкой в Latex файле.

\item \textbf{\hyperref[getFullLatexTableR]{getFullLatexTableR}} --- Получение текста переменной LatexTableR --- отображение сырых данных по надежности в виде полной таблицы с началом и концовкой в Latex файле.

\item \textbf{\hyperref[getLatex]{getLatex}} --- Получение текста переменной Latex. Здесь собран полный файл анализа данных из считываемого xml файла. Помните, что это не полноценный Latex код. Его нужно применять внутри файла из макета: https://github.com/Harrix/Harrix-Document-Template-LaTeX.

\item \textbf{\hyperref[getLatexAboutParameters]{getLatexAboutParameters}} --- Получение текста переменной LatexAboutParameters --- отображение данных о обнаруженных параметрах алгоритма и какие они бывают.

\item \textbf{\hyperref[getLatexAnalysis]{getLatexAnalysis}} --- Получение текста переменной LatexAnalysis --- отображение первоначального анализа данных.

\item \textbf{\hyperref[getLatexInfo]{getLatexInfo}} --- Получение текста переменной LatexInfo --- отображение информации о исследовании.

\item \textbf{\hyperref[getLatexTable]{getLatexTable}} --- Получение текста переменной LatexTable.

\item \textbf{\hyperref[getLatexTableEx]{getLatexTableEx}} --- Получение текста переменной LatexTableEx --- отображение сырых данных таблицы данных о ошибке Ex.

\item \textbf{\hyperref[getLatexTableEy]{getLatexTableEy}} --- Получение текста переменной LatexTableEy --- отображение сырых данных ошибки по значениям целевой функции в виде полной таблицы.

\item \textbf{\hyperref[getLatexTableR]{getLatexTableR}} --- Получение текста переменной LatexTableR --- отображение сырых данных по надежности в виде полной таблицы.

\end{enumerate}

\textbf{Возвращение переменных из класса}
\begin{enumerate}

\item \textbf{\hyperref[getAuthor]{getAuthor}} --- Получение текста переменной XML\_Author --- Автор документа.

\item \textbf{\hyperref[getCheckAllCombinations]{getCheckAllCombinations}} --- Получение текста переменной  XML\_All\_Combinations --- Все ли комбинации вариантов настроек просмотрены: 0 или 1.

\item \textbf{\hyperref[getData]{getData}} --- Возвращает содержимое исследований в виде экземпляра класса.

\item \textbf{\hyperref[getDate]{getDate}} --- Получение текста переменной  XML\_Date --- Дата создания документа.

\item \textbf{\hyperref[getDimensionTestFunction]{getDimensionTestFunction}} --- Получение текста переменной  XML\_DimensionTestFunction --- Размерность тестовой задачи.

\item \textbf{\hyperref[getEmail]{getEmail}} --- Получение текста переменной  XML\_Email --- Email автора, чтобы можно было с ним связаться

\item \textbf{\hyperref[getErrorEx]{getErrorEx}} --- Получение значения ошибки Ex.

\item \textbf{\hyperref[getErrorEy]{getErrorEy}} --- Получение значения ошибки Ey.

\item \textbf{\hyperref[getErrorR]{getErrorR}} --- Получение значения надежности R.

\item \textbf{\hyperref[getFormat]{getFormat}} --- Получение переменной XML\_Format, то есть возвращает название формата документа.

\item \textbf{\hyperref[getFullNameAlgorithm]{getFullNameAlgorithm}} --- Получение текста переменной  XML\_Full\_Name\_Algorithm --- Полное название алгоритма оптимизации.

\item \textbf{\hyperref[getFullNameTestFunction]{getFullNameTestFunction}} --- Получение текста переменной  XML\_Full\_Name\_Test\_Function --- Полное название тестовой функции.

\item \textbf{\hyperref[getLink]{getLink}} --- Получение переменной XML\_Link, то есть возвращает ссылку на описание формата файла.

\item \textbf{\hyperref[getLinkAlgorithm]{getLinkAlgorithm}} --- Получение текста переменной  XML\_Link\_Algorithm --- Ссылка на описание алгоритма оптимизации.

\item \textbf{\hyperref[getLinkTestFunction]{getLinkTestFunction}} --- Получение текста переменной  XML\_Link\_Test\_Function --- Ссылка на описание тестовой функции.

\item \textbf{\hyperref[getMaxCountOfFitness]{getMaxCountOfFitness}} --- Получение текста переменной  Max\_Count\_Of\_Fitness --- Максимальное допустимое число вычислений целевой функции для алгоритма.

\item \textbf{\hyperref[getMeanEx]{getMeanEx}} --- Получение среднего значения ошибки Ex по измерениям для настройки (сколько точек было по столько и усредняем).

\item \textbf{\hyperref[getMeanEy]{getMeanEy}} --- Получение среднего значения ошибки Ey по измерениям для настройки (сколько точек было по столько и усредняем).

\item \textbf{\hyperref[getMeanR]{getMeanR}} --- Получение среднего значения надежности R по измерениям для настройки (сколько точек было по столько и усредняем).

\item \textbf{\hyperref[getNameAlgorithm]{getNameAlgorithm}} --- Получение текста переменной  XML\_Name\_Algorithm --- Название алгоритма оптимизации.

\item \textbf{\hyperref[getNameParameter]{getNameParameter}} --- Получение значения параметра настройки какой-то в виде полного наименования.

\item \textbf{\hyperref[getNameTestFunction]{getNameTestFunction}} --- Получение текста переменной  XML\_Name\_Test\_Function --- Название тестовой функции.

\item \textbf{\hyperref[getNumberOfExperiments]{getNumberOfExperiments}} --- Получение текста переменной  XML\_Number\_Of\_Experiments --- Количество комбинаций вариантов настроек.

\item \textbf{\hyperref[getNumberOfMeasuring]{getNumberOfMeasuring}} --- Получение текста переменной  XML\_Number\_Of\_Measuring --- Размерность тестовой задачи (длина хромосомы решения).

\item \textbf{\hyperref[getNumberOfOption]{getNumberOfOption}} --- Получение номера параметра алгоритма по его имени.

\item \textbf{\hyperref[getNumberOfParameters]{getNumberOfParameters}} --- Получение текста переменной  XML\_Number\_Of\_Parameters --- Количество проверяемых параметров алгоритма оптимизации

\item \textbf{\hyperref[getNumberOfRuns]{getNumberOfRuns}} --- Получение текста переменной  XML\_Number\_Of\_Runs --- Количество прогонов, по которому делается усреднение для эксперимента.

\item \textbf{\hyperref[getParameter]{getParameter}} --- Получение значения параметра настройки какой-то.

\item \textbf{\hyperref[getSuccessReading]{getSuccessReading}} --- Получение значения переменной SuccessReading о удачности или не удачности считывания файла.

\item \textbf{\hyperref[getVarianceOfEx]{getVarianceOfEx}} --- Получение дисперсии значения ошибки Ex по измерениям для настройки (сколько точек было по столько и усредняем).

\item \textbf{\hyperref[getVarianceOfEy]{getVarianceOfEy}} --- Получение дисперсии значения ошибки Ey по измерениям для настройки (сколько точек было по столько и усредняем).

\item \textbf{\hyperref[getVarianceOfR]{getVarianceOfR}} --- Получение дисперсии значения надежности R по измерениям для настройки (сколько точек было по столько и усредняем).

\item \textbf{\hyperref[getVersion]{getVersion}} --- Получение переменной XML\_Version, то есть возвращает версию формата документа.

\end{enumerate}

\textbf{Операторы}
\begin{enumerate}

\item \textbf{\hyperref[operator =]{operator =}} --- Оператор присваивания.

\end{enumerate}

\textbf{Специализированные функции}
\begin{enumerate}

\item \textbf{\hyperref[makingLatexTable2D]{makingLatexTable2D}} --- Создает текст LaTeX в виде таблицы 2D для всех экспериментов для отображения какой-нибудь информации.

\end{enumerate}

\textbf{Внутренние функции}
\begin{enumerate}

\item \textbf{\hyperref[checkXmlLeafTags]{checkXmlLeafTags}} --- Проверяет наличие тэгов и правильное их выполнение. Внутренняя функция. Учитывает все "листовые" тэги кроме тэгов данных.

\item \textbf{\hyperref[getLatexBegin]{getLatexBegin}} --- Внутренняя функция. Возвращает начало для полноценного Latex файла.

\item \textbf{\hyperref[getLatexEnd]{getLatexEnd}} --- Внутренняя функция. Возвращает концовку для полноценного Latex файла.

\item \textbf{\hyperref[initializationOfVariables]{initializationOfVariables}} --- Обнуление переменных. Внутренняя функция.

\item \textbf{\hyperref[makingListOfVectorParameterOptions]{makingListOfVectorParameterOptions}} --- Заполняет список вектора названий вариантов параметров алгоритма оптимизации.

\item \textbf{\hyperref[memoryAllocation]{memoryAllocation}} --- Выделяет память под необходимые массивы. Внутренняя функция.

\item \textbf{\hyperref[memoryDeallocation]{memoryDeallocation}} --- Удаляет память из-под массивов. Внутренняя функция.

\item \textbf{\hyperref[readXml]{readXml}} --- Считывание XML файла и осуществление всех остальных анализов и др.

\item \textbf{\hyperref[readXmlDataTags]{readXmlDataTags}} --- Считывает и проверяет тэги данных. Внутренняя функция. Учитывает все "листовые" тэги кроме тэгов данных.

\item \textbf{\hyperref[readXmlLeafTag]{readXmlLeafTag}} --- Считывает и проверяет тэг, который должен являться "листом", то есть самым глубоким. Внутренняя функция. Учитывает все "листовые" тэги кроме тэгов данных.

\item \textbf{\hyperref[readXmlTreeTag]{readXmlTreeTag}} --- Считывает и проверяет тэг, который содержит внутри себя другие тэги. Внутренняя функция.

\item \textbf{\hyperref[zeroArray]{zeroArray}} --- Обнуляет массивы, в котрые записывается информация о данных из файла. Внутренняя функция.

\end{enumerate}

\textbf{Создание содержимого отчетов LaTeX и HTML}
\begin{enumerate}

\item \textbf{\hyperref[makingHtmlReport]{makingHtmlReport}} --- Создает текст Html для отображения отчета о считывании XML файла.

\item \textbf{\hyperref[makingLatexAboutParameters]{makingLatexAboutParameters}} --- Создает текст LaTeX для отображения данных о обнаруженных параметрах алгоритма и какие они бывают.

\item \textbf{\hyperref[makingLatexAnalysis]{makingLatexAnalysis}} --- Создает текст LaTeX для отображения первоначального анализа данных.

\item \textbf{\hyperref[makingLatexInfo]{makingLatexInfo}} --- Создает текст LaTeX для отображения информации о исследовании.

\item \textbf{\hyperref[makingLatexListOfVectorParameterOptions]{makingLatexListOfVectorParameterOptions}} --- Создает текст LaTeX для отображения списка вектора названий вариантов параметров алгоритма оптимизации.

\item \textbf{\hyperref[makingLatexListOfVectorParameterOptions2]{makingLatexListOfVectorParameterOptions2}} --- Создает текст LaTeX для отображения списка вектора названий вариантов параметров алгоритма оптимизации.

\item \textbf{\hyperref[makingLatexTableEx]{makingLatexTableEx}} --- Создает текст LaTeX для отображения сырых данных ошибки по входным параметрам в виде полной таблицы.

\item \textbf{\hyperref[makingLatexTableEy]{makingLatexTableEy}} --- Создает текст LaTeX для отображения сырых данных ошибки по значениям целевой функции в виде полной таблицы.

\item \textbf{\hyperref[makingLatexTableR]{makingLatexTableR}} --- Создает текст LaTeX для отображения сырых данных по надежности в виде полной таблицы.

\end{enumerate}

\textbf{Функции анализа данных}
\begin{enumerate}

\item \textbf{\hyperref[makingAnalysis]{makingAnalysis}} --- Выполняет анализ считанных данных. Внутренняя функция.

\end{enumerate}


\newpage
\section{Функции}
\subsection{\_Конструкторы и деструкторы}

\subsubsection{HarrixClass\_DataOfHarrixOptimizationTesting}\label{HarrixClass_DataOfHarrixOptimizationTesting}

Конструктор. Функция считывает данные о тестировании алгоритма оптимизации из файла формата HarrixOptimizationTesting. Во второй реализации это конструктор., который создает пустой экземпляр.


\begin{lstlisting}[label=code_syntax_HarrixClass_DataOfHarrixOptimizationTesting,caption=Синтаксис]
HarrixClass_DataOfHarrixOptimizationTesting(QString filename);
HarrixClass_DataOfHarrixOptimizationTesting();
\end{lstlisting}

\textbf{Входные параметры:}

filename --- полное имя считываемого файла.

\textbf{Возвращаемое значение:}

Отсутствует.


\subsubsection{$\sim$HarrixClass\_DataOfHarrixOptimizationTesting}\label{tildaHarrixClass_DataOfHarrixOptimizationTesting}

Деструктор класса.


\begin{lstlisting}[label=code_syntax_tildaHarrixClass_DataOfHarrixOptimizationTesting,caption=Синтаксис]
~HarrixClass_DataOfHarrixOptimizationTesting();
\end{lstlisting}

\textbf{Входные параметры:}

Отсутствуют.

\textbf{Возвращаемое значение:}

Отсутствует.


\subsection{Возвращение HTML данных из класса}

\subsubsection{getHtml}\label{getHtml}

Получение текста переменной Html. Это итоговый Html документ. Помните, что это не полноценный Html код. Его нужно применять в виде temp.html для макета:      https://github.com/Harrix/QtHarrixLibraryForQWebView.


\begin{lstlisting}[label=code_syntax_getHtml,caption=Синтаксис]
QString getHtml();
\end{lstlisting}

\textbf{Входные параметры:}

Отсутствуют.

\textbf{Возвращаемое значение:}

Итоговый Html документ.


\subsubsection{getHtmlMessageOfError}\label{getHtmlMessageOfError}

Получение текста переменной HtmlMessageOfError. Это часть html документа в виде кода о сообщениях ошибок. Помните, что это не полноценный Html код. Его нужно применять в виде temp.html для макета: https://github.com/Harrix/QtHarrixLibraryForQWebView.


\begin{lstlisting}[label=code_syntax_getHtmlMessageOfError,caption=Синтаксис]
QString getHtmlMessageOfError();
\end{lstlisting}

\textbf{Входные параметры:}

Отсутствуют.

\textbf{Возвращаемое значение:}

Часть html документа в виде кода о сообщениях ошибок.


\subsubsection{getHtmlReport}\label{getHtmlReport}

Получение текста переменной HtmlReport. Это часть html документа в виде отчета о проделанной работе. Помните, что это не полноценный Html код. Его нужно применять в виде temp.html для макета: https://github.com/Harrix/QtHarrixLibraryForQWebView.


\begin{lstlisting}[label=code_syntax_getHtmlReport,caption=Синтаксис]
QString getHtmlReport();
\end{lstlisting}

\textbf{Входные параметры:}

Отсутствуют.

\textbf{Возвращаемое значение:}

Часть html документа в виде отчета о проделанной работе.


\subsection{Возвращение LaTeX данных из класса}

\subsubsection{getFullLatex}\label{getFullLatex}

Получение текста переменной Latex в полном составе с началом и концовкой в Latex файле. Здесь собран полный файл анализа данных из считываемого xml файла. Это полноценный Latex код. Его нужно применять с файлами из макета: https://github.com/Harrix/Harrix-Document-Template-LaTeX.


\begin{lstlisting}[label=code_syntax_getFullLatex,caption=Синтаксис]
QString getFullLatex();
\end{lstlisting}

\textbf{Входные параметры:}

Отсутствуют.

\textbf{Возвращаемое значение:}

Полный файл анализа данных из считываемого xml файла.


\subsubsection{getFullLatexAboutParameters}\label{getFullLatexAboutParameters}

Получение текста переменной LatexAboutParameters --- отображение данных о обнаруженных параметрах алгоритма и какие они бывают с началом и концовкой в Latex файле.


\begin{lstlisting}[label=code_syntax_getFullLatexAboutParameters,caption=Синтаксис]
QString getFullLatexAboutParameters();
\end{lstlisting}

\textbf{Входные параметры:}

Отсутствуют.

\textbf{Возвращаемое значение:}

Полный переменной LatexAboutParameters --- отображение данных о обнаруженных параметрах алгоритма и какие они бывают с началом и концовкой в Latex файле.


\subsubsection{getFullLatexAnalysis}\label{getFullLatexAnalysis}

Получение текста переменной LatexAnalysis --- отображение данных первоначального анализа данных.


\begin{lstlisting}[label=code_syntax_getFullLatexAnalysis,caption=Синтаксис]
QString getFullLatexAnalysis();
\end{lstlisting}

\textbf{Входные параметры:}

Отсутствуют.

\textbf{Возвращаемое значение:}

Текст переменной LatexAnalysis --- отображение данных первоначального анализа данных.


\subsubsection{getFullLatexInfo}\label{getFullLatexInfo}

Получение текста переменной LatexInfo --- отображение информации о исследовании с началом и концовкой в Latex файле.


\begin{lstlisting}[label=code_syntax_getFullLatexInfo,caption=Синтаксис]
QString getFullLatexInfo();
\end{lstlisting}

\textbf{Входные параметры:}

Отсутствуют.

\textbf{Возвращаемое значение:}

Текст переменной LatexInfo --- отображение информации о исследовании с началом и концовкой в Latex файле.


\subsubsection{getFullLatexTable}\label{getFullLatexTable}

Получение текста переменной LatexTable в полном составе с началом и концовкой в Latex файле.


\begin{lstlisting}[label=code_syntax_getFullLatexTable,caption=Синтаксис]
QString getFullLatexTable();
\end{lstlisting}

\textbf{Входные параметры:}

Отсутствуют.

\textbf{Возвращаемое значение:}

Полный файл первичных данных из считываемого xml файла (без анализа).


\subsubsection{getFullLatexTableEx}\label{getFullLatexTableEx}

Получение текста переменной LatexTableEx --- отображение сырых данных таблицы данных о ошибке Ex с началом и концовкой в Latex файле.


\begin{lstlisting}[label=code_syntax_getFullLatexTableEx,caption=Синтаксис]
QString getFullLatexTableEx();
\end{lstlisting}

\textbf{Входные параметры:}

Отсутствуют.

\textbf{Возвращаемое значение:}

Текст переменной LatexTableEx --- отображение сырых данных таблицы данных о ошибке Ex с началом и концовкой в Latex файле.


\subsubsection{getFullLatexTableEy}\label{getFullLatexTableEy}

Получение текста переменной LatexTableEy --- отображение сырых данных таблицы данных о ошибке Ey с началом и концовкой в Latex файле.


\begin{lstlisting}[label=code_syntax_getFullLatexTableEy,caption=Синтаксис]
QString getFullLatexTableEy();
\end{lstlisting}

\textbf{Входные параметры:}

Отсутствуют.

\textbf{Возвращаемое значение:}

Текст переменной LatexTableEy --- отображение сырых данных таблицы данных о ошибке Ey с началом и концовкой в Latex файле.


\subsubsection{getFullLatexTableR}\label{getFullLatexTableR}

Получение текста переменной LatexTableR --- отображение сырых данных по надежности в виде полной таблицы с началом и концовкой в Latex файле.


\begin{lstlisting}[label=code_syntax_getFullLatexTableR,caption=Синтаксис]
QString getFullLatexTableR();
\end{lstlisting}

\textbf{Входные параметры:}

Отсутствуют.

\textbf{Возвращаемое значение:}

Текст переменной LatexTableR --- отображение сырых данных по надежности в виде полной таблицы с началом и концовкой в Latex файле.


\subsubsection{getLatex}\label{getLatex}

Получение текста переменной Latex. Здесь собран полный файл анализа данных из считываемого xml файла. Помните, что это не полноценный Latex код. Его нужно применять внутри файла из макета: https://github.com/Harrix/Harrix-Document-Template-LaTeX.


\begin{lstlisting}[label=code_syntax_getLatex,caption=Синтаксис]
QString getLatex();
\end{lstlisting}

\textbf{Входные параметры:}

Отсутствуют.

\textbf{Возвращаемое значение:}

Полный файл анализа данных из считываемого xml файла.


\subsubsection{getLatexAboutParameters}\label{getLatexAboutParameters}

Получение текста переменной LatexAboutParameters --- отображение данных о обнаруженных параметрах алгоритма и какие они бывают.


\begin{lstlisting}[label=code_syntax_getLatexAboutParameters,caption=Синтаксис]
QString getLatexAboutParameters();
\end{lstlisting}

\textbf{Входные параметры:}

Отсутствуют.

\textbf{Возвращаемое значение:}

Текст переменной LatexAboutParameters --- отображение данных о обнаруженных параметрах алгоритма и какие они бывают.


\subsubsection{getLatexAnalysis}\label{getLatexAnalysis}

Получение текста переменной LatexAnalysis --- отображение первоначального анализа данных.


\begin{lstlisting}[label=code_syntax_getLatexAnalysis,caption=Синтаксис]
QString getLatexAnalysis();
\end{lstlisting}

\textbf{Входные параметры:}

Отсутствуют.

\textbf{Возвращаемое значение:}

Полный переменной LatexAnalysis --- отображение первоначального анализа данных.


\subsubsection{getLatexInfo}\label{getLatexInfo}

Получение текста переменной LatexInfo --- отображение информации о исследовании.


\begin{lstlisting}[label=code_syntax_getLatexInfo,caption=Синтаксис]
QString getLatexInfo();
\end{lstlisting}

\textbf{Входные параметры:}

Отсутствуют.

\textbf{Возвращаемое значение:}

Текст переменной LatexInfo --- отображение информации о исследовании.


\subsubsection{getLatexTable}\label{getLatexTable}

Получение текста переменной LatexTable.


\begin{lstlisting}[label=code_syntax_getLatexTable,caption=Синтаксис]
QString getLatexTable();
\end{lstlisting}

\textbf{Входные параметры:}

Отсутствуют.

\textbf{Возвращаемое значение:}

Полный файл первичных данных из считываемого xml файла (без анализа).


\subsubsection{getLatexTableEx}\label{getLatexTableEx}

Получение текста переменной LatexTableEx --- отображение сырых данных таблицы данных о ошибке Ex.


\begin{lstlisting}[label=code_syntax_getLatexTableEx,caption=Синтаксис]
QString getLatexTableEx();
\end{lstlisting}

\textbf{Входные параметры:}

Отсутствуют.

\textbf{Возвращаемое значение:}

Текст переменной LatexTableEx --- отображение сырых данных таблицы данных о ошибке Ex.


\subsubsection{getLatexTableEy}\label{getLatexTableEy}

Получение текста переменной LatexTableEy --- отображение сырых данных ошибки по значениям целевой функции в виде полной таблицы.


\begin{lstlisting}[label=code_syntax_getLatexTableEy,caption=Синтаксис]
QString getLatexTableEy();
\end{lstlisting}

\textbf{Входные параметры:}

Отсутствуют.

\textbf{Возвращаемое значение:}

Текст переменной LatexTableEy --- отображение сырых данных таблицы данных о ошибке Ey.


\subsubsection{getLatexTableR}\label{getLatexTableR}

Получение текста переменной LatexTableR --- отображение сырых данных по надежности в виде полной таблицы.


\begin{lstlisting}[label=code_syntax_getLatexTableR,caption=Синтаксис]
QString getLatexTableR();
\end{lstlisting}

\textbf{Входные параметры:}

Отсутствуют.

\textbf{Возвращаемое значение:}

Текст переменной LatexTableR --- отображение сырых данных по надежности в виде полной таблицы.


\subsection{Возвращение переменных из класса}

\subsubsection{getAuthor}\label{getAuthor}

Получение текста переменной XML\_Author --- Автор документа.


\begin{lstlisting}[label=code_syntax_getAuthor,caption=Синтаксис]
QString getAuthor();
\end{lstlisting}

\textbf{Входные параметры:}

Отсутствуют.

\textbf{Возвращаемое значение:}

Значение переменной из описания.


\subsubsection{getCheckAllCombinations}\label{getCheckAllCombinations}

Получение текста переменной  XML\_All\_Combinations --- Все ли комбинации вариантов настроек просмотрены: 0 или 1.


\begin{lstlisting}[label=code_syntax_getCheckAllCombinations,caption=Синтаксис]
bool getCheckAllCombinations();
\end{lstlisting}

\textbf{Входные параметры:}

Отсутствуют.

\textbf{Возвращаемое значение:}

Значение переменной из описания.


\subsubsection{getData}\label{getData}

Возвращает содержимое исследований в виде экземпляра класса.


\begin{lstlisting}[label=code_syntax_getData,caption=Синтаксис]
HarrixClass_OnlyDataOfHarrixOptimizationTesting& getData();
\end{lstlisting}

\textbf{Входные параметры:}

Отсутствуют.

\textbf{Возвращаемое значение:}

Содержимое исследований в виде экземпляра класса.


\subsubsection{getDate}\label{getDate}

Получение текста переменной  XML\_Date --- Дата создания документа.


\begin{lstlisting}[label=code_syntax_getDate,caption=Синтаксис]
QString getDate();
\end{lstlisting}

\textbf{Входные параметры:}

Отсутствуют.

\textbf{Возвращаемое значение:}

Значение переменной из описания.


\subsubsection{getDimensionTestFunction}\label{getDimensionTestFunction}

Получение текста переменной  XML\_DimensionTestFunction --- Размерность тестовой задачи.


\begin{lstlisting}[label=code_syntax_getDimensionTestFunction,caption=Синтаксис]
qint64 getDimensionTestFunction();
\end{lstlisting}

\textbf{Входные параметры:}

Отсутствуют.

\textbf{Возвращаемое значение:}

Значение переменной из описания.


\subsubsection{getEmail}\label{getEmail}

Получение текста переменной  XML\_Email --- Email автора, чтобы можно было с ним связаться


\begin{lstlisting}[label=code_syntax_getEmail,caption=Синтаксис]
QString getEmail();
\end{lstlisting}

\textbf{Входные параметры:}

Отсутствуют.

\textbf{Возвращаемое значение:}

Значение переменной из описания.


\subsubsection{getErrorEx}\label{getErrorEx}

Получение значения ошибки Ex.


\begin{lstlisting}[label=code_syntax_getErrorEx,caption=Синтаксис]
double getErrorEx(int Number_Of_Experiment, int Number_Of_Measuring);
\end{lstlisting}

\textbf{Входные параметры:}

Number\_Of\_Experiment --- номер комбинации вариантов настроек;
 
    Number\_Of\_Measuring --- номер измерения варианта настроек.

\textbf{Возвращаемое значение:}

Значения ошибки Ex.


\subsubsection{getErrorEy}\label{getErrorEy}

Получение значения ошибки Ey.


\begin{lstlisting}[label=code_syntax_getErrorEy,caption=Синтаксис]
double getErrorEy(int Number_Of_Experiment, int Number_Of_Measuring);
\end{lstlisting}

\textbf{Входные параметры:}

Number\_Of\_Experiment --- номер комбинации вариантов настроек;
 
    Number\_Of\_Measuring --- номер измерения варианта настроек.

\textbf{Возвращаемое значение:}

Значения ошибки Ey.


\subsubsection{getErrorR}\label{getErrorR}

Получение значения надежности R.


\begin{lstlisting}[label=code_syntax_getErrorR,caption=Синтаксис]
double getErrorR(int Number_Of_Experiment, int Number_Of_Measuring);
\end{lstlisting}

\textbf{Входные параметры:}

Number\_Of\_Experiment --- номер комбинации вариантов настроек;
 
    Number\_Of\_Measuring --- номер измерения варианта настроек.

\textbf{Возвращаемое значение:}

Значения надежности R.


\subsubsection{getFormat}\label{getFormat}

Получение переменной XML\_Format, то есть возвращает название формата документа.


\begin{lstlisting}[label=code_syntax_getFormat,caption=Синтаксис]
QString getFormat();
\end{lstlisting}

\textbf{Входные параметры:}

Отсутствует.

\textbf{Возвращаемое значение:}

Если документ без ошибок в описании формата, то всегда должно возвращаться "Harrix Optimization Testing".


\subsubsection{getFullNameAlgorithm}\label{getFullNameAlgorithm}

Получение текста переменной  XML\_Full\_Name\_Algorithm --- Полное название алгоритма оптимизации.


\begin{lstlisting}[label=code_syntax_getFullNameAlgorithm,caption=Синтаксис]
QString getFullNameAlgorithm();
\end{lstlisting}

\textbf{Входные параметры:}

Отсутствуют.

\textbf{Возвращаемое значение:}

Значение переменной из описания.


\subsubsection{getFullNameTestFunction}\label{getFullNameTestFunction}

Получение текста переменной  XML\_Full\_Name\_Test\_Function --- Полное название тестовой функции.


\begin{lstlisting}[label=code_syntax_getFullNameTestFunction,caption=Синтаксис]
QString getFullNameTestFunction();
\end{lstlisting}

\textbf{Входные параметры:}

Отсутствуют.

\textbf{Возвращаемое значение:}

Значение переменной из описания.


\subsubsection{getLink}\label{getLink}

Получение переменной XML\_Link, то есть возвращает ссылку на описание формата файла.


\begin{lstlisting}[label=code_syntax_getLink,caption=Синтаксис]
QString getLink();
\end{lstlisting}

\textbf{Входные параметры:}

Отсутствует.

\textbf{Возвращаемое значение:}

Если документ без ошибок в описании формата, то всегда должно возвращаться "https://github.com/Harrix/HarrixFileFormats".


\subsubsection{getLinkAlgorithm}\label{getLinkAlgorithm}

Получение текста переменной  XML\_Link\_Algorithm --- Ссылка на описание алгоритма оптимизации.


\begin{lstlisting}[label=code_syntax_getLinkAlgorithm,caption=Синтаксис]
QString getLinkAlgorithm();
\end{lstlisting}

\textbf{Входные параметры:}

Отсутствуют.

\textbf{Возвращаемое значение:}

Значение переменной из описания.


\subsubsection{getLinkTestFunction}\label{getLinkTestFunction}

Получение текста переменной  XML\_Link\_Test\_Function --- Ссылка на описание тестовой функции.


\begin{lstlisting}[label=code_syntax_getLinkTestFunction,caption=Синтаксис]
QString getLinkTestFunction();
\end{lstlisting}

\textbf{Входные параметры:}

Отсутствуют.

\textbf{Возвращаемое значение:}

Значение переменной из описания.


\subsubsection{getMaxCountOfFitness}\label{getMaxCountOfFitness}

Получение текста переменной  Max\_Count\_Of\_Fitness --- Максимальное допустимое число вычислений целевой функции для алгоритма.


\begin{lstlisting}[label=code_syntax_getMaxCountOfFitness,caption=Синтаксис]
qint64 getMaxCountOfFitness();
\end{lstlisting}

\textbf{Входные параметры:}

Отсутствуют.

\textbf{Возвращаемое значение:}

Значение переменной из описания.


\subsubsection{getMeanEx}\label{getMeanEx}

Получение среднего значения ошибки Ex по измерениям для настройки (сколько точек было по столько и усредняем).


\begin{lstlisting}[label=code_syntax_getMeanEx,caption=Синтаксис]
double getMeanEx(int Number_Of_Experiment);
\end{lstlisting}

\textbf{Входные параметры:}

Number\_Of\_Experiment --- номер комбинации вариантов настроек.

\textbf{Возвращаемое значение:}

Значения среднего значения Ex.


\subsubsection{getMeanEy}\label{getMeanEy}

Получение среднего значения ошибки Ey по измерениям для настройки (сколько точек было по столько и усредняем).


\begin{lstlisting}[label=code_syntax_getMeanEy,caption=Синтаксис]
double getMeanEy(int Number_Of_Experiment);
\end{lstlisting}

\textbf{Входные параметры:}

Number\_Of\_Experiment --- номер комбинации вариантов настроек.

\textbf{Возвращаемое значение:}

Значения среднего значения Ey.


\subsubsection{getMeanR}\label{getMeanR}

Получение среднего значения надежности R по измерениям для настройки (сколько точек было по столько и усредняем).


\begin{lstlisting}[label=code_syntax_getMeanR,caption=Синтаксис]
double getMeanR(int Number_Of_Experiment);
\end{lstlisting}

\textbf{Входные параметры:}

Number\_Of\_Experiment --- номер комбинации вариантов настроек.

\textbf{Возвращаемое значение:}

Значения среднего значения R.


\subsubsection{getNameAlgorithm}\label{getNameAlgorithm}

Получение текста переменной  XML\_Name\_Algorithm --- Название алгоритма оптимизации.


\begin{lstlisting}[label=code_syntax_getNameAlgorithm,caption=Синтаксис]
QString getNameAlgorithm();
\end{lstlisting}

\textbf{Входные параметры:}

Отсутствуют.

\textbf{Возвращаемое значение:}

Значение переменной из описания.


\subsubsection{getNameParameter}\label{getNameParameter}

Получение значения параметра настройки какой-то в виде полного наименования.


\begin{lstlisting}[label=code_syntax_getNameParameter,caption=Синтаксис]
QString getNameParameter(int Number_Of_Experiment, int Number_Of_Parameter);
\end{lstlisting}

\textbf{Входные параметры:}

Number\_Of\_Experiment --- номер комбинации вариантов настроек;
 
Number\_Of\_Parameter --- номер параметра.

\textbf{Возвращаемое значение:}

Значения параметра в виде наименования.


\subsubsection{getNameTestFunction}\label{getNameTestFunction}

Получение текста переменной  XML\_Name\_Test\_Function --- Название тестовой функции.


\begin{lstlisting}[label=code_syntax_getNameTestFunction,caption=Синтаксис]
QString getNameTestFunction();
\end{lstlisting}

\textbf{Входные параметры:}

Отсутствуют.

\textbf{Возвращаемое значение:}

Значение переменной из описания.


\subsubsection{getNumberOfExperiments}\label{getNumberOfExperiments}

Получение текста переменной  XML\_Number\_Of\_Experiments --- Количество комбинаций вариантов настроек.


\begin{lstlisting}[label=code_syntax_getNumberOfExperiments,caption=Синтаксис]
qint64 getNumberOfExperiments();
\end{lstlisting}

\textbf{Входные параметры:}

Отсутствуют.

\textbf{Возвращаемое значение:}

Значение переменной из описания.


\subsubsection{getNumberOfMeasuring}\label{getNumberOfMeasuring}

Получение текста переменной  XML\_Number\_Of\_Measuring --- Размерность тестовой задачи (длина хромосомы решения).


\begin{lstlisting}[label=code_syntax_getNumberOfMeasuring,caption=Синтаксис]
qint64 getNumberOfMeasuring();
\end{lstlisting}

\textbf{Входные параметры:}

Отсутствуют.

\textbf{Возвращаемое значение:}

Значение переменной из описания.


\subsubsection{getNumberOfOption}\label{getNumberOfOption}

Получение номера параметра алгоритма по его имени.


\begin{lstlisting}[label=code_syntax_getNumberOfOption,caption=Синтаксис]
int getNumberOfOption(QString NameOption);
\end{lstlisting}

\textbf{Входные параметры:}

NameOption --- имя параметра.

\textbf{Возвращаемое значение:}

Значения параметра в виде номера (если не найдено, то возвращается -1.


\subsubsection{getNumberOfParameters}\label{getNumberOfParameters}

Получение текста переменной  XML\_Number\_Of\_Parameters --- Количество проверяемых параметров алгоритма оптимизации


\begin{lstlisting}[label=code_syntax_getNumberOfParameters,caption=Синтаксис]
qint64 getNumberOfParameters();
\end{lstlisting}

\textbf{Входные параметры:}

Отсутствуют.

\textbf{Возвращаемое значение:}

Значение переменной из описания.


\subsubsection{getNumberOfRuns}\label{getNumberOfRuns}

Получение текста переменной  XML\_Number\_Of\_Runs --- Количество прогонов, по которому делается усреднение для эксперимента.


\begin{lstlisting}[label=code_syntax_getNumberOfRuns,caption=Синтаксис]
qint64 getNumberOfRuns();
\end{lstlisting}

\textbf{Входные параметры:}

Отсутствуют.

\textbf{Возвращаемое значение:}

Значение переменной из описания.


\subsubsection{getParameter}\label{getParameter}

Получение значения параметра настройки какой-то.


\begin{lstlisting}[label=code_syntax_getParameter,caption=Синтаксис]
int getParameter(int Number_Of_Experiment, int Number_Of_Parameter);
\end{lstlisting}

\textbf{Входные параметры:}

Number\_Of\_Experiment --- номер комбинации вариантов настроек;
 
Number\_Of\_Parameter --- номер параметра.

\textbf{Возвращаемое значение:}

Значения параметра в виде числа (соответствие находим в ListOfParameterOptions).


\subsubsection{getSuccessReading}\label{getSuccessReading}

Получение значения переменной SuccessReading о удачности или не удачности считывания файла.


\begin{lstlisting}[label=code_syntax_getSuccessReading,caption=Синтаксис]
bool getSuccessReading();
\end{lstlisting}

\textbf{Входные параметры:}

Отсутствует.

\textbf{Возвращаемое значение:}

Значения переменной SuccessReading о удачности или не удачности считывания файла..


\subsubsection{getVarianceOfEx}\label{getVarianceOfEx}

Получение дисперсии значения ошибки Ex по измерениям для настройки (сколько точек было по столько и усредняем).


\begin{lstlisting}[label=code_syntax_getVarianceOfEx,caption=Синтаксис]
double getVarianceOfEx(int Number_Of_Experiment);
\end{lstlisting}

\textbf{Входные параметры:}

Number\_Of\_Experiment --- номер комбинации вариантов настроек.

\textbf{Возвращаемое значение:}

Значения дисперсии значения Ex.


\subsubsection{getVarianceOfEy}\label{getVarianceOfEy}

Получение дисперсии значения ошибки Ey по измерениям для настройки (сколько точек было по столько и усредняем).


\begin{lstlisting}[label=code_syntax_getVarianceOfEy,caption=Синтаксис]
double getVarianceOfEy(int Number_Of_Experiment);
\end{lstlisting}

\textbf{Входные параметры:}

Number\_Of\_Experiment --- номер комбинации вариантов настроек.

\textbf{Возвращаемое значение:}

Значения дисперсии значения Ey.


\subsubsection{getVarianceOfR}\label{getVarianceOfR}

Получение дисперсии значения надежности R по измерениям для настройки (сколько точек было по столько и усредняем).


\begin{lstlisting}[label=code_syntax_getVarianceOfR,caption=Синтаксис]
double getVarianceOfR(int Number_Of_Experiment);
\end{lstlisting}

\textbf{Входные параметры:}

Number\_Of\_Experiment --- номер комбинации вариантов настроек.

\textbf{Возвращаемое значение:}

Значения дисперсии значения надежности R.


\subsubsection{getVersion}\label{getVersion}

Получение переменной XML\_Version, то есть возвращает версию формата документа.


\begin{lstlisting}[label=code_syntax_getVersion,caption=Синтаксис]
QString getVersion();
\end{lstlisting}

\textbf{Входные параметры:}

Отсутствует.

\textbf{Возвращаемое значение:}

Если документ без ошибок в описании формата, то всегда должно возвращаться "1.0".


\subsection{Операторы}

\subsubsection{operator =}\label{operator =}

Оператор присваивания.


\begin{lstlisting}[label=code_syntax_operator =,caption=Синтаксис]
void operator = (HarrixClass_DataOfHarrixOptimizationTesting& B);
\end{lstlisting}

\textbf{Входные параметры:}

B --- Другой экземпляр класса, который и копируем.

\textbf{Возвращаемое значение:}

Отсутствует.


\subsection{Специализированные функции}

\subsubsection{makingLatexTable2D}\label{makingLatexTable2D}

Создает текст LaTeX в виде таблицы 2D для всех экспериментов для отображения какой-нибудь информации.


\begin{lstlisting}[label=code_syntax_makingLatexTable2D,caption=Синтаксис]
QString makingLatexTable2D(QString Title, QStringList InfoForEveryExperiment);
\end{lstlisting}

\textbf{Входные параметры:}

Title --- заголовок таблицы;

InfoForEvryExperiment --- информация выдаваемая в таблицу.

\textbf{Возвращаемое значение:}

Итоговая таблица в виде кода Latex.


\subsection{Внутренние функции}

\subsubsection{checkXmlLeafTags}\label{checkXmlLeafTags}

Проверяет наличие тэгов и правильное их выполнение. Внутренняя функция. Учитывает все "листовые" тэги кроме тэгов данных.


\begin{lstlisting}[label=code_syntax_checkXmlLeafTags,caption=Синтаксис]
void checkXmlLeafTags();
\end{lstlisting}

\textbf{Входные параметры:}

Отсутствуют.

\textbf{Возвращаемое значение:}

Отсутствует.


\subsubsection{getLatexBegin}\label{getLatexBegin}

Внутренняя функция. Возвращает начало для полноценного Latex файла.


\begin{lstlisting}[label=code_syntax_getLatexBegin,caption=Синтаксис]
QString getLatexBegin();
\end{lstlisting}

\textbf{Входные параметры:}

Отсутствуют.

\textbf{Возвращаемое значение:}

Начало для полноценного Latex файла.


\subsubsection{getLatexEnd}\label{getLatexEnd}

Внутренняя функция. Возвращает концовку для полноценного Latex файла.


\begin{lstlisting}[label=code_syntax_getLatexEnd,caption=Синтаксис]
QString getLatexEnd();
\end{lstlisting}

\textbf{Входные параметры:}

Отсутствуют.

\textbf{Возвращаемое значение:}

Концовка для полноценного Latex файла.


\subsubsection{initializationOfVariables}\label{initializationOfVariables}

Обнуление переменных. Внутренняя функция.


\begin{lstlisting}[label=code_syntax_initializationOfVariables,caption=Синтаксис]
void initializationOfVariables();
\end{lstlisting}

\textbf{Входные параметры:}

Отсутствуют.

\textbf{Возвращаемое значение:}

Отсутствует.


\subsubsection{makingListOfVectorParameterOptions}\label{makingListOfVectorParameterOptions}

Заполняет список вектора названий вариантов параметров алгоритма оптимизации.


\begin{lstlisting}[label=code_syntax_makingListOfVectorParameterOptions,caption=Синтаксис]
void makingListOfVectorParameterOptions();
\end{lstlisting}

\textbf{Входные параметры:}

Отсутствуют.

\textbf{Возвращаемое значение:}

Отсутствует. Значение возвращается в переменную LatexListOfParameterOptions.


\subsubsection{memoryAllocation}\label{memoryAllocation}

Выделяет память под необходимые массивы. Внутренняя функция.


\begin{lstlisting}[label=code_syntax_memoryAllocation,caption=Синтаксис]
void memoryAllocation();
\end{lstlisting}

\textbf{Входные параметры:}

Отсутствуют.

\textbf{Возвращаемое значение:}

Отсутствует.


\subsubsection{memoryDeallocation}\label{memoryDeallocation}

Удаляет память из-под массивов. Внутренняя функция.


\begin{lstlisting}[label=code_syntax_memoryDeallocation,caption=Синтаксис]
void memoryDeallocation();
\end{lstlisting}

\textbf{Входные параметры:}

Отсутствуют.

\textbf{Возвращаемое значение:}

Отсутствует.


\subsubsection{readXml}\label{readXml}

Считывание XML файла и осуществление всех остальных анализов и др.


\begin{lstlisting}[label=code_syntax_readXml,caption=Синтаксис]
void readXml();
\end{lstlisting}

\textbf{Входные параметры:}

Отсутствуют.

\textbf{Возвращаемое значение:}

Отсутствует.


\subsubsection{readXmlDataTags}\label{readXmlDataTags}

Считывает и проверяет тэги данных. Внутренняя функция. Учитывает все "листовые" тэги кроме тэгов данных.


\begin{lstlisting}[label=code_syntax_readXmlDataTags,caption=Синтаксис]
void readXmlDataTags();
\end{lstlisting}

\textbf{Входные параметры:}

Отсутствуют.

\textbf{Возвращаемое значение:}

Отсутствует.


\subsubsection{readXmlLeafTag}\label{readXmlLeafTag}

Считывает и проверяет тэг, который должен являться "листом", то есть самым глубоким. Внутренняя функция. Учитывает все "листовые" тэги кроме тэгов данных.


\begin{lstlisting}[label=code_syntax_readXmlLeafTag,caption=Синтаксис]
void readXmlLeafTag();
\end{lstlisting}

\textbf{Входные параметры:}

Отсутствуют.

\textbf{Возвращаемое значение:}

Отсутствует.


\subsubsection{readXmlTreeTag}\label{readXmlTreeTag}

Считывает и проверяет тэг, который содержит внутри себя другие тэги. Внутренняя функция.


\begin{lstlisting}[label=code_syntax_readXmlTreeTag,caption=Синтаксис]
bool readXmlTreeTag(QString tag);
\end{lstlisting}

\textbf{Входные параметры:}

tag --- какой тэг мы ищем.

\textbf{Возвращаемое значение:}

 
    true --- текущий тэг это тот самый, что нам и нужен;
 
    false --- иначе.


\subsubsection{zeroArray}\label{zeroArray}

Обнуляет массивы, в котрые записывается информация о данных из файла. Внутренняя функция.


\begin{lstlisting}[label=code_syntax_zeroArray,caption=Синтаксис]
void zeroArray();
\end{lstlisting}

\textbf{Входные параметры:}

Отсутствуют.

\textbf{Возвращаемое значение:}

Отсутствует.


\subsection{Создание содержимого отчетов LaTeX и HTML}

\subsubsection{makingHtmlReport}\label{makingHtmlReport}

Создает текст Html для отображения отчета о считывании XML файла.


\begin{lstlisting}[label=code_syntax_makingHtmlReport,caption=Синтаксис]
void makingHtmlReport();
\end{lstlisting}

\textbf{Входные параметры:}

Отсутствуют.

\textbf{Возвращаемое значение:}

Отсутствует. Значение возвращается в переменную HtmlReport, которую можно вызвать getHtmlReport.


\subsubsection{makingLatexAboutParameters}\label{makingLatexAboutParameters}

Создает текст LaTeX для отображения данных о обнаруженных параметрах алгоритма и какие они бывают.


\begin{lstlisting}[label=code_syntax_makingLatexAboutParameters,caption=Синтаксис]
void makingLatexAboutParameters();
\end{lstlisting}

\textbf{Входные параметры:}

Отсутствуют.

\textbf{Возвращаемое значение:}

Отсутствует. Значение возвращается в переменную LatexTableEx, которую можно вызвать getLatexAboutParameters


\subsubsection{makingLatexAnalysis}\label{makingLatexAnalysis}

Создает текст LaTeX для отображения первоначального анализа данных.


\begin{lstlisting}[label=code_syntax_makingLatexAnalysis,caption=Синтаксис]
void makingLatexAnalysis();
\end{lstlisting}

\textbf{Входные параметры:}

Отсутствуют.

\textbf{Возвращаемое значение:}

Отсутствует. Значение возвращается в переменную LatexAnalysis, которую можно вызвать getLatexAnalysis.


\subsubsection{makingLatexInfo}\label{makingLatexInfo}

Создает текст LaTeX для отображения информации о исследовании.


\begin{lstlisting}[label=code_syntax_makingLatexInfo,caption=Синтаксис]
void makingLatexInfo();
\end{lstlisting}

\textbf{Входные параметры:}

Отсутствуют.

\textbf{Возвращаемое значение:}

Отсутствует. Значение возвращается в переменную LatexTableEx, которую можно вызвать getLatexInfo.


\subsubsection{makingLatexListOfVectorParameterOptions}\label{makingLatexListOfVectorParameterOptions}

Создает текст LaTeX для отображения списка вектора названий вариантов параметров алгоритма оптимизации.


\begin{lstlisting}[label=code_syntax_makingLatexListOfVectorParameterOptions,caption=Синтаксис]
void makingLatexListOfVectorParameterOptions();
\end{lstlisting}

\textbf{Входные параметры:}

Отсутствуют.

\textbf{Возвращаемое значение:}

Отсутствует. Значение возвращается в переменную LatexListOfParameterOptions.


\subsubsection{makingLatexListOfVectorParameterOptions2}\label{makingLatexListOfVectorParameterOptions2}

Создает текст LaTeX для отображения списка вектора названий вариантов параметров алгоритма оптимизации.


\begin{lstlisting}[label=code_syntax_makingLatexListOfVectorParameterOptions2,caption=Синтаксис]
void makingLatexListOfVectorParameterOptions2();
\end{lstlisting}

\textbf{Входные параметры:}

Отсутствуют.

\textbf{Возвращаемое значение:}

Отсутствует. Значение возвращается в переменную LatexListOfParameterOptions.


\subsubsection{makingLatexTableEx}\label{makingLatexTableEx}

Создает текст LaTeX для отображения сырых данных ошибки по входным параметрам в виде полной таблицы.


\begin{lstlisting}[label=code_syntax_makingLatexTableEx,caption=Синтаксис]
void makingLatexTableEx();
\end{lstlisting}

\textbf{Входные параметры:}

Отсутствуют.

\textbf{Возвращаемое значение:}

Отсутствует. Значение возвращается в переменную LatexTableEx, которую можно вызвать getLatexTableEx.


\subsubsection{makingLatexTableEy}\label{makingLatexTableEy}

Создает текст LaTeX для отображения сырых данных ошибки по значениям целевой функции в виде полной таблицы.


\begin{lstlisting}[label=code_syntax_makingLatexTableEy,caption=Синтаксис]
void makingLatexTableEy();
\end{lstlisting}

\textbf{Входные параметры:}

Отсутствуют.

\textbf{Возвращаемое значение:}

Отсутствует. Значение возвращается в переменную LatexTableEy, которую можно вызвать getLatexTableEy.


\subsubsection{makingLatexTableR}\label{makingLatexTableR}

Создает текст LaTeX для отображения сырых данных по надежности в виде полной таблицы.


\begin{lstlisting}[label=code_syntax_makingLatexTableR,caption=Синтаксис]
void makingLatexTableR();
\end{lstlisting}

\textbf{Входные параметры:}

Отсутствуют.

\textbf{Возвращаемое значение:}

Отсутствует. Значение возвращается в переменную LatexTableR, которую можно вызвать getLatexTableR.


\subsection{Функции анализа данных}

\subsubsection{makingAnalysis}\label{makingAnalysis}

Выполняет анализ считанных данных. Внутренняя функция.


\begin{lstlisting}[label=code_syntax_makingAnalysis,caption=Синтаксис]
void makingAnalysis();
\end{lstlisting}

\textbf{Входные параметры:}

Отсутствуют.

\textbf{Возвращаемое значение:}

Отсутствует.
%%%%%%%%%%%%%%%%%%%%%%%%%%%%%%%%%%%%%%%%%%%%%%%%%%%%%%%%%%

\end{document}