\textbf{Входные параметры:}

FitnessOfx --- значение целевой функции найденного решения алгоритмом оптимизации;

VHML\_N --- размер массива x.

В переопределяемой функции также есть параметр:
  
Type --- обозначение тестовой функции, которую вызываем.
Смотреть виды в переменных перечисляемого типа в начале HarrixMathLibrary.h файла: TestFunction\_Ackley, TestFunction\_ParaboloidOfRevolution, TestFunction\_Rastrigin и др. Они совпадают с названиями одноименных тестовых функций, но без приставки \textbf{HML\_}.

\textbf{Возвращаемое значение:}
 
Значение ошибки по значениям целевой функции Ey.

Итак, для обычного использования (без параметра Type) нужно вызвать функцию HML\_DefineTestFunction. Иначе использовать переопределенную функцию и самому указать тип тестовой функции.

Конкретную формулу, которые используются для нахождения для каждой тестовой функции, смотрите в функциях этих тестовых функций. Обратите внимание, что данная функция находит ошибку только для одного решения, тогда как по формулам нужно множество решений.

Все функции так высчитываются, чтобы алгоритм решал задачу поиска максимального значения целевой функции, поэтому тестовые функции на минимум умножаются на $-1$. Поэтому, фактически алгоритмы оптимизации находят максимум перевернутой функции. А значит, чтобы правильно посчитать ошибку по значениям целевой функции, нужно найденное решение умножить на $-1$.