\textbf{Входные параметры:}

NumberOfParts --- указатель на массив, куда будет записываться результат;

VHML\_N --- размер массива NumberOfParts.

В переопределяемой функции также есть параметр:
  
Type --- обозначение тестовой функции, которую вызываем.
Смотреть виды в переменных перечисляемого типа в начале HarrixMathLibrary.h файла: TestFunction\_Ackley, TestFunction\_ParaboloidOfRevolution, TestFunction\_Rastrigin и др. Они совпадают с названиями одноименных тестовых функций, но без приставки \textbf{HML\_}.

\textbf{Возвращаемое значение:}
 
Точность вычислений.

Итак, для обычного использования (без параметра Type) нужно вызвать функцию HML\_DefineTestFunction. Иначе использовать переопределенную функцию и самому указать тип тестовой функции.