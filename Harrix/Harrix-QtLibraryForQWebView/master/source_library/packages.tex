%%% HarrixLaTeXDocumentTemplate
%%% Версия 1.22
%%% Шаблон документов в LaTeX на русском языке. Данный шаблон применяется в проектах HarrixTestFunctions, MathHarrixLibrary, Standard-Genetic-Algorithm  и др.
%%% https://github.com/Harrix/HarrixLaTeXDocumentTemplate
%%% Шаблон распространяется по лицензии Apache License, Version 2.0.

%%% Проверка используемого TeX-движка %%%
\usepackage{ifxetex}

%%% Поля и разметка страницы %%%
\usepackage{lscape} % Для включения альбомных страниц
\usepackage{geometry} % Для последующего задания полей

%%% Кодировки и шрифты %%%
\ifxetex
\usepackage{polyglossia} % Поддержка многоязычности
\usepackage{fontspec} % TrueType-шрифты
\else
\usepackage{cmap}  % Улучшенный поиск русских слов в полученном pdf-файле
\usepackage[T2A]{fontenc} % Поддержка русских букв
\usepackage[utf8]{inputenc} % Кодировка utf8
\usepackage[english, russian]{babel} % Языки: русский, английский
\IfFileExists{pscyr.sty}{\usepackage{pscyr}}{} % Красивые русские шрифты
\fi

%%% Математические пакеты %%%
\usepackage{amsthm,amsfonts,amsmath,amssymb,amscd} % Математические дополнения от AMS
% Для жиного курсива в формулах %
\usepackage{bm}
% Для рисования некоторых математических символов (например, закрашенных треугольников)
\usepackage{mathabx}

%%% Оформление абзацев %%%
\usepackage{indentfirst} % Красная строка
\usepackage{setspace} % Расстояние между строками
\usepackage{enumitem} % Для список обнуление расстояния до абзаца

%%% Цвета %%%
\usepackage[usenames]{color}
\usepackage{color}
\usepackage{colortbl}

%%% Таблицы %%%
\usepackage{longtable} % Длинные таблицы
\usepackage{multirow,makecell,array} % Улучшенное форматирование таблиц

%%% Общее форматирование
\usepackage[singlelinecheck=off,center]{caption} % Многострочные подписи
\usepackage{soul} % Поддержка переносоустойчивых подчёркиваний и зачёркиваний
\usepackage{icomma} % Запятая в десятичных дробях

%%% Библиография %%%
\usepackage{cite}

%%% Гиперссылки %%%
\usepackage{hyperref}

%%% Изображения %%%
\usepackage{graphicx} % Подключаем пакет работы с графикой
\usepackage{epstopdf}
\usepackage{subcaption}

%%% Оглавление %%%
\usepackage{tocloft}

%%% Колонтитулы %%%
\usepackage{fancyhdr}

%%% Отображение кода %%%
\usepackage{xcolor}
\usepackage{listings}
\usepackage{caption}

%%% Псевдокоды %%%
\usepackage{algorithm} 
\usepackage{algpseudocode}

%%% Рисование графиков %%%
\usepackage{pgfplots}
