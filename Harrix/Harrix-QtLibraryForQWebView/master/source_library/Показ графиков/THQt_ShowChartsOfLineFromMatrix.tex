\textbf{Входные параметры:}
 
    VHQt\_MatrixXY --- указатель на матрицу значений X и Y графиков;
 
    VHQt\_N --- количество точек;
 
    VHQt\_M --- количество столбцов матрицы (1+количество графиков);
 
    TitleChart --- заголовок графика;
 
    NameVectorX --- название оси Ox;
 
    NameVectorY --- название оси Oy;
 
    NameLine --- указатель на вектор названий графиков (для легенды) количество элементов VHQt\_M---1 (так как первый столбец --- это X значения);
 
    ShowLine --- показывать ли линию;
 
    ShowPoints --- показывать ли точки;
 
    ShowArea --- показывать ли закрашенную область под кривой;
 
    ShowSpecPoints --- показывать ли специальные точки.

\textbf{Возвращаемое значение:}

Строка с HTML кодами с выводимым графиком.

\textbf{Примечание:}

Используются случайные числа, так что рекомендуется вызвать в программе иницилизатор случайных чисел qsrand. Рекомендую так: qsrand(QDateTime::currentMSecsSinceEpoch () % 1000000);

Требует наличия в папке с html файлом файлы jsxgraph.css и jsxgraphcore.js из библиотеки JSXGraph.