\textbf{Входные параметры:}
 
VHQt\_MatrixXY --- указатель на матрицу значений X и Н графиков;
 
VHQt\_N\_EveryCol --- количество элементов в каждом столбце (так как столбцы идут по парам, то число элементов в нечетном и
 
следующем за ним четном столбце должны совпадать, например 10,10,5,5,7,7);
 
VHQt\_M --- количество столбцов матрицы (должно быть четным числом конечно);
 
TitleChart --- заголовок графика;
 
NameVectorX --- название оси Ox;
 
NameVectorY --- название оси Oy;
 
NameLine --- указатель на вектор названий графиков (для легенды) количество элементов VHQt\_M/2;
 
ShowLine --- показывать ли линию;
 
ShowPoints --- показывать ли точки;
 
ShowArea --- показывать ли закрашенную область под кривой;
 
ShowSpecPoints --- показывать ли специальные точки.

\textbf{Возвращаемое значение:}

Строка с HTML кодами с выводимым графиком.

\textbf{Примечание:}

Используются случайные числа, так что рекомендуется вызвать в программе иницилизатор случайных чисел qsrand. Рекомендую так: qsrand(QDateTime::currentMSecsSinceEpoch () % 1000000);

Требует наличия в папке с html файлом файлы jsxgraph.css и jsxgraphcore.js из библиотеки JSXGraph.

Нечетные столбцы - это значения координат X графиков. Следующие за ними четные столбцы - соответствующие значения Y. То есть графики друг от друга независимы.