\documentclass[a4paper,12pt]{article}

\input{packages}
\input{styles}

\title{HarrixMakeLibrary v.1.5}
\author{А.\,Б. Сергиенко}
\date{\today}


\begin{document}

\input{names}

\maketitle

\begin{abstract}
\textbf{HarrixMakeLibrary} --- это программа собирающая библиотеки функций на языке C++ и справку к ним из исходных материалов.
\end{abstract}

\tableofcontents

\newpage

\section{Внешний вид программы}

\begin{figure} [h] 
  \center
  \includegraphics [scale=0.5] {makemainwindow.png}
  \caption{Внешний вид программы} 
  \label{img:latex}  
\end{figure}

При нажатии на кнопку \textbf{<<Собрать библиотеку>>} будет производиться сборка библиотеки вместе с файлами справки. После чего будет открыта папка с сформированными файлами.

В текстовом поле под кнопкой будет отображаться ход работы программы.

Это программа является упрощенной версией программы \textbf{MakeHarrixMathLibrary} из библиотеки \textbf{HarrixMathLibrary}. Помогает собирать файлы библиотек, если каждая функция прописана в отдельных файлах.

\section{Результат работы программы}

В папке \textbf{source\_library} находится исходный материал, который обрабатывается программой HarrixMakeLibrary.exe, в результате чего образуются следующие элементы:

\begin{itemize}
\item \textbf{Library.cpp} --- главный файл библиотеки;
\item \textbf{Library.h} --- заголовочный файл;
\item \textbf{Library\_Help.tex} --- файл справки в формате \LaTeX.
\end{itemize}

Обратите внимание, что это не полноценные файлы исходников библиотеки --- это только сборка функций в соответствующие файлы. И код этих файлов надо вставлять в ваши основные файлы исходных кодов библиотек. В общем, программа генерирует основу, а всякие дополнительные элементы вроде подключение include и др. это вам самим прописывать.

Все данные файлы собираются в папке \textbf{temp\_library}.

\section{Как собирается библиотека}
Исходники библиотеки находятся с папке \textbf{source\_library}.

Файлы \textbf{Library.cpp} и \textbf{Library.h} собираются следующим образом.

В папке \textbf{source\_library} есть директории. Каждая директория --- это множество функций какого-то раздела. Перед рассмотрением файлов папки программа добавляет в файл Library.cpp следующий код:

\begin{lstlisting}[label=make_sectioncpp,caption=Название раздела]
//*****************************************************************
//[Название папки]
//*****************************************************************
\end{lstlisting}

А в файл Library.h добавляется код:

\begin{lstlisting}[label=make_sectionh,caption=Название раздела]
//[Название папки]
\end{lstlisting}

После каждой функции в Library.cpp вставляется код:
\begin{lstlisting}[label=make_sectionh,caption=Название раздела]
//---------------------------------------------------------------------------
\end{lstlisting}

Далее программа пробегает по каждой папке, которая представляет собой раздел функций в библиотеке. Каждая функция в разделе предоставляется следующими файлами:
\begin{itemize}
\item \textbf{<File>.cpp} или \textbf{<File>.tpp} --- код функции;
\item \textbf{<File>.h} --- заголовочный файл функции;
\item \textbf{<File>.tex} --- справка по функции;
\item \textbf{<File>.desc} --- описание функции;
\item \textbf{<File>.use} --- пример использования функции (из него удаляются пробелы в начале строк, равным числу пробелов вначале первой строки);
\item \textbf{<File>\_<name>.pdf} --- множество рисунков, необходимых для справки по функции (необязательные файлы);
\item \textbf{<File>\_<name>.png} --- множество рисунков, необходимых для справки по функции (необязательные файлы);
\end{itemize}

Важно помнить, что каждый *.cpp, *.h, *.tex файл в папках папки \textbf{source\_library} не является полноценным файлом соответствующего расширения и без сборки в единые файлы библиотеки не может использоваться.

Разница файлов *.cpp и *.tpp в том, что в *.tpp  пишется код шаблонов функций, а в *.cpp пишутся обычные функции, и их реализация располагается в Library.cpp файле, тогда как шаблоны располагаются в Library.h файле.

Ниже показан алгоритм (Алгоритм \ref{alg:MakingCppH}) формирования файлов библиотеки.

Итоговое количество функций определяется как количество знаков <<;>> в h файлах функций, которые располагаются в папках.


\begin{algorithm}
\caption{Алгоритм собирания файлов библиотеки} \label{alg:MakingCppH}
\begin{algorithmic}
\State \textbf{Начало алгоритма}
\ForAll {папок}
\State $ Library.cpp+=\text{\textit{Код 1. Название раздела}} $;
\State $ Library.h+=\text{\textit{Код 2. Название раздела}} $;
\ForAll {файлов папки расширения *.cpp, *.tpp и *.h}
\If {есть файл *.cpp}
\State $ Library.cpp+=<File>.cpp $;
\Else
\State $ ResultTpp+=<File>.tpp $;
\EndIf
\State $ Library.h+=<File>.h $;
\EndFor
\EndFor
\State $ Library.h+= ResultTpp$;
\State Сохранить $ Library.cpp $ в папке temp\_library;
\State Сохранить $ Library.h $ в папке temp\_library;
\State \textbf{Конец алгоритма}
\end{algorithmic}
\end{algorithm}

Стоит отметить, что все разделы функций и сами функции сортируются в алфавитном порядке.


\section{Как собирается справка}
Исходники файлов справки библиотеки находятся с папке
\textbf{source\_library}.

Файлы \textbf{Library.tex} собирается следующим образом, как показано ниже в алгоритме (Алгоритм \ref{alg:MakingHelp}) формирования файлов справки библиотеки.

Некоторые моменты по преобразованию некоторых данных (например, преобразование $<File>.desc$) не рассматривается в алгоритме, но вы можете все посмотреть в исходном коде программы, которая поставляется с данной библиотекой в папке \textbf{source\textbackslash}

Также еще собирается файл \textbf{FUNCTIONS.md} со списком функций в формате \textbf{Markdown}, например, для размещения в GitHub.

\section{Исходники HarrixMakeLibrary.exe и справки по ним}

HarrixMakeLibrary написан на Qt.  Не требует каких-то дополнительных файлов. Исходники программы располагаются в папке \textbf{source\textbackslash}.

Исходники справки HarrixMakeLibrary (данного файла, который вы читаете) по располагаются в папке \textbf{source\textbackslash help\textbackslash}. Главный файл исходника справки --- это файл HarrixMakeLibrary\_Help.tex.

\begin{algorithm}
\caption{Алгоритм собирания файлов справки библиотеки} \label{alg:MakingHelp}
\begin{algorithmic}
\State \textbf{Начало алгоритма}
\State $ResultTexList += \text {\textit{Заголовок для списка функций}}$;
\State $ResultTexFunctions += \text {\textit{Заголовок для функций}}$;
\ForAll {папок}
\State $ ResultTexList+=\text{\textit{Заголовок раздела}} $;
\State $ ResultTexFunctions+=\text{\textit{Заголовок раздела}} $;
\State $n=0$;
\ForAll {файлов папки расширения *.desc, *.tex, *.h, *.use}
\State $ ResultTexList+=<File>.desc $ в обработке;
\State $ ResultTexFunctions+=<File>.desc $ в обработке;
\State $ ResultTexFunctions+=<File>.h $ в обертке;
\State $ ResultTexFunctions+=<File>.tex $;
\State $ ResultTexFunctions+=<File>.use $ в обертке;
\State $n++$;
\EndFor
\ForAll {файлов папки расширения *.pdf и *.png}
\State  Скопировать файл <File>.<png|pdf> в папку \textbf{\textbackslash images\textbackslash};
\EndFor
\EndFor
\State $ Library\_Help.tex+=ResultTexList $;
\State $ Library\_Help.tex+=ResultTexFunctions $;
\State Сохранить $ Library\_Help.tex $ в папке temp\_library;
\State \textbf{Конец алгоритма}
\end{algorithmic}
\end{algorithm}

\end{document}
