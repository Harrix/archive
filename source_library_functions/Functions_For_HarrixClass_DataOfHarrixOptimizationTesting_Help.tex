\documentclass[a4paper,12pt]{article}

\input{packages}
\input{styles}

\title{Fu\-ncti\-ons\_For\_Har\-rix\-Class\_Da\-ta\-Of\-Har\-rix\-Opt\-im\-iz\-at\-ion\-Tes\-ting - Har\-rix\-Class\_Da\-ta\-Of\-Har\-rix\-Op\-ti\-mi\-za\-tion\-Test\-ing v.1.26}
\author{А.\,Б. Сергиенко}
\date{\today}


\begin{document}

\input{names}

\maketitle

\begin{abstract}
Класс HarrixClass\_DataOfHarrixOptimizationTesting для считывания информации формата данных Harrix Optimization Testing на C++ для Qt. Рассматривается Functions\_For\_HarrixClass\_DataOfHarrixOptimizationTesting.cpp.
\end{abstract}

\tableofcontents

\newpage

\section{Введение}

Класс HarrixClass\_DataOfHarrixOptimizationTesting для считывания информации формата данных Harrix Optimization Testing на C++ для Qt.

Последнюю версию документа можно найти по адресу:

\href{https://github.com/Harrix/HarrixClass\_DataOfHarrixOptimizationTesting}{https://github.com/Harrix/HarrixClass\_DataOfHarrixOptimizationTesting}

Об установке библиотеки можно прочитать тут:

\href{http://blog.harrix.org/?p=992}{http://blog.harrix.org/?p=992}

С автором можно связаться по адресу \href{mailto:sergienkoanton@mail.ru}{sergienkoanton@mail.ru} или  \href{http://vk.com/harrix}{http://vk.com/harrix}.

Сайт автора, где публикуются последние новости: \href{http://blog.harrix.org/}{http://blog.harrix.org/}, а проекты располагаются по адресу \href{http://harrix.org/}{http://harrix.org/}.

%%%%%%%%%%%%%%%%%%%%%%%%%%%%%%%%%%%%%%%%%%%%%%%%%%%%%%%%%% ВСТАВЛЯТЬ НИЖЕ
\newpage
\section{Список функций}\label{section_listfunctions}
\textbf{Блок функций проверки равенства переменных нескольких исследований}
\begin{enumerate}
	
	\item \textbf{\hyperref[HCDOHOT_CompareOfDataByWilcoxonW]{HCDOHOT\_CompareOfDataByWilcoxonW}} --- Проверяет по критерию Вилкосона два исследования алгоритмов.
	
	\item \textbf{\hyperref[HCDOHOT_CompareOfDataForAuthor]{HCDOHOT\_CompareOfDataForAuthor}} --- Проверяет равенство авторов исследований.
	
	\item \textbf{\hyperref[HCDOHOT_CompareOfDataForCheckAllCombinations]{HCDOHOT\_CompareOfDataForCheckAllCombinations}} --- Проверяет равенство переменной, которая говорит все ли рассмотрены функции в исследованиях.
	
	\item \textbf{\hyperref[HCDOHOT_CompareOfDataForDate]{HCDOHOT\_CompareOfDataForDate}} --- Проверяет равенство дат исследований.
	
	\item \textbf{\hyperref[HCDOHOT_CompareOfDataForDimensionTestFunction]{HCDOHOT\_CompareOfDataForDimensionTestFunction}} --- Проверяет равенство размерностей тестовой задачи (длина хромосомы решения) в исследованиях.
	
	\item \textbf{\hyperref[HCDOHOT_CompareOfDataForEmail]{HCDOHOT\_CompareOfDataForEmail}} --- Проверяет равенство email авторов исследований.
	
	\item \textbf{\hyperref[HCDOHOT_CompareOfDataForFormat]{HCDOHOT\_CompareOfDataForFormat}} --- Проверяет равенство форматов файлов в исследованиях.
	
	\item \textbf{\hyperref[HCDOHOT_CompareOfDataForFullNameAlgorithm]{HCDOHOT\_CompareOfDataForFullNameAlgorithm}} --- Проверяет равенство полных названий алгоритмов в исследованиях.
	
	\item \textbf{\hyperref[HCDOHOT_CompareOfDataForFullNameTestFunction]{HCDOHOT\_CompareOfDataForFullNameTestFunction}} --- Проверяет равенство полных названий тестовых функций в исследованиях.
	
	\item \textbf{\hyperref[HCDOHOT_CompareOfDataForLink]{HCDOHOT\_CompareOfDataForLink}} --- Проверяет равенство ссылок на описание версий формата файла в исследованиях.
	
	\item \textbf{\hyperref[HCDOHOT_CompareOfDataForMaxCountOfFitness]{HCDOHOT\_CompareOfDataForMaxCountOfFitness}} --- Проверяет равенство максимальных допустимых чисел вычислений целевой функции для алгоритма в исследованиях.
	
	\item \textbf{\hyperref[HCDOHOT_CompareOfDataForNameAlgorithm]{HCDOHOT\_CompareOfDataForNameAlgorithm}} --- Проверяет равенство идентификаторов алгоритмов оптимизации: в данных содержится один и тот же алгоритм или же нет.
	
	\item \textbf{\hyperref[HCDOHOT_CompareOfDataForNameTestFunction]{HCDOHOT\_CompareOfDataForNameTestFunction}} --- Проверяет равенство идентификаторов тестовых функций в исследованиях.
	
	\item \textbf{\hyperref[HCDOHOT_CompareOfDataForNumberOfExperiments]{HCDOHOT\_CompareOfDataForNumberOfExperiments}} --- Проверяет равенство количества комбинаций вариантов настроек в исследованиях.
	
	\item \textbf{\hyperref[HCDOHOT_CompareOfDataForNumberOfMeasuring]{HCDOHOT\_CompareOfDataForNumberOfMeasuring}} --- Проверяет равенство количества экспериментов для каждого набора параметров алгоритма в исследованиях.
	
	\item \textbf{\hyperref[HCDOHOT_CompareOfDataForNumberOfParameters]{HCDOHOT\_CompareOfDataForNumberOfParameters}} --- Проверяет равенство количества проверяемых параметров алгоритма оптимизации в исследованиях.
	
	\item \textbf{\hyperref[HCDOHOT_CompareOfDataForNumberOfRuns]{HCDOHOT\_CompareOfDataForNumberOfRuns}} --- Проверяет равенство количества прогонов, по которому делается усреднение для эксперимента в исследованиях.
	
	\item \textbf{\hyperref[HCDOHOT_CompareOfDataForVersion]{HCDOHOT\_CompareOfDataForVersion}} --- Проверяет равенство версий формата файла в исследованиях.
	
\end{enumerate}

\textbf{Генерация отчетов}
\begin{enumerate}
	
	\item \textbf{\hyperref[HCDOHOT_GeneratedAnalysisReportFromFile]{HCDOHOT\_GeneratedAnalysisReportFromFile}} --- Генерирует отчет-анализ Latex по алгоритму по файлу *.hdata.
	
	\item \textbf{\hyperref[HCDOHOT_GeneratedReportAboutAlgorithmFromDir]{HCDOHOT\_GeneratedReportAboutAlgorithmFromDir}} --- Генерирует отчет Latex по алгоритму по файлам *.hdata алгоритма, просматривая все файлы в папке. То, чтобы в папке были файлы только одного алгоритма, вы берете на себя.
	
	\item \textbf{\hyperref[HCDOHOT_GeneratedSimpleReportFromFile]{HCDOHOT\_GeneratedSimpleReportFromFile}} --- Генерирует простой отчет Latex по алгоритму по файлу *.hdata.
	
\end{enumerate}

\textbf{Функции по работе с классом}
\begin{enumerate}
	
	\item \textbf{\hyperref[HCDOHOT_NumberFilesInDir]{HCDOHOT\_NumberFilesInDir}} --- Подсчитывает число HarrixClass\_DataOfHarrixOptimizationTesting файлов в папке.
	
	\item \textbf{\hyperref[HCDOHOT_ReadFilesInDir]{HCDOHOT\_ReadFilesInDir}} --- Заполняет массив SeveralData данными из всех файлов *.hdata из папки.
	
	\item \textbf{\hyperref[HCDOHOT_ReadFilesOnlyDataInDir]{HCDOHOT\_ReadFilesOnlyDataInDir}} --- Заполняет массив SeveralData данными (только исследования) из всех файлов *.hdata из папки.
	
\end{enumerate}


\newpage
\section{Функции}
\subsection{Блок функций проверки равенства переменных нескольких исследований}

\subsubsection{HCDOHOT\_CompareOfDataByWilcoxonW}\label{HCDOHOT_CompareOfDataByWilcoxonW}

Проверяет по критерию Вилкосона два исследования алгоритмов.


\begin{lstlisting}[label=code_syntax_HCDOHOT_CompareOfDataByWilcoxonW,caption=Синтаксис]
int HCDOHOT_CompareOfDataByWilcoxonW (HarrixClass_OnlyDataOfHarrixOptimizationTesting *Data1, HarrixClass_OnlyDataOfHarrixOptimizationTesting *Data2, double Q);
\end{lstlisting}

\textbf{Входные параметры:}


Data1 --- первое исследование;

Data2 --- второе исследование;

Q --- уровень значимости. Может принимать значения:

\begin{itemize}
	\item 0.002; 
	\item 0.01; 
	\item 0.02; 
	\item 0.05; 
	\item 0.1; 
	\item 0.2.
\end{itemize}

\textbf{Возвращаемое значение:}


-2 --- уровень значимости выбран неправильно (не из допустимого множества);

-1 --- объемы выборок не позволяют провести проверку при данном уровне значимости (или они не положительные);

0 --- выборки не однородны  при данном уровне значимости;

1 --- выборки однородны  при данном уровне значимости;


\subsubsection{HCDOHOT\_CompareOfDataForAuthor}\label{HCDOHOT_CompareOfDataForAuthor}

Проверяет равенство авторов исследований.


\begin{lstlisting}[label=code_syntax_HCDOHOT_CompareOfDataForAuthor,caption=Синтаксис]
bool HCDOHOT_CompareOfDataForAuthor (HarrixClass_DataOfHarrixOptimizationTesting *Data1, HarrixClass_DataOfHarrixOptimizationTesting *Data2);
bool HCDOHOT_CompareOfDataForAuthor (HarrixClass_DataOfHarrixOptimizationTesting *SeveralData, int N);
bool HCDOHOT_CompareOfDataForAuthor (HarrixClass_OnlyDataOfHarrixOptimizationTesting *SeveralData, int N);
\end{lstlisting}

\textbf{Входные параметры:}

Data1 --- первое исследование;

Data2 --- второе исследование.

\textbf{Входные параметры в функциях перегрузках:}

SeveralData --- массив исследований;

N --- количество исследований в массиве.

\textbf{Возвращаемое значение:}

true --- если исследуемый параметр алгоритмов одинаков.

false --- если разные.


\subsubsection{HCDOHOT\_CompareOfDataForCheckAllCombinations}\label{HCDOHOT_CompareOfDataForCheckAllCombinations}

Проверяет равенство переменной, которая говорит все ли рассмотрены функции в исследованиях.


\begin{lstlisting}[label=code_syntax_HCDOHOT_CompareOfDataForCheckAllCombinations,caption=Синтаксис]
bool HCDOHOT_CompareOfDataForCheckAllCombinations (HarrixClass_DataOfHarrixOptimizationTesting *Data1, HarrixClass_DataOfHarrixOptimizationTesting *Data2);
bool HCDOHOT_CompareOfDataForCheckAllCombinations (HarrixClass_DataOfHarrixOptimizationTesting *SeveralData, int N);
bool HCDOHOT_CompareOfDataForCheckAllCombinations (HarrixClass_OnlyDataOfHarrixOptimizationTesting *SeveralData, int N);
\end{lstlisting}

\textbf{Входные параметры:}

Data1 --- первое исследование;

Data2 --- второе исследование.

\textbf{Входные параметры в функциях перегрузках:}

SeveralData --- массив исследований;

N --- количество исследований в массиве.

\textbf{Возвращаемое значение:}

true --- если исследуемый параметр алгоритмов одинаков.

false --- если разные.


\subsubsection{HCDOHOT\_CompareOfDataForDate}\label{HCDOHOT_CompareOfDataForDate}

Проверяет равенство дат исследований.


\begin{lstlisting}[label=code_syntax_HCDOHOT_CompareOfDataForDate,caption=Синтаксис]
bool HCDOHOT_CompareOfDataForDate (HarrixClass_DataOfHarrixOptimizationTesting *Data1, HarrixClass_DataOfHarrixOptimizationTesting *Data2);
bool HCDOHOT_CompareOfDataForDate (HarrixClass_DataOfHarrixOptimizationTesting *SeveralData, int N);
bool HCDOHOT_CompareOfDataForDate (HarrixClass_OnlyDataOfHarrixOptimizationTesting *SeveralData, int N);
\end{lstlisting}

\textbf{Входные параметры:}

Data1 --- первое исследование;

Data2 --- второе исследование.

\textbf{Входные параметры в функциях перегрузках:}

SeveralData --- массив исследований;

N --- количество исследований в массиве.

\textbf{Возвращаемое значение:}

true --- если исследуемый параметр алгоритмов одинаков.

false --- если разные.


\subsubsection{HCDOHOT\_CompareOfDataForDimensionTestFunction}\label{HCDOHOT_CompareOfDataForDimensionTestFunction}

Проверяет равенство размерностей тестовой задачи (длина хромосомы решения) в исследованиях.


\begin{lstlisting}[label=code_syntax_HCDOHOT_CompareOfDataForDimensionTestFunction,caption=Синтаксис]
bool HCDOHOT_CompareOfDataForDimensionTestFunction (HarrixClass_DataOfHarrixOptimizationTesting *Data1, HarrixClass_DataOfHarrixOptimizationTesting *Data2);
bool HCDOHOT_CompareOfDataForDimensionTestFunction (HarrixClass_DataOfHarrixOptimizationTesting *SeveralData, int N);
bool HCDOHOT_CompareOfDataForDimensionTestFunction (HarrixClass_OnlyDataOfHarrixOptimizationTesting *SeveralData, int N);
\end{lstlisting}

\textbf{Входные параметры:}

Data1 --- первое исследование;

Data2 --- второе исследование.

\textbf{Входные параметры в функциях перегрузках:}

SeveralData --- массив исследований;

N --- количество исследований в массиве.

\textbf{Возвращаемое значение:}

true --- если исследуемый параметр алгоритмов одинаков.

false --- если разные.


\subsubsection{HCDOHOT\_CompareOfDataForEmail}\label{HCDOHOT_CompareOfDataForEmail}

Проверяет равенство email авторов исследований.


\begin{lstlisting}[label=code_syntax_HCDOHOT_CompareOfDataForEmail,caption=Синтаксис]
bool HCDOHOT_CompareOfDataForEmail (HarrixClass_DataOfHarrixOptimizationTesting *Data1, HarrixClass_DataOfHarrixOptimizationTesting *Data2);
bool HCDOHOT_CompareOfDataForEmail (HarrixClass_DataOfHarrixOptimizationTesting *SeveralData, int N);
bool HCDOHOT_CompareOfDataForEmail (HarrixClass_OnlyDataOfHarrixOptimizationTesting *SeveralData, int N);
\end{lstlisting}

\textbf{Входные параметры:}

Data1 --- первое исследование;

Data2 --- второе исследование.

\textbf{Входные параметры в функциях перегрузках:}

SeveralData --- массив исследований;

N --- количество исследований в массиве.

\textbf{Возвращаемое значение:}

true --- если исследуемый параметр алгоритмов одинаков.

false --- если разные.


\subsubsection{HCDOHOT\_CompareOfDataForFormat}\label{HCDOHOT_CompareOfDataForFormat}

Проверяет равенство форматов файлов в исследованиях.


\begin{lstlisting}[label=code_syntax_HCDOHOT_CompareOfDataForFormat,caption=Синтаксис]
bool HCDOHOT_CompareOfDataForFormat (HarrixClass_DataOfHarrixOptimizationTesting *Data1, HarrixClass_DataOfHarrixOptimizationTesting *Data2);
bool HCDOHOT_CompareOfDataForFormat (HarrixClass_DataOfHarrixOptimizationTesting *SeveralData, int N);
bool HCDOHOT_CompareOfDataForFormat (HarrixClass_OnlyDataOfHarrixOptimizationTesting *SeveralData, int N);
\end{lstlisting}

\textbf{Входные параметры:}

Data1 --- первое исследование;

Data2 --- второе исследование.

\textbf{Входные параметры в функциях перегрузках:}

SeveralData --- массив исследований;

N --- количество исследований в массиве.

\textbf{Возвращаемое значение:}

true --- если исследуемый параметр алгоритмов одинаков.

false --- если разные.


\subsubsection{HCDOHOT\_CompareOfDataForFullNameAlgorithm}\label{HCDOHOT_CompareOfDataForFullNameAlgorithm}

Проверяет равенство полных названий алгоритмов в исследованиях.


\begin{lstlisting}[label=code_syntax_HCDOHOT_CompareOfDataForFullNameAlgorithm,caption=Синтаксис]
bool HCDOHOT_CompareOfDataForFullNameAlgorithm (HarrixClass_DataOfHarrixOptimizationTesting *Data1, HarrixClass_DataOfHarrixOptimizationTesting *Data2);
bool HCDOHOT_CompareOfDataForFullNameAlgorithm (HarrixClass_DataOfHarrixOptimizationTesting *SeveralData, int N);
bool HCDOHOT_CompareOfDataForFullNameAlgorithm (HarrixClass_OnlyDataOfHarrixOptimizationTesting *SeveralData, int N);
\end{lstlisting}

\textbf{Входные параметры:}

Data1 --- первое исследование;

Data2 --- второе исследование.

\textbf{Входные параметры в функциях перегрузках:}

SeveralData --- массив исследований;

N --- количество исследований в массиве.

\textbf{Возвращаемое значение:}

true --- если исследуемый параметр алгоритмов одинаков.

false --- если разные.


\subsubsection{HCDOHOT\_CompareOfDataForFullNameTestFunction}\label{HCDOHOT_CompareOfDataForFullNameTestFunction}

Проверяет равенство полных названий тестовых функций в исследованиях.


\begin{lstlisting}[label=code_syntax_HCDOHOT_CompareOfDataForFullNameTestFunction,caption=Синтаксис]
bool HCDOHOT_CompareOfDataForFullNameTestFunction (HarrixClass_DataOfHarrixOptimizationTesting *Data1, HarrixClass_DataOfHarrixOptimizationTesting *Data2);
bool HCDOHOT_CompareOfDataForFullNameTestFunction (HarrixClass_DataOfHarrixOptimizationTesting *SeveralData, int N);
bool HCDOHOT_CompareOfDataForFullNameTestFunction (HarrixClass_OnlyDataOfHarrixOptimizationTesting *SeveralData, int N);
\end{lstlisting}

\textbf{Входные параметры:}

Data1 --- первое исследование;

Data2 --- второе исследование.

\textbf{Входные параметры в функциях перегрузках:}

SeveralData --- массив исследований;

N --- количество исследований в массиве.

\textbf{Возвращаемое значение:}

true --- если исследуемый параметр алгоритмов одинаков.

false --- если разные.


\subsubsection{HCDOHOT\_CompareOfDataForLink}\label{HCDOHOT_CompareOfDataForLink}

Проверяет равенство ссылок на описание версий формата файла в исследованиях.


\begin{lstlisting}[label=code_syntax_HCDOHOT_CompareOfDataForLink,caption=Синтаксис]
bool HCDOHOT_CompareOfDataForLink (HarrixClass_DataOfHarrixOptimizationTesting *Data1, HarrixClass_DataOfHarrixOptimizationTesting *Data2);
bool HCDOHOT_CompareOfDataForLink (HarrixClass_DataOfHarrixOptimizationTesting *SeveralData, int N);
bool HCDOHOT_CompareOfDataForLink (HarrixClass_OnlyDataOfHarrixOptimizationTesting *SeveralData, int N);
\end{lstlisting}

\textbf{Входные параметры:}

Data1 --- первое исследование;

Data2 --- второе исследование.

\textbf{Входные параметры в функциях перегрузках:}

SeveralData --- массив исследований;

N --- количество исследований в массиве.

\textbf{Возвращаемое значение:}

true --- если исследуемый параметр алгоритмов одинаков.

false --- если разные.


\subsubsection{HCDOHOT\_CompareOfDataForMaxCountOfFitness}\label{HCDOHOT_CompareOfDataForMaxCountOfFitness}

Проверяет равенство максимальных допустимых чисел вычислений целевой функции для алгоритма в исследованиях.


\begin{lstlisting}[label=code_syntax_HCDOHOT_CompareOfDataForMaxCountOfFitness,caption=Синтаксис]
bool HCDOHOT_CompareOfDataForMaxCountOfFitness (HarrixClass_DataOfHarrixOptimizationTesting *Data1, HarrixClass_DataOfHarrixOptimizationTesting *Data2);
bool HCDOHOT_CompareOfDataForMaxCountOfFitness (HarrixClass_DataOfHarrixOptimizationTesting *SeveralData, int N);
bool HCDOHOT_CompareOfDataForMaxCountOfFitness (HarrixClass_OnlyDataOfHarrixOptimizationTesting *SeveralData, int N);
\end{lstlisting}

\textbf{Входные параметры:}

Data1 --- первое исследование;

Data2 --- второе исследование.

\textbf{Входные параметры в функциях перегрузках:}

SeveralData --- массив исследований;

N --- количество исследований в массиве.

\textbf{Возвращаемое значение:}

true --- если исследуемый параметр алгоритмов одинаков.

false --- если разные.


\subsubsection{HCDOHOT\_CompareOfDataForNameAlgorithm}\label{HCDOHOT_CompareOfDataForNameAlgorithm}

Проверяет равенство идентификаторов алгоритмов оптимизации: в данных содержится один и тот же алгоритм или же нет.


\begin{lstlisting}[label=code_syntax_HCDOHOT_CompareOfDataForNameAlgorithm,caption=Синтаксис]
bool HCDOHOT_CompareOfDataForNameAlgorithm (HarrixClass_DataOfHarrixOptimizationTesting *Data1, HarrixClass_DataOfHarrixOptimizationTesting *Data2);
bool HCDOHOT_CompareOfDataForNameAlgorithm (HarrixClass_DataOfHarrixOptimizationTesting *SeveralData, int N);
bool HCDOHOT_CompareOfDataForNameAlgorithm (HarrixClass_OnlyDataOfHarrixOptimizationTesting *SeveralData, int N);
\end{lstlisting}

\textbf{Входные параметры:}

Data1 --- первое исследование;

Data2 --- второе исследование.

\textbf{Входные параметры в функциях перегрузках:}

SeveralData --- массив исследований;

N --- количество исследований в массиве.

\textbf{Возвращаемое значение:}

true --- если имена алгоритмов одинаковы.

false --- если разные.


\subsubsection{HCDOHOT\_CompareOfDataForNameTestFunction}\label{HCDOHOT_CompareOfDataForNameTestFunction}

Проверяет равенство идентификаторов тестовых функций в исследованиях.


\begin{lstlisting}[label=code_syntax_HCDOHOT_CompareOfDataForNameTestFunction,caption=Синтаксис]
bool HCDOHOT_CompareOfDataForNameTestFunction (HarrixClass_DataOfHarrixOptimizationTesting *Data1, HarrixClass_DataOfHarrixOptimizationTesting *Data2);
bool HCDOHOT_CompareOfDataForNameTestFunction (HarrixClass_DataOfHarrixOptimizationTesting *SeveralData, int N);
bool HCDOHOT_CompareOfDataForNameTestFunction (HarrixClass_OnlyDataOfHarrixOptimizationTesting *SeveralData, int N);
\end{lstlisting}

\textbf{Входные параметры:}

Data1 --- первое исследование;

Data2 --- второе исследование.

\textbf{Входные параметры в функциях перегрузках:}

SeveralData --- массив исследований;

N --- количество исследований в массиве.

\textbf{Возвращаемое значение:}

true --- если исследуемый параметр алгоритмов одинаков.

false --- если разные.


\subsubsection{HCDOHOT\_CompareOfDataForNumberOfExperiments}\label{HCDOHOT_CompareOfDataForNumberOfExperiments}

Проверяет равенство количества комбинаций вариантов настроек в исследованиях.


\begin{lstlisting}[label=code_syntax_HCDOHOT_CompareOfDataForNumberOfExperiments,caption=Синтаксис]
bool HCDOHOT_CompareOfDataForNumberOfExperiments (HarrixClass_DataOfHarrixOptimizationTesting *Data1, HarrixClass_DataOfHarrixOptimizationTesting *Data2);
bool HCDOHOT_CompareOfDataForNumberOfExperiments (HarrixClass_DataOfHarrixOptimizationTesting *SeveralData, int N);
bool HCDOHOT_CompareOfDataForNumberOfExperiments (HarrixClass_OnlyDataOfHarrixOptimizationTesting *SeveralData, int N);
\end{lstlisting}

\textbf{Входные параметры:}

Data1 --- первое исследование;

Data2 --- второе исследование.

\textbf{Входные параметры в функциях перегрузках:}

SeveralData --- массив исследований;

N --- количество исследований в массиве.

\textbf{Возвращаемое значение:}

true --- если исследуемый параметр алгоритмов одинаков.

false --- если разные.


\subsubsection{HCDOHOT\_CompareOfDataForNumberOfMeasuring}\label{HCDOHOT_CompareOfDataForNumberOfMeasuring}

Проверяет равенство количества экспериментов для каждого набора параметров алгоритма в исследованиях.


\begin{lstlisting}[label=code_syntax_HCDOHOT_CompareOfDataForNumberOfMeasuring,caption=Синтаксис]
bool HCDOHOT_CompareOfDataForNumberOfMeasuring (HarrixClass_DataOfHarrixOptimizationTesting *Data1, HarrixClass_DataOfHarrixOptimizationTesting *Data2);
bool HCDOHOT_CompareOfDataForNumberOfMeasuring (HarrixClass_DataOfHarrixOptimizationTesting *SeveralData, int N);
bool HCDOHOT_CompareOfDataForNumberOfMeasuring (HarrixClass_OnlyDataOfHarrixOptimizationTesting *SeveralData, int N);
\end{lstlisting}

\textbf{Входные параметры:}

Data1 --- первое исследование;

Data2 --- второе исследование.

\textbf{Входные параметры в функциях перегрузках:}

SeveralData --- массив исследований;

N --- количество исследований в массиве.

\textbf{Возвращаемое значение:}

true --- если исследуемый параметр алгоритмов одинаков.

false --- если разные.


\subsubsection{HCDOHOT\_CompareOfDataForNumberOfParameters}\label{HCDOHOT_CompareOfDataForNumberOfParameters}

Проверяет равенство количества проверяемых параметров алгоритма оптимизации в исследованиях.


\begin{lstlisting}[label=code_syntax_HCDOHOT_CompareOfDataForNumberOfParameters,caption=Синтаксис]
bool HCDOHOT_CompareOfDataForNumberOfParameters (HarrixClass_DataOfHarrixOptimizationTesting *Data1, HarrixClass_DataOfHarrixOptimizationTesting *Data2);
bool HCDOHOT_CompareOfDataForNumberOfParameters (HarrixClass_DataOfHarrixOptimizationTesting *SeveralData, int N);
bool HCDOHOT_CompareOfDataForNumberOfParameters (HarrixClass_OnlyDataOfHarrixOptimizationTesting *SeveralData, int N);
\end{lstlisting}

\textbf{Входные параметры:}

Data1 --- первое исследование;

Data2 --- второе исследование.

\textbf{Входные параметры в функциях перегрузках:}

SeveralData --- массив исследований;

N --- количество исследований в массиве.

\textbf{Возвращаемое значение:}

true --- если исследуемый параметр алгоритмов одинаков.

false --- если разные.


\subsubsection{HCDOHOT\_CompareOfDataForNumberOfRuns}\label{HCDOHOT_CompareOfDataForNumberOfRuns}

Проверяет равенство количества прогонов, по которому делается усреднение для эксперимента в исследованиях.


\begin{lstlisting}[label=code_syntax_HCDOHOT_CompareOfDataForNumberOfRuns,caption=Синтаксис]
bool HCDOHOT_CompareOfDataForNumberOfRuns (HarrixClass_DataOfHarrixOptimizationTesting *Data1, HarrixClass_DataOfHarrixOptimizationTesting *Data2);
bool HCDOHOT_CompareOfDataForNumberOfRuns (HarrixClass_DataOfHarrixOptimizationTesting *SeveralData, int N);
bool HCDOHOT_CompareOfDataForNumberOfRuns (HarrixClass_OnlyDataOfHarrixOptimizationTesting *SeveralData, int N);
\end{lstlisting}

\textbf{Входные параметры:}

Data1 --- первое исследование;

Data2 --- второе исследование.

\textbf{Входные параметры в функциях перегрузках:}

SeveralData --- массив исследований;

N --- количество исследований в массиве.

\textbf{Возвращаемое значение:}

true --- если исследуемый параметр алгоритмов одинаков.

false --- если разные.


\subsubsection{HCDOHOT\_CompareOfDataForVersion}\label{HCDOHOT_CompareOfDataForVersion}

Проверяет равенство версий формата файла в исследованиях.


\begin{lstlisting}[label=code_syntax_HCDOHOT_CompareOfDataForVersion,caption=Синтаксис]
bool HCDOHOT_CompareOfDataForVersion (HarrixClass_DataOfHarrixOptimizationTesting *Data1, HarrixClass_DataOfHarrixOptimizationTesting *Data2);
bool HCDOHOT_CompareOfDataForVersion (HarrixClass_DataOfHarrixOptimizationTesting *SeveralData, int N);
bool HCDOHOT_CompareOfDataForVersion (HarrixClass_OnlyDataOfHarrixOptimizationTesting *SeveralData, int N);
\end{lstlisting}

\textbf{Входные параметры:}

Data1 --- первое исследование;

Data2 --- второе исследование.

\textbf{Входные параметры в функциях перегрузках:}

SeveralData --- массив исследований;

N --- количество исследований в массиве.

\textbf{Возвращаемое значение:}

true --- если исследуемый параметр алгоритмов одинаков.

false --- если разные.


\subsection{Генерация отчетов}

\subsubsection{HCDOHOT\_GeneratedAnalysisReportFromFile}\label{HCDOHOT_GeneratedAnalysisReportFromFile}

Генерирует отчет-анализ Latex по алгоритму по файлу *.hdata.


\begin{lstlisting}[label=code_syntax_HCDOHOT_GeneratedAnalysisReportFromFile,caption=Синтаксис]
void HCDOHOT_GeneratedAnalysisReportFromFile(QString filename, QString pathForSave, QString pathForTempHtml);
void HCDOHOT_GeneratedAnalysisReportFromFile(QString filename, QString pathForSave);
\end{lstlisting}

\textbf{Входные параметры:}

filename --- путь к файлу, из которого считываем данные.

pathForSave --- путь к папке, куда сохраняем Latex файлы.

pathForTempHtml --- путь к папке куда сохраняем во время работы функции отчет в виде temp.html.

\textbf{Возвращаемое значение:}

Отсутствует.

В папке сохранения должны быть находиться файлы names.tex, packages.tex, styles.tex из проекта https://github.com/Harrix/HarrixLaTeXDocumentTemplate. Для отчета в виде html берется проект: https://github.com/Harrix/HarrixHtmlForQWebView.


\subsubsection{HCDOHOT\_GeneratedReportAboutAlgorithmFromDir}\label{HCDOHOT_GeneratedReportAboutAlgorithmFromDir}

Генерирует отчет Latex по алгоритму по файлам *.hdata алгоритма, просматривая все файлы в папке. То, чтобы в папке были файлы только одного алгоритма, вы берете на себя.


\begin{lstlisting}[label=code_syntax_HCDOHOT_GeneratedReportAboutAlgorithmFromDir,caption=Синтаксис]
void HCDOHOT_GeneratedReportAboutAlgorithmFromDir(QString path, QString pathForSave, QString pathForTempHtml);
void HCDOHOT_GeneratedReportAboutAlgorithmFromDir(QString path, QString pathForSave);
\end{lstlisting}

\textbf{Входные параметры:}

path --- путь к папке, из которой считаем файлы.

pathForSave --- путь к папке, куда сохраняем Latex файлы.

pathForTempHtml --- путь к папке куда сохраняем во время работы функции отчет в виде temp.html.

\textbf{Возвращаемое значение:}

Отсутствует.

В папке сохранения должны быть находиться файлы names.tex, packages.tex, styles.tex из проекта https://github.com/Harrix/HarrixLaTeXDocumentTemplate. Для отчета в виде html берется проект: https://github.com/Harrix/HarrixHtmlForQWebView.


\subsubsection{HCDOHOT\_GeneratedSimpleReportFromFile}\label{HCDOHOT_GeneratedSimpleReportFromFile}

Генерирует простой отчет Latex по алгоритму по файлу *.hdata.


\begin{lstlisting}[label=code_syntax_HCDOHOT_GeneratedSimpleReportFromFile,caption=Синтаксис]
void HCDOHOT_GeneratedSimpleReportFromFile(QString filename, QString pathForSave, QString pathForTempHtml);
void HCDOHOT_GeneratedSimpleReportFromFile(QString filename, QString pathForSave);
\end{lstlisting}

\textbf{Входные параметры:}

filename --- путь к файлу, из которого считываем данные.

pathForSave --- путь к папке, куда сохраняем Latex файлы.

pathForTempHtml --- путь к папке куда сохраняем во время работы функции отчет в виде temp.html.

\textbf{Возвращаемое значение:}

Отсутствует.

В папке сохранения должны быть находиться файлы names.tex, packages.tex, styles.tex из проекта https://github.com/Harrix/HarrixLaTeXDocumentTemplate. Для отчета в виде html берется проект: https://github.com/Harrix/HarrixHtmlForQWebView.


\subsection{Функции по работе с классом}

\subsubsection{HCDOHOT\_NumberFilesInDir}\label{HCDOHOT_NumberFilesInDir}

Подсчитывает число HarrixClass\_DataOfHarrixOptimizationTesting файлов в папке.


\begin{lstlisting}[label=code_syntax_HCDOHOT_NumberFilesInDir,caption=Синтаксис]
int HCDOHOT_NumberFilesInDir(QString path);
\end{lstlisting}

\textbf{Входные параметры:}

path --- путь к папке, из которой считаем файлы.

\textbf{Возвращаемое значение:}

Число файлов HarrixClass\_DataOfHarrixOptimizationTesting файлов в папке.


\subsubsection{HCDOHOT\_ReadFilesInDir}\label{HCDOHOT_ReadFilesInDir}

Заполняет массив SeveralData данными из всех файлов *.hdata из папки.


\begin{lstlisting}[label=code_syntax_HCDOHOT_ReadFilesInDir,caption=Синтаксис]
int HCDOHOT_ReadFilesInDir(HarrixClass_DataOfHarrixOptimizationTesting *SeveralData, QString path, QString pathForTempHtml);
int HCDOHOT_ReadFilesInDir(HarrixClass_DataOfHarrixOptimizationTesting *SeveralData, QString path);
\end{lstlisting}

\textbf{Входные параметры:}

SeveralData --- массив, в который записываем данные.

path --- путь к папке, из которой считаем файлы.

pathForTempHtml --- путь к папке куда сохраняем во время работы функции отчет в виде temp.html.

\textbf{Возвращаемое значение:}

Число файлов HarrixClass\_DataOfHarrixOptimizationTesting файлов в папке.

\textbf{Примечание:}

Подсчитать число файлов в папке до вызова этой функции можно функцией HCDOHOT\_NumberFilesInDir.



\subsubsection{HCDOHOT\_ReadFilesOnlyDataInDir}\label{HCDOHOT_ReadFilesOnlyDataInDir}

Заполняет массив SeveralData данными (только исследования) из всех файлов *.hdata из папки.


\begin{lstlisting}[label=code_syntax_HCDOHOT_ReadFilesOnlyDataInDir,caption=Синтаксис]
int HCDOHOT_ReadFilesOnlyDataInDir(HarrixClass_OnlyDataOfHarrixOptimizationTesting *SeveralData, QString path, QString pathForTempHtml);
int HCDOHOT_ReadFilesOnlyDataInDir(HarrixClass_OnlyDataOfHarrixOptimizationTesting *SeveralData, QString path);
\end{lstlisting}

\textbf{Входные параметры:}

SeveralData --- массив, в который записываем данные.

path --- путь к папке, из которой считаем файлы.

pathForTempHtml --- путь к папке куда сохраняем во время работы функции отчет в виде temp.html.

\textbf{Возвращаемое значение:}


Число файлов HarrixClass\_DataOfHarrixOptimizationTesting файлов в папке.

\textbf{Примечание:}

Подсчитать число файлов в папке до вызова этой функции можно функцией HCDOHOT\_NumberFilesInDir.

%%%%%%%%%%%%%%%%%%%%%%%%%%%%%%%%%%%%%%%%%%%%%%%%%%%%%%%%%%

\end{document}