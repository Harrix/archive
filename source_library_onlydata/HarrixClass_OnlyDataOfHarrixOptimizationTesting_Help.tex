\documentclass[a4paper,12pt]{article}

\input{packages}
\input{styles}

\title{Har\-rix\-Class\_On\-ly\-Da\-ta\-Of\-Har\-rix\-Op\-ti\-mi\-za\-tion\-Test\-ing - Har\-rix\-Class\_Da\-ta\-Of\-Har\-rix\-Op\-ti\-mi\-za\-ti\-on\-Tes\-ting v.1.25}
\author{А.\,Б. Сергиенко}
\date{\today}


\begin{document}

\input{names}

\maketitle

\begin{abstract}
Класс HarrixClass\_DataOfHarrixOptimizationTesting для считывания информации формата данных Harrix Optimization Testing на C++ для Qt. Рассматривается HarrixClass\_OnlyDataOfHarrixOptimizationTesting.cpp.
\end{abstract}

\tableofcontents

\newpage

\section{Введение}

Класс HarrixClass\_DataOfHarrixOptimizationTesting для считывания информации формата данных Harrix Optimization Testing на C++ для Qt.

Последнюю версию документа можно найти по адресу:

\href{https://github.com/Harrix/HarrixClass\_DataOfHarrixOptimizationTesting}{https://github.com/Harrix/HarrixClass\_DataOfHarrixOptimizationTesting}

Об установке библиотеки можно прочитать тут:

\href{http://blog.harrix.org/?p=992}{http://blog.harrix.org/?p=992}

С автором можно связаться по адресу \href{mailto:sergienkoanton@mail.ru}{sergienkoanton@mail.ru} или  \href{http://vk.com/harrix}{http://vk.com/harrix}.

Сайт автора, где публикуются последние новости: \href{http://blog.harrix.org/}{http://blog.harrix.org/}, а проекты располагаются по адресу \href{http://harrix.org/}{http://harrix.org/}.

%%%%%%%%%%%%%%%%%%%%%%%%%%%%%%%%%%%%%%%%%%%%%%%%%%%%%%%%%% ВСТАВЛЯТЬ НИЖЕ
\newpage
\section{Список функций}\label{section_listfunctions}
\textbf{\_Конструкторы и деструкторы}
\begin{enumerate}

\item \textbf{\hyperref[HarrixClass_OnlyDataOfHarrixOptimizationTesting]{HarrixClass\_OnlyDataOfHarrixOptimizationTesting}} --- Конструктор. Создает пустой экземпляр.

\item \textbf{\hyperref[tildaHarrixClass_OnlyDataOfHarrixOptimizationTesting]{$\sim$HarrixClass\_OnlyDataOfHarrixOptimizationTesting}} --- Деструктор класса.

\end{enumerate}

\textbf{Возвращение данных из класса}
\begin{enumerate}

\item \textbf{\hyperref[getAuthor]{getAuthor}} --- Получение текста переменной XML\_Author - Автор документа.

\item \textbf{\hyperref[getCheckAllCombinations]{getCheckAllCombinations}} --- Получение текста переменной  XML\_All\_Combinations --- Все ли комбинации вариантов настроек просмотрены: 0 или 1.

\item \textbf{\hyperref[getDate]{getDate}} --- Получение текста переменной  XML\_Date - Дата создания документа.

\item \textbf{\hyperref[getDimensionTestFunction]{getDimensionTestFunction}} --- Получение текста переменной  XML\_DimensionTestFunction --- Размерность тестовой задачи.

\item \textbf{\hyperref[getEmail]{getEmail}} --- Получение текста переменной  XML\_Email - Email автора, чтобы можно было с ним связаться.

\item \textbf{\hyperref[getErrorEx]{getErrorEx}} --- Получение значения ошибки Ex.

\item \textbf{\hyperref[getErrorEy]{getErrorEy}} --- Получение значения ошибки Ey.

\item \textbf{\hyperref[getErrorR]{getErrorR}} --- Получение значения надежности R.

\item \textbf{\hyperref[getFormat]{getFormat}} --- Получение переменной XML\_Format, то есть возвращает название формата документа.

\item \textbf{\hyperref[getFullNameAlgorithm]{getFullNameAlgorithm}} --- Получение текста переменной  XML\_Full\_Name\_Algorithm --- Полное название алгоритма оптимизации.

\item \textbf{\hyperref[getFullNameTestFunction]{getFullNameTestFunction}} --- Получение текста переменной  XML\_Full\_Name\_Test\_Function --- Полное название тестовой функции.

\item \textbf{\hyperref[getLink]{getLink}} --- Получение переменной XML\_Link, то есть возвращает ссылку на описание формата файла.

\item \textbf{\hyperref[getLinkAlgorithm]{getLinkAlgorithm}} --- Получение текста переменной  XML\_Link\_Algorithm --- Ссылка на описание алгоритма оптимизации.

\item \textbf{\hyperref[getLinkTestFunction]{getLinkTestFunction}} --- Получение текста переменной  XML\_Link\_Test\_Function --- Ссылка на описание тестовой функции.

\item \textbf{\hyperref[getListOfParameterOptions]{getListOfParameterOptions}} --- Получение списка вектора названий вариантов параметров алгоритма оптимизации.

\item \textbf{\hyperref[getListOfVectorParameterOptions]{getListOfVectorParameterOptions}} --- Получение списка вектора названий вариантов параметров алгоритма оптимизации --- это сборник строк из MatrixOfNameParameters, где объединены столбцы. Получение строки параметров эксперимента из списка вектора названий вариантов параметров алгоритма оптимизации --- это сборник строк из MatrixOfNameParameters, где объединены столбцы.

\item \textbf{\hyperref[getMaxCountOfFitness]{getMaxCountOfFitness}} --- Получение текста переменной  Max\_Count\_Of\_Fitness --- Максимальное допустимое число вычислений целевой функции для алгоритма.

\item \textbf{\hyperref[getMeanEx]{getMeanEx}} --- Получение среднего значения ошибки Ex по измерениям для настройки (сколько точек было по столько и усредняем).

\item \textbf{\hyperref[getMeanEy]{getMeanEy}} --- Получение среднего значения ошибки Ey по измерениям для настройки (сколько точек было по столько и усредняем).

\item \textbf{\hyperref[getMeanOfAllEx]{getMeanOfAllEx}} --- Получение значения переменной MeanOfAllEx --- среднее значение ошибок Ex алгоритма оптимизации по измерениям по всем измерениям вообще

\item \textbf{\hyperref[getMeanOfAllEy]{getMeanOfAllEy}} --- Получение значения переменной MeanOfAllEy --- среднее значение ошибок Ey алгоритма оптимизации по измерениям по всем измерениям вообще.

\item \textbf{\hyperref[getMeanOfAllR]{getMeanOfAllR}} --- Получение значения переменной MeanOfAllR --- среднее значение надежностей R алгоритма оптимизации по измерениям по всем измерениям вообще.

\item \textbf{\hyperref[getMeanR]{getMeanR}} --- Получение среднего значения надежности R по измерениям для настройки (сколько точек было по столько и усредняем).

\item \textbf{\hyperref[getNameAlgorithm]{getNameAlgorithm}} --- Получение текста переменной  XML\_Name\_Algorithm - Название алгоритма оптимизации.

\item \textbf{\hyperref[getNameOption]{getNameOption}} --- Получение имени параметра алгоритма по его номеру.

\item \textbf{\hyperref[getNameParameter]{getNameParameter}} --- Получение значения параметра настройки какой-то в виде полного наименования.

\item \textbf{\hyperref[getNameTestFunction]{getNameTestFunction}} --- Получение текста переменной  XML\_Name\_Test\_Function --- Название тестовой функции.

\item \textbf{\hyperref[getNamesOfParameters]{getNamesOfParameters}} --- Получение списка параметров алгоритма (тип селекции, тип скрещивания).

\item \textbf{\hyperref[getNumberOfExperiments]{getNumberOfExperiments}} --- Получение текста переменной  XML\_Number\_Of\_Experiments --- Количество комбинаций вариантов настроек.

\item \textbf{\hyperref[getNumberOfMeasuring]{getNumberOfMeasuring}} --- Получение текста переменной  XML\_Number\_Of\_Measuring --- Размерность тестовой задачи (длина хромосомы решения).

\item \textbf{\hyperref[getNumberOfParameters]{getNumberOfParameters}} --- Получение текста переменной  XML\_Number\_Of\_Parameters --- Количество проверяемых параметров алгоритма оптимизации.

\item \textbf{\hyperref[getNumberOfRuns]{getNumberOfRuns}} --- Получение текста переменной  XML\_Number\_Of\_Runs --- Количество прогонов по которому делается усреднение для эксперимента.

\item \textbf{\hyperref[getOptionFromListOfParameterOptions]{getOptionFromListOfParameterOptions}} --- Получение названия вариантов параметров алгоритма оптимизации.

\item \textbf{\hyperref[getOptionFromListOfParameterOptionsForTable]{getOptionFromListOfParameterOptionsForTable}} --- Получение названия вариантов параметров алгоритма оптимизации. Но старается где-то сокращать, а где-то добавлять строки.

\item \textbf{\hyperref[getParameter]{getParameter}} --- Получение значения параметра настройки какой-то.

\item \textbf{\hyperref[getSuccessReading]{getSuccessReading}} --- Получение текста переменной SuccessReading --- Успешно ли прошло считывание.

\item \textbf{\hyperref[getVarianceOfAllEx]{getVarianceOfAllEx}} --- Получение значения переменной VarianceOfAllEx --- дисперсия ошибок Ex алгоритма оптимизации по измерениям по всем измерениям вообще.

\item \textbf{\hyperref[getVarianceOfAllEy]{getVarianceOfAllEy}} --- Получение значения переменной VarianceOfAllEy --- дисперсия ошибок Ey алгоритма оптимизации по измерениям по всем измерениям вообще

\item \textbf{\hyperref[getVarianceOfAllR]{getVarianceOfAllR}} --- Получение значения переменной VarianceOfAllR --- дисперсия надежностей R алгоритма оптимизации по измерениям по всем измерениям вообще.

\item \textbf{\hyperref[getVarianceOfEx]{getVarianceOfEx}} --- Получение дисперсии значения ошибки Ex по измерениям для настройки (сколько точек было по столько и усредняем).

\item \textbf{\hyperref[getVarianceOfEy]{getVarianceOfEy}} --- Получение дисперсии значения ошибки Ey по измерениям для настройки (сколько точек было по столько и усредняем).

\item \textbf{\hyperref[getVarianceOfR]{getVarianceOfR}} --- Получение дисперсии значения надежности R по измерениям для настройки (сколько точек было по столько и усредняем).

\item \textbf{\hyperref[getVersion]{getVersion}} --- Получение переменной Version, то есть возвращает версию формата документа.

\end{enumerate}

\textbf{Задание данных в класс}
\begin{enumerate}

\item \textbf{\hyperref[addListOfParameterOptions]{addListOfParameterOptions}} --- Добавление списка вектора названий вариантов параметров алгоритма оптимизации.

\item \textbf{\hyperref[addListOfVectorParameterOptions]{addListOfVectorParameterOptions}} --- Добавление строки параметров эксперимента из списка вектора названий вариантов параметров алгоритма оптимизации --- это сборник строк из MatrixOfNameParameters, где объединены столбцы.

\item \textbf{\hyperref[addNameOption]{addNameOption}} --- Добавление имени параметра алгоритма.

\item \textbf{\hyperref[addNameParameter]{addNameParameter}} --- Получение значения параметра настройки какой-то в виде полного наименования.

\item \textbf{\hyperref[setAuthor]{setAuthor}} --- Задание текста переменной XML\_Author - Автор документа.

\item \textbf{\hyperref[setCheckAllCombinations]{setCheckAllCombinations}} --- Задание текста переменной  XML\_All\_Combinations --- Все ли комбинации вариантов настроек просмотрены: 0 или 1.

\item \textbf{\hyperref[setDate]{setDate}} --- Задание текста переменной  XML\_Date - Дата создания документа.

\item \textbf{\hyperref[setDimensionTestFunction]{setDimensionTestFunction}} --- Задание текста переменной  XML\_DimensionTestFunction --- Размерность тестовой задачи.

\item \textbf{\hyperref[setEmail]{setEmail}} --- Задание текста переменной  XML\_Email - Email автора, чтобы можно было с ним связаться

\item \textbf{\hyperref[setErrorEx]{setErrorEx}} --- Задание значения ошибки Ex.

\item \textbf{\hyperref[setErrorEy]{setErrorEy}} --- Задание значения ошибки Ey.

\item \textbf{\hyperref[setErrorR]{setErrorR}} --- Задание значения надежности R.

\item \textbf{\hyperref[setFormat]{setFormat}} --- Задание переменной XML\_Format --- название формата документа.

\item \textbf{\hyperref[setFullNameAlgorithm]{setFullNameAlgorithm}} --- Задание текста переменной  XML\_Full\_Name\_Algorithm --- Полное название алгоритма оптимизации.

\item \textbf{\hyperref[setFullNameTestFunction]{setFullNameTestFunction}} --- Задание текста переменной  XML\_Full\_Name\_Test\_Function --- Полное название тестовой функции.

\item \textbf{\hyperref[setLink]{setLink}} --- Задание переменной XML\_Link --- ссылка на описание формата файла.

\item \textbf{\hyperref[setLinkAlgorithm]{setLinkAlgorithm}} --- Задание текста переменной  XML\_Link\_Algorithm --- Ссылка на описание алгоритма оптимизации.

\item \textbf{\hyperref[setLinkTestFunction]{setLinkTestFunction}} --- Задание текста переменной  XML\_Link\_Test\_Function --- Ссылка на описание тестовой функции.

\item \textbf{\hyperref[setListOfParameterOptions]{setListOfParameterOptions}} --- Задание списка вектора названий вариантов параметров алгоритма оптимизации.

\item \textbf{\hyperref[setMaxCountOfFitness]{setMaxCountOfFitness}} --- Задание текста переменной  Max\_Count\_Of\_Fitness --- Максимальное допустимое число вычислений целевой функции для алгоритма.

\item \textbf{\hyperref[setMeanEx]{setMeanEx}} --- Задание среднего значения ошибки Ex по измерениям для настройки (сколько точек было по столько и усредняем).

\item \textbf{\hyperref[setMeanEy]{setMeanEy}} --- Задание среднего значения ошибки Ey по измерениям для настройки (сколько точек было по столько и усредняем).

\item \textbf{\hyperref[setMeanOfAllEx]{setMeanOfAllEx}} --- Задание значения переменной MeanOfAllEx - среднее значение ошибок Ex алгоритма оптимизации по измерениям по всем измерениям вообще

\item \textbf{\hyperref[setMeanOfAllEy]{setMeanOfAllEy}} --- Задание значения переменной MeanOfAllEy - среднее значение ошибок Ey алгоритма оптимизации по измерениям по всем измерениям вообще.

\item \textbf{\hyperref[setMeanOfAllR]{setMeanOfAllR}} --- Задание значения переменной MeanOfAllR --- среднее значение надежностей R алгоритма оптимизации по измерениям по всем измерениям вообще.

\item \textbf{\hyperref[setMeanR]{setMeanR}} --- Задание среднего значения надежности R по измерениям для настройки (сколько точек было по столько и усредняем).

\item \textbf{\hyperref[setNameAlgorithm]{setNameAlgorithm}} --- Получение текста переменной  XML\_Name\_Algorithm - Название алгоритма оптимизации.

\item \textbf{\hyperref[setNameTestFunction]{setNameTestFunction}} --- Задание текста переменной  XML\_Name\_Test\_Function --- Название тестовой функции.

\item \textbf{\hyperref[setNumberOfExperiments]{setNumberOfExperiments}} --- Задание текста переменной  XML\_Number\_Of\_Experiments --- Количество комбинаций вариантов настроек.

\item \textbf{\hyperref[setNumberOfListOfVectorParameterOptions]{setNumberOfListOfVectorParameterOptions}} --- Задание значения элемента массива NumberOfListOfVectorParameterOptions.

\item \textbf{\hyperref[setNumberOfMeasuring]{setNumberOfMeasuring}} --- Задание текста переменной  XML\_Number\_Of\_Measuring --- Размерность тестовой задачи (длина хромосомы решения).

\item \textbf{\hyperref[setNumberOfParameters]{setNumberOfParameters}} --- Задание текста переменной  XML\_Number\_Of\_Parameters --- Количество проверяемых параметров алгоритма оптимизации.

\item \textbf{\hyperref[setNumberOfRuns]{setNumberOfRuns}} --- Задание текста переменной  XML\_Number\_Of\_Runs --- Количество прогонов по которому делается усреднение для эксперимента.

\item \textbf{\hyperref[setParameter]{setParameter}} --- Задание значения параметра настройки какой-то.

\item \textbf{\hyperref[setSuccessReading]{setSuccessReading}} --- Задание текста переменной SuccessReading --- Успешно ли прошло считывание.

\item \textbf{\hyperref[setVarianceOfAllEx]{setVarianceOfAllEx}} --- Задание значения переменной VarianceOfAllEx --- дисперсия ошибок Ex алгоритма оптимизации по измерениям по всем измерениям вообще

\item \textbf{\hyperref[setVarianceOfAllEy]{setVarianceOfAllEy}} --- Задание значения переменной VarianceOfAllEy --- дисперсия ошибок Ey алгоритма оптимизации по измерениям по всем измерениям вообще.

\item \textbf{\hyperref[setVarianceOfAllR]{setVarianceOfAllR}} --- Задание значения переменной VarianceOfAllR --- дисперсия надежностей R алгоритма оптимизации по измерениям по всем измерениям вообще.

\item \textbf{\hyperref[setVarianceOfEx]{setVarianceOfEx}} --- Получение дисперсии значения ошибки Ex по измерениям для настройки (сколько точек было по столько и усредняем).

\item \textbf{\hyperref[setVarianceOfEy]{setVarianceOfEy}} --- Получение дисперсии значения ошибки Ey по измерениям для настройки (сколько точек было по столько и усредняем).

\item \textbf{\hyperref[setVarianceOfR]{setVarianceOfR}} --- Получение дисперсии значения надежности R по измерениям для настройки (сколько точек было по столько и усредняем).

\item \textbf{\hyperref[setVersion]{setVersion}} --- Задание переменной XML\_Version --- версия формата документа.

\end{enumerate}

\textbf{Операторы}
\begin{enumerate}

\item \textbf{\hyperref[operator =]{operator =}} --- Оператор присваивания.

\end{enumerate}

\textbf{Служебные функции}
\begin{enumerate}

\item \textbf{\hyperref[initializationOfVariables]{initializationOfVariables}} --- Обнуление переменных. Внутренняя функция.

\item \textbf{\hyperref[memoryAllocation]{memoryAllocation}} --- Выделяет память под необходимые массивы.

\item \textbf{\hyperref[memoryDeallocation]{memoryDeallocation}} --- Удаляет память из-под массивов. Внутренняя функция.

\end{enumerate}


\newpage
\section{Функции}
\subsection{\_Конструкторы и деструкторы}

\subsubsection{HarrixClass\_OnlyDataOfHarrixOptimizationTesting}\label{HarrixClass_OnlyDataOfHarrixOptimizationTesting}

Конструктор. Создает пустой экземпляр.


\begin{lstlisting}[label=code_syntax_HarrixClass_OnlyDataOfHarrixOptimizationTesting,caption=Синтаксис]
HarrixClass_OnlyDataOfHarrixOptimizationTesting();
\end{lstlisting}

\textbf{Входные параметры:}

Отсутствуют.

\textbf{Возвращаемое значение:}

Отсутствует.


\subsubsection{$\sim$HarrixClass\_OnlyDataOfHarrixOptimizationTesting}\label{tildaHarrixClass_OnlyDataOfHarrixOptimizationTesting}

Деструктор класса.


\begin{lstlisting}[label=code_syntax_tildaHarrixClass_OnlyDataOfHarrixOptimizationTesting,caption=Синтаксис]
~HarrixClass_OnlyDataOfHarrixOptimizationTesting();
\end{lstlisting}

\textbf{Входные параметры:}

Отсутствуют.

\textbf{Возвращаемое значение:}

Отсутствует.


\subsection{Возвращение данных из класса}

\subsubsection{getAuthor}\label{getAuthor}

Получение текста переменной XML\_Author - Автор документа.


\begin{lstlisting}[label=code_syntax_getAuthor,caption=Синтаксис]
QString getAuthor();
\end{lstlisting}

\textbf{Входные параметры:}

Отсутствуют.

\textbf{Возвращаемое значение:}

Значение переменной из описания.


\subsubsection{getCheckAllCombinations}\label{getCheckAllCombinations}

Получение текста переменной  XML\_All\_Combinations --- Все ли комбинации вариантов настроек просмотрены: 0 или 1.


\begin{lstlisting}[label=code_syntax_getCheckAllCombinations,caption=Синтаксис]
bool getCheckAllCombinations();
\end{lstlisting}

\textbf{Входные параметры:}

Отсутствуют.

\textbf{Возвращаемое значение:}

Значение переменной из описания.


\subsubsection{getDate}\label{getDate}

Получение текста переменной  XML\_Date - Дата создания документа.


\begin{lstlisting}[label=code_syntax_getDate,caption=Синтаксис]
QString getDate();
\end{lstlisting}

\textbf{Входные параметры:}

Отсутствуют.

\textbf{Возвращаемое значение:}

Значение переменной из описания.


\subsubsection{getDimensionTestFunction}\label{getDimensionTestFunction}

Получение текста переменной  XML\_DimensionTestFunction --- Размерность тестовой задачи.


\begin{lstlisting}[label=code_syntax_getDimensionTestFunction,caption=Синтаксис]
qint64 getDimensionTestFunction();
\end{lstlisting}

\textbf{Входные параметры:}

Отсутствуют.

\textbf{Возвращаемое значение:}

Значение переменной из описания.


\subsubsection{getEmail}\label{getEmail}

Получение текста переменной  XML\_Email - Email автора, чтобы можно было с ним связаться.


\begin{lstlisting}[label=code_syntax_getEmail,caption=Синтаксис]
QString getEmail();
\end{lstlisting}

\textbf{Входные параметры:}

Отсутствуют.

\textbf{Возвращаемое значение:}

Значение переменной из описания.


\subsubsection{getErrorEx}\label{getErrorEx}

Получение значения ошибки Ex.


\begin{lstlisting}[label=code_syntax_getErrorEx,caption=Синтаксис]
double getErrorEx(int Number_Of_Experiment, int Number_Of_Measuring);
\end{lstlisting}

\textbf{Входные параметры:}

Number\_Of\_Experiment --- номер комбинации вариантов настроек;
 
    Number\_Of\_Measuring --- номер измерения варианта настроек.

\textbf{Возвращаемое значение:}

 
Значения ошибки Ex.


\subsubsection{getErrorEy}\label{getErrorEy}

Получение значения ошибки Ey.


\begin{lstlisting}[label=code_syntax_getErrorEy,caption=Синтаксис]
double getErrorEy(int Number_Of_Experiment, int Number_Of_Measuring);
\end{lstlisting}

\textbf{Входные параметры:}

Number\_Of\_Experiment --- номер комбинации вариантов настроек;
 
    Number\_Of\_Measuring --- номер измерения варианта настроек.

\textbf{Возвращаемое значение:}

 
Значения ошибки Ey.


\subsubsection{getErrorR}\label{getErrorR}

Получение значения надежности R.


\begin{lstlisting}[label=code_syntax_getErrorR,caption=Синтаксис]
double getErrorR(int Number_Of_Experiment, int Number_Of_Measuring);
\end{lstlisting}

\textbf{Входные параметры:}

Number\_Of\_Experiment --- номер комбинации вариантов настроек;
 
    Number\_Of\_Measuring --- номер измерения варианта настроек.

\textbf{Возвращаемое значение:}

 
Значения надежности R.


\subsubsection{getFormat}\label{getFormat}

Получение переменной XML\_Format, то есть возвращает название формата документа.


\begin{lstlisting}[label=code_syntax_getFormat,caption=Синтаксис]
QString getFormat();
\end{lstlisting}

\textbf{Входные параметры:}

Отсутствуют.

\textbf{Возвращаемое значение:}

Значение переменной из описания.


\subsubsection{getFullNameAlgorithm}\label{getFullNameAlgorithm}

Получение текста переменной  XML\_Full\_Name\_Algorithm --- Полное название алгоритма оптимизации.


\begin{lstlisting}[label=code_syntax_getFullNameAlgorithm,caption=Синтаксис]
QString getFullNameAlgorithm();
\end{lstlisting}

\textbf{Входные параметры:}

Отсутствуют.

\textbf{Возвращаемое значение:}

Значение переменной из описания.


\subsubsection{getFullNameTestFunction}\label{getFullNameTestFunction}

Получение текста переменной  XML\_Full\_Name\_Test\_Function --- Полное название тестовой функции.


\begin{lstlisting}[label=code_syntax_getFullNameTestFunction,caption=Синтаксис]
QString getFullNameTestFunction();
\end{lstlisting}

\textbf{Входные параметры:}

Отсутствуют.

\textbf{Возвращаемое значение:}

Значение переменной из описания.


\subsubsection{getLink}\label{getLink}

Получение переменной XML\_Link, то есть возвращает ссылку на описание формата файла.


\begin{lstlisting}[label=code_syntax_getLink,caption=Синтаксис]
QString getLink();
\end{lstlisting}

\textbf{Входные параметры:}

Отсутствуют.

\textbf{Возвращаемое значение:}

Значение переменной из описания.


\subsubsection{getLinkAlgorithm}\label{getLinkAlgorithm}

Получение текста переменной  XML\_Link\_Algorithm --- Ссылка на описание алгоритма оптимизации.


\begin{lstlisting}[label=code_syntax_getLinkAlgorithm,caption=Синтаксис]
QString getLinkAlgorithm();
\end{lstlisting}

\textbf{Входные параметры:}

Отсутствуют.

\textbf{Возвращаемое значение:}

Значение переменной из описания.


\subsubsection{getLinkTestFunction}\label{getLinkTestFunction}

Получение текста переменной  XML\_Link\_Test\_Function --- Ссылка на описание тестовой функции.


\begin{lstlisting}[label=code_syntax_getLinkTestFunction,caption=Синтаксис]
QString getLinkTestFunction();
\end{lstlisting}

\textbf{Входные параметры:}

Отсутствуют.

\textbf{Возвращаемое значение:}

Значение переменной из описания.


\subsubsection{getListOfParameterOptions}\label{getListOfParameterOptions}

Получение списка вектора названий вариантов параметров алгоритма оптимизации.


\begin{lstlisting}[label=code_syntax_getListOfParameterOptions,caption=Синтаксис]
QStringList getListOfParameterOptions(int Number_Of_Parameter);
\end{lstlisting}

\textbf{Входные параметры:}

Number\_Of\_Parameter --- номер параметра.

\textbf{Возвращаемое значение:}

Список вектора названий вариантов параметров алгоритма оптимизации.


\subsubsection{getListOfVectorParameterOptions}\label{getListOfVectorParameterOptions}

Получение списка вектора названий вариантов параметров алгоритма оптимизации --- это сборник строк из MatrixOfNameParameters, где объединены столбцы. Получение строки параметров эксперимента из списка вектора названий вариантов параметров алгоритма оптимизации --- это сборник строк из MatrixOfNameParameters, где объединены столбцы.


\begin{lstlisting}[label=code_syntax_getListOfVectorParameterOptions,caption=Синтаксис]
QStringList getListOfVectorParameterOptions();
QString getListOfVectorParameterOptions(int Number_Of_Experiment);
\end{lstlisting}

\textbf{Входные параметры:}

Number\_Of\_Experiment --- номер эксперимента.

\textbf{Возвращаемое значение:}

Значения строки параметров эксперимента.


\subsubsection{getMaxCountOfFitness}\label{getMaxCountOfFitness}

Получение текста переменной  Max\_Count\_Of\_Fitness --- Максимальное допустимое число вычислений целевой функции для алгоритма.


\begin{lstlisting}[label=code_syntax_getMaxCountOfFitness,caption=Синтаксис]
qint64 getMaxCountOfFitness();
\end{lstlisting}

\textbf{Входные параметры:}

Отсутствуют.

\textbf{Возвращаемое значение:}

Значение переменной из описания.


\subsubsection{getMeanEx}\label{getMeanEx}

Получение среднего значения ошибки Ex по измерениям для настройки (сколько точек было по столько и усредняем).


\begin{lstlisting}[label=code_syntax_getMeanEx,caption=Синтаксис]
double getMeanEx(int Number_Of_Experiment);
\end{lstlisting}

\textbf{Входные параметры:}

Number\_Of\_Experiment --- номер комбинации вариантов настроек.

\textbf{Возвращаемое значение:}

Значения среднего значения Ex.


\subsubsection{getMeanEy}\label{getMeanEy}

Получение среднего значения ошибки Ey по измерениям для настройки (сколько точек было по столько и усредняем).


\begin{lstlisting}[label=code_syntax_getMeanEy,caption=Синтаксис]
double getMeanEy(int Number_Of_Experiment);
\end{lstlisting}

\textbf{Входные параметры:}

Number\_Of\_Experiment --- номер комбинации вариантов настроек.

\textbf{Возвращаемое значение:}

Значения среднего значения Ey.


\subsubsection{getMeanOfAllEx}\label{getMeanOfAllEx}

Получение значения переменной MeanOfAllEx --- среднее значение ошибок Ex алгоритма оптимизации по измерениям по всем измерениям вообще


\begin{lstlisting}[label=code_syntax_getMeanOfAllEx,caption=Синтаксис]
double getMeanOfAllEx();
\end{lstlisting}

\textbf{Входные параметры:}

Отсутствуют.

\textbf{Возвращаемое значение:}

Среднее значение ошибок Ex алгоритма оптимизации по измерениям по всем измерениям вообще


\subsubsection{getMeanOfAllEy}\label{getMeanOfAllEy}

Получение значения переменной MeanOfAllEy --- среднее значение ошибок Ey алгоритма оптимизации по измерениям по всем измерениям вообще.


\begin{lstlisting}[label=code_syntax_getMeanOfAllEy,caption=Синтаксис]
double getMeanOfAllEy();
\end{lstlisting}

\textbf{Входные параметры:}

Отсутствуют.

\textbf{Возвращаемое значение:}

Среднее значение ошибок Ey алгоритма оптимизации по измерениям по всем измерениям вообще.


\subsubsection{getMeanOfAllR}\label{getMeanOfAllR}

Получение значения переменной MeanOfAllR --- среднее значение надежностей R алгоритма оптимизации по измерениям по всем измерениям вообще.


\begin{lstlisting}[label=code_syntax_getMeanOfAllR,caption=Синтаксис]
double getMeanOfAllR();
\end{lstlisting}

\textbf{Входные параметры:}

Отсутствуют.

\textbf{Возвращаемое значение:}

Среднее значение надежностей R алгоритма оптимизации по измерениям по всем измерениям вообще


\subsubsection{getMeanR}\label{getMeanR}

Получение среднего значения надежности R по измерениям для настройки (сколько точек было по столько и усредняем).


\begin{lstlisting}[label=code_syntax_getMeanR,caption=Синтаксис]
double getMeanR(int Number_Of_Experiment);
\end{lstlisting}

\textbf{Входные параметры:}

Number\_Of\_Experiment --- номер комбинации вариантов настроек.

\textbf{Возвращаемое значение:}

Значения среднего значения R.


\subsubsection{getNameAlgorithm}\label{getNameAlgorithm}

Получение текста переменной  XML\_Name\_Algorithm - Название алгоритма оптимизации.


\begin{lstlisting}[label=code_syntax_getNameAlgorithm,caption=Синтаксис]
QString getNameAlgorithm();
\end{lstlisting}

\textbf{Входные параметры:}

Отсутствуют.

\textbf{Возвращаемое значение:}

Значение переменной из описания.


\subsubsection{getNameOption}\label{getNameOption}

Получение имени параметра алгоритма по его номеру.


\begin{lstlisting}[label=code_syntax_getNameOption,caption=Синтаксис]
QString getNameOption(int Number_Of_Parameter);
\end{lstlisting}

\textbf{Входные параметры:}

Number\_Of\_Parameter --- номер параметра.

\textbf{Возвращаемое значение:}

Значения параметра в виде наименования.


\subsubsection{getNameParameter}\label{getNameParameter}

Получение значения параметра настройки какой-то в виде полного наименования.


\begin{lstlisting}[label=code_syntax_getNameParameter,caption=Синтаксис]
QString getNameParameter(int Number_Of_Experiment, int Number_Of_Parameter);
\end{lstlisting}

\textbf{Входные параметры:}

Number\_Of\_Experiment --- номер комбинации вариантов настроек;
 
    Number\_Of\_Parameter --- номер параметра.

\textbf{Возвращаемое значение:}

Значения параметра в виде наименования.


\subsubsection{getNameTestFunction}\label{getNameTestFunction}

Получение текста переменной  XML\_Name\_Test\_Function --- Название тестовой функции.


\begin{lstlisting}[label=code_syntax_getNameTestFunction,caption=Синтаксис]
QString getNameTestFunction();
\end{lstlisting}

\textbf{Входные параметры:}

Отсутствуют.

\textbf{Возвращаемое значение:}

Значение переменной из описания.


\subsubsection{getNamesOfParameters}\label{getNamesOfParameters}

Получение списка параметров алгоритма (тип селекции, тип скрещивания).


\begin{lstlisting}[label=code_syntax_getNamesOfParameters,caption=Синтаксис]
QStringList getNamesOfParameters();
\end{lstlisting}

\textbf{Входные параметры:}

Отсутствуют.

\textbf{Возвращаемое значение:}

Список параметров алгоритма.


\subsubsection{getNumberOfExperiments}\label{getNumberOfExperiments}

Получение текста переменной  XML\_Number\_Of\_Experiments --- Количество комбинаций вариантов настроек.


\begin{lstlisting}[label=code_syntax_getNumberOfExperiments,caption=Синтаксис]
qint64 getNumberOfExperiments();
\end{lstlisting}

\textbf{Входные параметры:}

Отсутствуют.

\textbf{Возвращаемое значение:}

Значение переменной из описания.


\subsubsection{getNumberOfMeasuring}\label{getNumberOfMeasuring}

Получение текста переменной  XML\_Number\_Of\_Measuring --- Размерность тестовой задачи (длина хромосомы решения).


\begin{lstlisting}[label=code_syntax_getNumberOfMeasuring,caption=Синтаксис]
qint64 getNumberOfMeasuring();
\end{lstlisting}

\textbf{Входные параметры:}

Отсутствуют.

\textbf{Возвращаемое значение:}

Значение переменной из описания.


\subsubsection{getNumberOfParameters}\label{getNumberOfParameters}

Получение текста переменной  XML\_Number\_Of\_Parameters --- Количество проверяемых параметров алгоритма оптимизации.


\begin{lstlisting}[label=code_syntax_getNumberOfParameters,caption=Синтаксис]
qint64 getNumberOfParameters();
\end{lstlisting}

\textbf{Входные параметры:}

Отсутствуют.

\textbf{Возвращаемое значение:}

Значение переменной из описания.


\subsubsection{getNumberOfRuns}\label{getNumberOfRuns}

Получение текста переменной  XML\_Number\_Of\_Runs --- Количество прогонов по которому делается усреднение для эксперимента.


\begin{lstlisting}[label=code_syntax_getNumberOfRuns,caption=Синтаксис]
qint64 getNumberOfRuns();
\end{lstlisting}

\textbf{Входные параметры:}

Отсутствуют.

\textbf{Возвращаемое значение:}

Значение переменной из описания.


\subsubsection{getOptionFromListOfParameterOptions}\label{getOptionFromListOfParameterOptions}

Получение названия вариантов параметров алгоритма оптимизации.


\begin{lstlisting}[label=code_syntax_getOptionFromListOfParameterOptions,caption=Синтаксис]
QString getOptionFromListOfParameterOptions(int Number_Of_Parameter, int Number_Of_Option);
\end{lstlisting}

\textbf{Входные параметры:}

Number\_Of\_Parameter --- номер параметра.
 
    Number\_Of\_Option --- номер считываемой опции у параметра алгоритма оптимизации.

\textbf{Возвращаемое значение:}

Название вариантов параметров алгоритма оптимизации.


\subsubsection{getOptionFromListOfParameterOptionsForTable}\label{getOptionFromListOfParameterOptionsForTable}

Получение названия вариантов параметров алгоритма оптимизации. Но старается где-то сокращать, а где-то добавлять строки.


\begin{lstlisting}[label=code_syntax_getOptionFromListOfParameterOptionsForTable,caption=Синтаксис]
QString getOptionFromListOfParameterOptionsForTable(int Number_Of_Parameter, int Number_Of_Option);
\end{lstlisting}

\textbf{Входные параметры:}

Number\_Of\_Parameter --- номер параметра.
 
    Number\_Of\_Option --- номер считываемой опции у параметра алгоритма оптимизации.

\textbf{Возвращаемое значение:}

Название вариантов параметров алгоритма оптимизации.


\subsubsection{getParameter}\label{getParameter}

Получение значения параметра настройки какой-то.


\begin{lstlisting}[label=code_syntax_getParameter,caption=Синтаксис]
int getParameter(int Number_Of_Experiment, int Number_Of_Parameter);
\end{lstlisting}

\textbf{Входные параметры:}

Number\_Of\_Experiment --- номер комбинации вариантов настроек;
 
    Number\_Of\_Parameter --- номер параметра.

\textbf{Возвращаемое значение:}

Значения параметра в виде числа (соответствие находим в ListOfParameterOptions).


\subsubsection{getSuccessReading}\label{getSuccessReading}

Получение текста переменной SuccessReading --- Успешно ли прошло считывание.


\begin{lstlisting}[label=code_syntax_getSuccessReading,caption=Синтаксис]
bool getSuccessReading();
\end{lstlisting}

\textbf{Входные параметры:}

Отсутствуют.

\textbf{Возвращаемое значение:}

Значение переменной из описания.


\subsubsection{getVarianceOfAllEx}\label{getVarianceOfAllEx}

Получение значения переменной VarianceOfAllEx --- дисперсия ошибок Ex алгоритма оптимизации по измерениям по всем измерениям вообще.


\begin{lstlisting}[label=code_syntax_getVarianceOfAllEx,caption=Синтаксис]
double getVarianceOfAllEx();
\end{lstlisting}

\textbf{Входные параметры:}

Отсутствуют.

\textbf{Возвращаемое значение:}

Дисперсия ошибок Ex алгоритма оптимизации по измерениям по всем измерениям вообще.


\subsubsection{getVarianceOfAllEy}\label{getVarianceOfAllEy}

Получение значения переменной VarianceOfAllEy --- дисперсия ошибок Ey алгоритма оптимизации по измерениям по всем измерениям вообще


\begin{lstlisting}[label=code_syntax_getVarianceOfAllEy,caption=Синтаксис]
double getVarianceOfAllEy();
\end{lstlisting}

\textbf{Входные параметры:}

Отсутствуют.

\textbf{Возвращаемое значение:}

Дисперсия ошибок Ey алгоритма оптимизации по измерениям по всем измерениям вообще.


\subsubsection{getVarianceOfAllR}\label{getVarianceOfAllR}

Получение значения переменной VarianceOfAllR --- дисперсия надежностей R алгоритма оптимизации по измерениям по всем измерениям вообще.


\begin{lstlisting}[label=code_syntax_getVarianceOfAllR,caption=Синтаксис]
double getVarianceOfAllR();
\end{lstlisting}

\textbf{Входные параметры:}

Отсутствуют.

\textbf{Возвращаемое значение:}

Дисперсия надежностей R алгоритма оптимизации по измерениям по всем измерениям вообще.


\subsubsection{getVarianceOfEx}\label{getVarianceOfEx}

Получение дисперсии значения ошибки Ex по измерениям для настройки (сколько точек было по столько и усредняем).


\begin{lstlisting}[label=code_syntax_getVarianceOfEx,caption=Синтаксис]
double getVarianceOfEx(int Number_Of_Experiment);
\end{lstlisting}

\textbf{Входные параметры:}

Number\_Of\_Experiment --- номер комбинации вариантов настроек.

\textbf{Возвращаемое значение:}

Значения дисперсии значения Ex.


\subsubsection{getVarianceOfEy}\label{getVarianceOfEy}

Получение дисперсии значения ошибки Ey по измерениям для настройки (сколько точек было по столько и усредняем).


\begin{lstlisting}[label=code_syntax_getVarianceOfEy,caption=Синтаксис]
double getVarianceOfEy(int Number_Of_Experiment);
\end{lstlisting}

\textbf{Входные параметры:}

Number\_Of\_Experiment --- номер комбинации вариантов настроек.

\textbf{Возвращаемое значение:}

Значения дисперсии значения Ey.


\subsubsection{getVarianceOfR}\label{getVarianceOfR}

Получение дисперсии значения надежности R по измерениям для настройки (сколько точек было по столько и усредняем).


\begin{lstlisting}[label=code_syntax_getVarianceOfR,caption=Синтаксис]
double getVarianceOfR(int Number_Of_Experiment);
\end{lstlisting}

\textbf{Входные параметры:}

Number\_Of\_Experiment --- номер комбинации вариантов настроек.

\textbf{Возвращаемое значение:}

Значения дисперсии значения надежности R.


\subsubsection{getVersion}\label{getVersion}

Получение переменной Version, то есть возвращает версию формата документа.


\begin{lstlisting}[label=code_syntax_getVersion,caption=Синтаксис]
QString getVersion();
\end{lstlisting}

\textbf{Входные параметры:}

Отсутствуют.

\textbf{Возвращаемое значение:}

Значение переменной из описания.


\subsection{Задание данных в класс}

\subsubsection{addListOfParameterOptions}\label{addListOfParameterOptions}

Добавление списка вектора названий вариантов параметров алгоритма оптимизации.


\begin{lstlisting}[label=code_syntax_addListOfParameterOptions,caption=Синтаксис]
void addListOfParameterOptions(QString Option, int Number_Of_Parameter);
\end{lstlisting}

\textbf{Входные параметры:}

Option --- добавляемое название варианта параметра алгоритма.
 
    Number\_Of\_Parameter --- номер параметра.

\textbf{Возвращаемое значение:}

Отсутствуют.


\subsubsection{addListOfVectorParameterOptions}\label{addListOfVectorParameterOptions}

Добавление строки параметров эксперимента из списка вектора названий вариантов параметров алгоритма оптимизации --- это сборник строк из MatrixOfNameParameters, где объединены столбцы.


\begin{lstlisting}[label=code_syntax_addListOfVectorParameterOptions,caption=Синтаксис]
void addListOfVectorParameterOptions(QString Option);
\end{lstlisting}

\textbf{Входные параметры:}

Option --- добавляемая строка.

\textbf{Возвращаемое значение:}

Отсутствуют.


\subsubsection{addNameOption}\label{addNameOption}

Добавление имени параметра алгоритма.


\begin{lstlisting}[label=code_syntax_addNameOption,caption=Синтаксис]
void addNameOption(QString Option);
\end{lstlisting}

\textbf{Входные параметры:}

Option - имени параметра алгоритма.

\textbf{Возвращаемое значение:}

Отсутствуют.


\subsubsection{addNameParameter}\label{addNameParameter}

Получение значения параметра настройки какой-то в виде полного наименования.


\begin{lstlisting}[label=code_syntax_addNameParameter,caption=Синтаксис]
void addNameParameter(QString Name, int Number_Of_Experiment);
\end{lstlisting}

\textbf{Входные параметры:}

Number\_Of\_Experiment --- номер комбинации вариантов настроек;
 
    Number\_Of\_Parameter --- номер параметра.

\textbf{Возвращаемое значение:}

Отсутствуют.


\subsubsection{setAuthor}\label{setAuthor}

Задание текста переменной XML\_Author - Автор документа.


\begin{lstlisting}[label=code_syntax_setAuthor,caption=Синтаксис]
void setAuthor(QString Author);
\end{lstlisting}

\textbf{Входные параметры:}

Author --- значение переменной из описания.

\textbf{Возвращаемое значение:}

Отсутствуют.


\subsubsection{setCheckAllCombinations}\label{setCheckAllCombinations}

Задание текста переменной  XML\_All\_Combinations --- Все ли комбинации вариантов настроек просмотрены: 0 или 1.


\begin{lstlisting}[label=code_syntax_setCheckAllCombinations,caption=Синтаксис]
void setCheckAllCombinations(bool AllCombinations);
\end{lstlisting}

\textbf{Входные параметры:}

AllCombinations --- Все ли комбинации вариантов настроек просмотрены: 0 или 1.

\textbf{Возвращаемое значение:}

Отсутствуют.


\subsubsection{setDate}\label{setDate}

Задание текста переменной  XML\_Date - Дата создания документа.


\begin{lstlisting}[label=code_syntax_setDate,caption=Синтаксис]
void setDate(QString Date);
\end{lstlisting}

\textbf{Входные параметры:}

Date --- Дата создания документа.

\textbf{Возвращаемое значение:}

Отсутствуют.


\subsubsection{setDimensionTestFunction}\label{setDimensionTestFunction}

Задание текста переменной  XML\_DimensionTestFunction --- Размерность тестовой задачи.


\begin{lstlisting}[label=code_syntax_setDimensionTestFunction,caption=Синтаксис]
void setDimensionTestFunction(qint64 DimensionTestFunction);
\end{lstlisting}

\textbf{Входные параметры:}

DimensionTestFunction --- Размерность тестовой задачи.

\textbf{Возвращаемое значение:}

Отсутствуют.


\subsubsection{setEmail}\label{setEmail}

Задание текста переменной  XML\_Email - Email автора, чтобы можно было с ним связаться


\begin{lstlisting}[label=code_syntax_setEmail,caption=Синтаксис]
void setEmail(QString Email);
\end{lstlisting}

\textbf{Входные параметры:}

Email - Email автора, чтобы можно было с ним связаться.

\textbf{Возвращаемое значение:}

Отсутствуют.


\subsubsection{setErrorEx}\label{setErrorEx}

Задание значения ошибки Ex.


\begin{lstlisting}[label=code_syntax_setErrorEx,caption=Синтаксис]
void setErrorEx(double ErrorEx,int Number_Of_Experiment, int Number_Of_Measuring);
\end{lstlisting}

\textbf{Входные параметры:}

ErrorEx --- задаваемое значение ошибки;
 
    Number\_Of\_Experiment --- номер комбинации вариантов настроек;
 
    Number\_Of\_Measuring --- номер измерения варианта настроек.

\textbf{Возвращаемое значение:}

Отсутствуют.


\subsubsection{setErrorEy}\label{setErrorEy}

Задание значения ошибки Ey.


\begin{lstlisting}[label=code_syntax_setErrorEy,caption=Синтаксис]
void setErrorEy(double ErrorEy,int Number_Of_Experiment, int Number_Of_Measuring);
\end{lstlisting}

\textbf{Входные параметры:}

ErrorEy --- задаваемое значение ошибки;
 
    Number\_Of\_Experiment --- номер комбинации вариантов настроек;
 
    Number\_Of\_Measuring --- номер измерения варианта настроек.

\textbf{Возвращаемое значение:}

Отсутствуют.


\subsubsection{setErrorR}\label{setErrorR}

Задание значения надежности R.


\begin{lstlisting}[label=code_syntax_setErrorR,caption=Синтаксис]
void setErrorR(double ErrorR,int Number_Of_Experiment, int Number_Of_Measuring);
\end{lstlisting}

\textbf{Входные параметры:}

ErrorR --- задаваемое значение надежности;
 
    Number\_Of\_Experiment --- номер комбинации вариантов настроек;
 
    Number\_Of\_Measuring --- номер измерения варианта настроек.

\textbf{Возвращаемое значение:}

Отсутствуют.


\subsubsection{setFormat}\label{setFormat}

Задание переменной XML\_Format --- название формата документа.


\begin{lstlisting}[label=code_syntax_setFormat,caption=Синтаксис]
void setFormat(QString Format);
\end{lstlisting}

\textbf{Входные параметры:}

Format --- название формата документа.

\textbf{Возвращаемое значение:}

Отсутствуют.


\subsubsection{setFullNameAlgorithm}\label{setFullNameAlgorithm}

Задание текста переменной  XML\_Full\_Name\_Algorithm --- Полное название алгоритма оптимизации.


\begin{lstlisting}[label=code_syntax_setFullNameAlgorithm,caption=Синтаксис]
void setFullNameAlgorithm(QString FullNameAlgorithm);
\end{lstlisting}

\textbf{Входные параметры:}

FullNameAlgorithm --- Полное название алгоритма оптимизации.

\textbf{Возвращаемое значение:}

Отсутствуют.


\subsubsection{setFullNameTestFunction}\label{setFullNameTestFunction}

Задание текста переменной  XML\_Full\_Name\_Test\_Function --- Полное название тестовой функции.


\begin{lstlisting}[label=code_syntax_setFullNameTestFunction,caption=Синтаксис]
void setFullNameTestFunction(QString FullNameTestFunction);
\end{lstlisting}

\textbf{Входные параметры:}

FullNameTestFunction --- Полное название тестовой функции.

\textbf{Возвращаемое значение:}

Отсутствуют.


\subsubsection{setLink}\label{setLink}

Задание переменной XML\_Link --- ссылка на описание формата файла.


\begin{lstlisting}[label=code_syntax_setLink,caption=Синтаксис]
void setLink(QString Link);
\end{lstlisting}

\textbf{Входные параметры:}

Link --- ссылка на описание формата файла.

\textbf{Возвращаемое значение:}

Отсутствуют.


\subsubsection{setLinkAlgorithm}\label{setLinkAlgorithm}

Задание текста переменной  XML\_Link\_Algorithm --- Ссылка на описание алгоритма оптимизации.


\begin{lstlisting}[label=code_syntax_setLinkAlgorithm,caption=Синтаксис]
void setLinkAlgorithm(QString LinkAlgorithm);
\end{lstlisting}

\textbf{Входные параметры:}

LinkAlgorithm --- Ссылка на описание алгоритма оптимизации.

\textbf{Возвращаемое значение:}

Отсутствуют.


\subsubsection{setLinkTestFunction}\label{setLinkTestFunction}

Задание текста переменной  XML\_Link\_Test\_Function --- Ссылка на описание тестовой функции.


\begin{lstlisting}[label=code_syntax_setLinkTestFunction,caption=Синтаксис]
void setLinkTestFunction(QString LinkTestFunction);
\end{lstlisting}

\textbf{Входные параметры:}

LinkTestFunction --- Ссылка на описание тестовой функции.

\textbf{Возвращаемое значение:}

Отсутствуют.


\subsubsection{setListOfParameterOptions}\label{setListOfParameterOptions}

Задание списка вектора названий вариантов параметров алгоритма оптимизации.


\begin{lstlisting}[label=code_syntax_setListOfParameterOptions,caption=Синтаксис]
void setListOfParameterOptions(QStringList List, int Number_Of_Parameter);
\end{lstlisting}

\textbf{Входные параметры:}

List --- список названий параметров, которым будем заменять текущий список.
 
    Number\_Of\_Parameter --- номер параметра.

\textbf{Возвращаемое значение:}

Значения параметра в виде наименования.


\subsubsection{setMaxCountOfFitness}\label{setMaxCountOfFitness}

Задание текста переменной  Max\_Count\_Of\_Fitness --- Максимальное допустимое число вычислений целевой функции для алгоритма.


\begin{lstlisting}[label=code_syntax_setMaxCountOfFitness,caption=Синтаксис]
void setMaxCountOfFitness(qint64 MaxCountOfFitness);
\end{lstlisting}

\textbf{Входные параметры:}

MaxCountOfFitness --- Максимальное допустимое число вычислений целевой функции для алгоритма.

\textbf{Возвращаемое значение:}

Отсутствуют.


\subsubsection{setMeanEx}\label{setMeanEx}

Задание среднего значения ошибки Ex по измерениям для настройки (сколько точек было по столько и усредняем).


\begin{lstlisting}[label=code_syntax_setMeanEx,caption=Синтаксис]
void setMeanEx(double MeanEx, int Number_Of_Experiment);
\end{lstlisting}

\textbf{Входные параметры:}

MeanEx --- значение ошибки;
 
    Number\_Of\_Experiment --- номер комбинации вариантов настроек.

\textbf{Возвращаемое значение:}

Отсутствуют.


\subsubsection{setMeanEy}\label{setMeanEy}

Задание среднего значения ошибки Ey по измерениям для настройки (сколько точек было по столько и усредняем).


\begin{lstlisting}[label=code_syntax_setMeanEy,caption=Синтаксис]
void setMeanEy(double MeanEy, int Number_Of_Experiment);
\end{lstlisting}

\textbf{Входные параметры:}

MeanEy --- значение ошибки;
 
    Number\_Of\_Experiment --- номер комбинации вариантов настроек.

\textbf{Возвращаемое значение:}

Отсутствуют.


\subsubsection{setMeanOfAllEx}\label{setMeanOfAllEx}

Задание значения переменной MeanOfAllEx - среднее значение ошибок Ex алгоритма оптимизации по измерениям по всем измерениям вообще


\begin{lstlisting}[label=code_syntax_setMeanOfAllEx,caption=Синтаксис]
void setMeanOfAllEx(double Mean);
\end{lstlisting}

\textbf{Входные параметры:}

Mean --- среднее значение ошибок Ex алгоритма оптимизации по измерениям по всем измерениям вообще.

\textbf{Возвращаемое значение:}

Отсутствуют.


\subsubsection{setMeanOfAllEy}\label{setMeanOfAllEy}

Задание значения переменной MeanOfAllEy - среднее значение ошибок Ey алгоритма оптимизации по измерениям по всем измерениям вообще.


\begin{lstlisting}[label=code_syntax_setMeanOfAllEy,caption=Синтаксис]
void setMeanOfAllEy(double Mean);
\end{lstlisting}

\textbf{Входные параметры:}

Mean --- среднее значение ошибок Ey алгоритма оптимизации по измерениям по всем измерениям вообще.

\textbf{Возвращаемое значение:}

Отсутствуют.


\subsubsection{setMeanOfAllR}\label{setMeanOfAllR}

Задание значения переменной MeanOfAllR --- среднее значение надежностей R алгоритма оптимизации по измерениям по всем измерениям вообще.


\begin{lstlisting}[label=code_syntax_setMeanOfAllR,caption=Синтаксис]
void setMeanOfAllR(double Mean);
\end{lstlisting}

\textbf{Входные параметры:}

Mean --- среднее значение надежностей R алгоритма оптимизации по измерениям по всем измерениям вообще.

\textbf{Возвращаемое значение:}

Отсутствуют.


\subsubsection{setMeanR}\label{setMeanR}

Задание среднего значения надежности R по измерениям для настройки (сколько точек было по столько и усредняем).


\begin{lstlisting}[label=code_syntax_setMeanR,caption=Синтаксис]
void setMeanR(double MeanR, int Number_Of_Experiment);
\end{lstlisting}

\textbf{Входные параметры:}

MeanR --- значение ошибки;
 
    Number\_Of\_Experiment --- номер комбинации вариантов настроек.

\textbf{Возвращаемое значение:}

Отсутствуют.


\subsubsection{setNameAlgorithm}\label{setNameAlgorithm}

Получение текста переменной  XML\_Name\_Algorithm - Название алгоритма оптимизации.


\begin{lstlisting}[label=code_syntax_setNameAlgorithm,caption=Синтаксис]
void setNameAlgorithm(QString NameAlgorithm);
\end{lstlisting}

\textbf{Входные параметры:}

NameAlgorithm --- Название алгоритма оптимизации

\textbf{Возвращаемое значение:}

Отсутствуют.


\subsubsection{setNameTestFunction}\label{setNameTestFunction}

Задание текста переменной  XML\_Name\_Test\_Function --- Название тестовой функции.


\begin{lstlisting}[label=code_syntax_setNameTestFunction,caption=Синтаксис]
void setNameTestFunction(QString NameTestFunction);
\end{lstlisting}

\textbf{Входные параметры:}

NameTestFunction --- Название тестовой функции.

\textbf{Возвращаемое значение:}

Отсутствуют.


\subsubsection{setNumberOfExperiments}\label{setNumberOfExperiments}

Задание текста переменной  XML\_Number\_Of\_Experiments --- Количество комбинаций вариантов настроек.


\begin{lstlisting}[label=code_syntax_setNumberOfExperiments,caption=Синтаксис]
void setNumberOfExperiments(qint64 NumberOfExperiments);
\end{lstlisting}

\textbf{Входные параметры:}

NumberOfExperiments --- Количество комбинаций вариантов настроек.

\textbf{Возвращаемое значение:}

Отсутствуют.


\subsubsection{setNumberOfListOfVectorParameterOptions}\label{setNumberOfListOfVectorParameterOptions}

Задание значения элемента массива NumberOfListOfVectorParameterOptions.


\begin{lstlisting}[label=code_syntax_setNumberOfListOfVectorParameterOptions,caption=Синтаксис]
void setNumberOfListOfVectorParameterOptions(double Num,int Number);
\end{lstlisting}

\textbf{Входные параметры:}

Num --- значение элемента;
 
    Number --- номер элемента.

\textbf{Возвращаемое значение:}

Значения параметра в виде наименования.


\subsubsection{setNumberOfMeasuring}\label{setNumberOfMeasuring}

Задание текста переменной  XML\_Number\_Of\_Measuring --- Размерность тестовой задачи (длина хромосомы решения).


\begin{lstlisting}[label=code_syntax_setNumberOfMeasuring,caption=Синтаксис]
void setNumberOfMeasuring(qint64 NumberOfMeasuring);
\end{lstlisting}

\textbf{Входные параметры:}

NumberOfMeasuring --- Размерность тестовой задачи (длина хромосомы решения).

\textbf{Возвращаемое значение:}

Отсутствуют.


\subsubsection{setNumberOfParameters}\label{setNumberOfParameters}

Задание текста переменной  XML\_Number\_Of\_Parameters --- Количество проверяемых параметров алгоритма оптимизации.


\begin{lstlisting}[label=code_syntax_setNumberOfParameters,caption=Синтаксис]
void setNumberOfParameters(qint64 NumberOfParameters);
\end{lstlisting}

\textbf{Входные параметры:}

NumberOfParameters --- Количество проверяемых параметров алгоритма оптимизации.

\textbf{Возвращаемое значение:}

Отсутствуют.


\subsubsection{setNumberOfRuns}\label{setNumberOfRuns}

Задание текста переменной  XML\_Number\_Of\_Runs --- Количество прогонов по которому делается усреднение для эксперимента.


\begin{lstlisting}[label=code_syntax_setNumberOfRuns,caption=Синтаксис]
void setNumberOfRuns(qint64 NumberOfRuns);
\end{lstlisting}

\textbf{Входные параметры:}

NumberOfRuns --- Количество прогонов по которому делается усреднение для эксперимента.

\textbf{Возвращаемое значение:}

Отсутствуют.


\subsubsection{setParameter}\label{setParameter}

Задание значения параметра настройки какой-то.


\begin{lstlisting}[label=code_syntax_setParameter,caption=Синтаксис]
void setParameter(int Parameter, int Number_Of_Experiment, int Number_Of_Parameter);
\end{lstlisting}

\textbf{Входные параметры:}

	 Parameter --- значение параметра в виде числа;
 
    Number\_Of\_Experiment --- номер комбинации вариантов настроек;
 
    Number\_Of\_Parameter --- номер параметра.

\textbf{Возвращаемое значение:}

Отсутствуют.


\subsubsection{setSuccessReading}\label{setSuccessReading}

Задание текста переменной SuccessReading --- Успешно ли прошло считывание.


\begin{lstlisting}[label=code_syntax_setSuccessReading,caption=Синтаксис]
void setSuccessReading(bool Success);
\end{lstlisting}

\textbf{Входные параметры:}

SuccessReading --- Успешно ли прошло считывание.

\textbf{Возвращаемое значение:}

Отсутствуют.


\subsubsection{setVarianceOfAllEx}\label{setVarianceOfAllEx}

Задание значения переменной VarianceOfAllEx --- дисперсия ошибок Ex алгоритма оптимизации по измерениям по всем измерениям вообще


\begin{lstlisting}[label=code_syntax_setVarianceOfAllEx,caption=Синтаксис]
void setVarianceOfAllEx(double Variance);
\end{lstlisting}

\textbf{Входные параметры:}

Variance --- дисперсия ошибок Ex алгоритма оптимизации по измерениям по всем измерениям вообще.

\textbf{Возвращаемое значение:}

Отсутствуют.


\subsubsection{setVarianceOfAllEy}\label{setVarianceOfAllEy}

Задание значения переменной VarianceOfAllEy --- дисперсия ошибок Ey алгоритма оптимизации по измерениям по всем измерениям вообще.


\begin{lstlisting}[label=code_syntax_setVarianceOfAllEy,caption=Синтаксис]
void setVarianceOfAllEy(double Variance);
\end{lstlisting}

\textbf{Входные параметры:}

Variance --- дисперсия ошибок Ey алгоритма оптимизации по измерениям по всем измерениям вообще.

\textbf{Возвращаемое значение:}

Отсутствуют.


\subsubsection{setVarianceOfAllR}\label{setVarianceOfAllR}

Задание значения переменной VarianceOfAllR --- дисперсия надежностей R алгоритма оптимизации по измерениям по всем измерениям вообще.


\begin{lstlisting}[label=code_syntax_setVarianceOfAllR,caption=Синтаксис]
void setVarianceOfAllR(double Variance);
\end{lstlisting}

\textbf{Входные параметры:}

Variance --- дисперсия надежностей R алгоритма оптимизации по измерениям по всем измерениям вообще.

\textbf{Возвращаемое значение:}

Отсутствуют.


\subsubsection{setVarianceOfEx}\label{setVarianceOfEx}

Получение дисперсии значения ошибки Ex по измерениям для настройки (сколько точек было по столько и усредняем).


\begin{lstlisting}[label=code_syntax_setVarianceOfEx,caption=Синтаксис]
void setVarianceOfEx(double Variance, int Number_Of_Experiment);
\end{lstlisting}

\textbf{Входные параметры:}

	 Variance --- значение заносимой дисперсии;
 
    Number\_Of\_Experiment --- номер комбинации вариантов настроек.

\textbf{Возвращаемое значение:}

Отсутствуют.


\subsubsection{setVarianceOfEy}\label{setVarianceOfEy}

Получение дисперсии значения ошибки Ey по измерениям для настройки (сколько точек было по столько и усредняем).


\begin{lstlisting}[label=code_syntax_setVarianceOfEy,caption=Синтаксис]
void setVarianceOfEy(double Variance, int Number_Of_Experiment);
\end{lstlisting}

\textbf{Входные параметры:}

	 Variance --- значение заносимой дисперсии;
 
    Number\_Of\_Experiment --- номер комбинации вариантов настроек.

\textbf{Возвращаемое значение:}

Отсутствуют.


\subsubsection{setVarianceOfR}\label{setVarianceOfR}

Получение дисперсии значения надежности R по измерениям для настройки (сколько точек было по столько и усредняем).


\begin{lstlisting}[label=code_syntax_setVarianceOfR,caption=Синтаксис]
void setVarianceOfR(double Variance, int Number_Of_Experiment);
\end{lstlisting}

\textbf{Входные параметры:}

	 Variance --- значение заносимой дисперсии;
 
    Number\_Of\_Experiment --- номер комбинации вариантов настроек.

\textbf{Возвращаемое значение:}

Отсутствуют.


\subsubsection{setVersion}\label{setVersion}

Задание переменной XML\_Version --- версия формата документа.


\begin{lstlisting}[label=code_syntax_setVersion,caption=Синтаксис]
void setVersion(QString Version);
\end{lstlisting}

\textbf{Входные параметры:}

Version --- версия формата документа.

\textbf{Возвращаемое значение:}

Отсутствуют.


\subsection{Операторы}

\subsubsection{operator =}\label{operator =}

Оператор присваивания.


\begin{lstlisting}[label=code_syntax_operator =,caption=Синтаксис]
void operator = (HarrixClass_OnlyDataOfHarrixOptimizationTesting& B);
\end{lstlisting}

\textbf{Входные параметры:}

B --- Другой экземпляр класса, который и копируем.

\textbf{Возвращаемое значение:}

Отсутствует.


\subsection{Служебные функции}

\subsubsection{initializationOfVariables}\label{initializationOfVariables}

Обнуление переменных. Внутренняя функция.


\begin{lstlisting}[label=code_syntax_initializationOfVariables,caption=Синтаксис]
void initializationOfVariables();
\end{lstlisting}

\textbf{Входные параметры:}

Отсутствуют.

\textbf{Возвращаемое значение:}

Отсутствует.


\subsubsection{memoryAllocation}\label{memoryAllocation}

Выделяет память под необходимые массивы.


\begin{lstlisting}[label=code_syntax_memoryAllocation,caption=Синтаксис]
void memoryAllocation();
\end{lstlisting}

\textbf{Входные параметры:}

Отсутствуют.

\textbf{Возвращаемое значение:}

Отсутствует.


\subsubsection{memoryDeallocation}\label{memoryDeallocation}

Удаляет память из-под массивов. Внутренняя функция.


\begin{lstlisting}[label=code_syntax_memoryDeallocation,caption=Синтаксис]
void memoryDeallocation();
\end{lstlisting}

\textbf{Входные параметры:}

Отсутствуют.

\textbf{Возвращаемое значение:}

Отсутствует.

%%%%%%%%%%%%%%%%%%%%%%%%%%%%%%%%%%%%%%%%%%%%%%%%%%%%%%%%%%

\end{document}